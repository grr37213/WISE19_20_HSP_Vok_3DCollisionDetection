\label{sec:physical_realism}

Es ist üblicherweise das Ziel von physikalische Simulation einen realen Sachverhalt möglichst akkurat und konform mit dem etablierten physikalischen Verständnis umzusetzen. Diese Anforderungen müssen für den bestehenden Verwendungszweck der Simulation relativiert werden.\\
In Videospielen dient die akkurate Umsetzung nur der Immersion des Konsumenten,~d.h. der Konsument soll der Illusion von absolutem Realismus ausgesetzt sein, deren Bruch nicht kritisch ist, während absoluter Realismus aber teilweise sogar nicht ausreichend umgesetzt werden kann. Die Simulation soll sich dabei nach der Erwartung des Konsumenten verhalten und dieser möglichst nicht zuwiderlaufen.\\
Videospiele sind in dieser Hinsicht sehr vergebend. Zum Vergleich: Simulationen mit zu niedrigen Fehlertoleranzen in z.B.~der Industrierobotik können zu erheblichem Sach- oder gar Personenschaden führen.
Dem Entwickler ist die Aufgabe gestellt, zwischen den Faktoren Performanz, Realismus und Umsetzbarkeit einen Vernünftigen Kompromiss zu finden. Die Unterschiedlichkeit dieser Kompromissfindung kann in~\ref{sec:hitbox} am Besipiel von Minecraft und CSGO betrachtet werden. Realismus ist danach kein ausschlaggebender Faktor für Erfolg.\\

Auch in diesem Projekt müssen die Erwartungen an physikalischen Realismus relativiert werden. Es bestehen Freiheitsgrade in sowohl der Definition der physikalischen Vorgänge als auch der algorithmischen Umsetzung. 
Videospiele, als Kunstform, können so auch surreale Konzepte umsetzen und Echzeitanforderungen können leichter eingehalten werden.

