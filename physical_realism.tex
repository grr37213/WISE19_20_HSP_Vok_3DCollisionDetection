\label{sec:physical_realism}
Es ist weiter üblicherweise ein Ziel von physikalischen Simulationen einen realen Sachverhalt möglichst akkurat und konform mit dem etablierten physikalischen Verständnis umzusetzen. Diese Anforderungen müssen für den bestehenden Verwendungszweck der Simulation relativiert werden.\\
In Videospielen dient die akkurate Umsetzung nur der Immersion des Konsumenten,~d.h. der Konsument soll der Illusion von absolutem Realismus ausgesetzt sein, während absoluter Realismus aber teilweise sogar nicht ausreichend umgesetzt werden kann. Je nach Umstand ist sogar der Bruch der Illusion in Maßen nicht kritisch. Es sollte dem Konsumenten möglich sein mit Hilfe des vorhandenen physikalischen Verständnis ein Verständnis für die Vorgänge der Simulation intuitiv entwickeln zu können.\\
Videospiele sind im Kontext der Simulationsgenauigkeit generell sehr vergebend. Zum Vergleich: Simulationen mit zu niedrigen Fehlertoleranzen in z.B.~der Industrierobotik können zu erheblichem Sach- oder gar Personenschaden führen.
Dem Entwickler ist die Aufgabe gestellt, zwischen den Faktoren Performanz, Realismus und Umsetzbarkeit einen Vernünftigen Kompromiss zu finden. Die Unterschiedlichkeit dieser Kompromissfindung kann in~\ref{sec:hitbox} am Beispiel von Minecraft und CSGO betrachtet werden. Realismus ist danach kein ausschlaggebender Faktor für Erfolg.\\

Auch in diesem Projekt müssen die Erwartungen an physikalischen Realismus relativiert werden. Es bestehen Freiheitsgrade in sowohl der Definition der physikalischen Vorgänge als auch der algorithmischen Umsetzung. 
Videospiele, als Kunstform, können so auch surreale Konzepte umsetzen und Echzeitanforderungen können leichter eingehalten werden.

