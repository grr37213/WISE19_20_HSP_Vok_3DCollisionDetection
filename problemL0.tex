Die Repr"asentation stellt sich dabei jedoch Anforderungen verschiedener technologischer Perspektiven:
		\begin{itemize}
			\item graphische Darstellung/Rendering
				F"ur die graphische Darstellung m"ussen bestimmte Formate eingehalten werden. Datenkonfomit"at mit graphischen Bibliotheken erspart umst"andliche Umwandlungen von Repr"asentationen.
			\item physikalische Berechnung
				F"ur die Kollisionsberechnung k"onnen bestimmte Repr"asentationen vorteilhaft sein. Die Repr"asentation versucht m"oglichst statisch Information "uber Objekte bereitzustellen und vermeidet dynamische Nachberechnung von Objekteigenschaften.
			\item menschliche Manipulation
				F"ur Entwicklungszwecke ist es zum Vorteil, wenn Objektrepr"asentationen menschlich lesbar, verst"andlich und nachvollziehbar sind. Dieses Konzept vereinfacht die Erstellung von Tests und die Behebung von Fehlern.
		\end{itemize}

