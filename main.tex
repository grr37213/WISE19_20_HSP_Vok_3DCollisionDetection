\documentclass[11pt,twoside,a4paper]{article}
\usepackage{german,a4wide,amsmath,amssymb}

% Mann will direkt Umlaute eingeben k�nnen statt \"a, \"o, \"u usw.
% Entweder:
%\usepackage[latin1]{inputenc}
% oder:
%\usepackage{umlaut}


% Trennvorschl"age (in {} einfuegen, wenn nicht automatisch getrennt wird:
% z.B. Authen-ti-ka-tions-sys-tem)
\hyphenation{}

\hyphenation{min-des-tens}


%-------------------------- Formatsachen --------------------------%

% Bild-, Tabellenunterschriften veraendern:
% Nummer fett, kleinerer Text fuer Bildunterschrift
\usepackage[bf,small]{caption}

\usepackage{mathpazo}  % -- Palatino als Zeichensatz -- einfach diese
					   % Zeile auskommentieren, falls nicht installiert
%\usepackage{mathptmx}  % -- Times als Zeichensatz

% Zum Unterscheiden von Entwurfs- und endgueltiger Fassung
%\usepackage{draftcopy}
%\draftcopySetGrey{0.90}   %   90% = sehr helles Grau
%\draftcopyName{ENTWURF}{155}   % statt ``DRAFT''
%\draftcopySetScale{1}

%--------------- Zeilen- und Absatzabstaende ----------------------%
\setlength{\parindent}{0em}
\setlength{\parskip}{\medskipamount}    % Abstand zwischen Abs"atzen

% ---------- Umgebungen f"ur Satz/ Lemma, etc. --------------------%
\newtheorem{satz}{Satz}
\newtheorem{nota}{Notation}
\newtheorem{defi}{Definition}
\newtheorem{kons}{Konstruktion}

\newenvironment{notation}{\noindent \textbf{Notation: }}{}
\newenvironment{beweis}{\noindent \textbf{Beweis: }}{}
\newenvironment{anmerkung}{\noindent \textbf{Anmerkung: }}{}
\newenvironment{anmerkungen}{\noindent \textbf{Anmerkungen: }}{}
\newenvironment{beispiel}{\noindent \textbf{Beispiel: }}{}
\newenvironment{beispiele}{\noindent \textbf{Beispiele: }}{}

% URLs und Mailadressen etc. richtig trennen:
\usepackage{url}
% Auch praktisch fuer Mailadressen: \url{blabla@laberlaber.de}

% --------------------- Eigene Befehle fuer math. Mengen ---------%
\newcommand{\N}{{\rm I\!N}}             % die natuerlichen Zahlen
\newcommand{\Z}{\mathbb{Z}}             % fuer ganze Zahlen
\newcommand{\R}{\mathbb{R}}             % die reellen Zahlen
\newcommand{\Prim}{{\rm I\!P}}          % die Primzahlen


% -- Sinnvolle Befehle, um sich selbst Notizen im Text zu machen --%

% "Ungeordnete Gedanken, die noch irgendwo reinsollen":
\newcommand{\kramsubsection}[1][Unsortierte Textfragmente]{%
\subsection*{#1}%
\addcontentsline{toc}{subsection}{#1}%
}

% Randbemerkung:
\newcommand{\bemerkung}[1]{\marginpar{\small\textsl{\textsf{#1}}}}

% "Hier muss noch [weiter-]geschrieben werden" (Baustellensymbol am Rand)
%
% [Damit dieser Befehl funktioniert, muss man natuerlich erstmal
%  das Icon "Baustelle.eps" besorgen!!  Also entweder selbermachen
%  oder downloaden:
%  http://www.net.in.tum.de/teaching/WS04/routing/Baustelle.eps.gz  ]
\newcommand{\baustelle}[1][]{
 \marginpar{%
   \centerline{\includegraphics[scale=0.3]{Baustelle.eps}}
   {\small\textsl{\textsf{\raggedright #1}}}
}}




\begin{document}

\title{HSP-Projektarbeit im Master Informatik \\
\small Kollisionsdetection in in echtzeit-gerenderten 3D Simulationen}
\author{Robert Graf, Lukas Hermann\\
%  (\texttt{fridolw@in.tum.de})\\[5mm]
%  Seminar "`Internetrouting"' , \\
  Ostbayerische Technische Hochschule Regensburg\\
  \\
  Projektbetreuung: Prof. Dr. Klaus Volbert
}
  
\date{WS\, 2019/2020 (Version vom \today)}


\maketitle


\abstract{Dieses Projekt umfasst die Erstellung einer 3D-Simulation mit dem Fokus auf die Implementierung von Algorithmen zur Kollisionserkennung von rigiden 3D-Objekten. In diesem Bericht werden verschiedene Ansätze behandelt als auch Probleme und Hürden in der Bearbeitung dargestellt.}


\section{Einleitung}

Dieses Dokument dient als Spezifikation für ein 3D-Simulationsprogramm, welches für die Umsetzung diverser Videospielideen konzipiert ist.
Es sollen ausgewählte Aspekte der Simulation deklarativ beschrieben und/oder spezifiziert werden, ohne eine spezifische Umsetzung zu sehr einzuschränken.
Das Dokument kann ebenfalls detaillierte Informationen zur derzeitigen Umsetzung bestimmter Aspekte der Simulation liefern.
Weiter dient es als Leitfaden, in dem relevantes Kontextwissen gesammelt und referenziert ist.

\section{Problemdefinition}
Die 3D-Simulation umfasst hier die Echtzeitsimulation von Entitäten in einem 3D-Kontext. Darin enthalten sind ebenfalls die Aufgaben der Video- und Audioanzeige des Inhalts der Simulation und die Steuerung von bestimmten Inhalten in Echtzeit.\\
Die konkreten Ausmaße des Problems der Erstellung dieser Simulation ist von den Features der zu simulierenden Entitäten abhängig.

Um einen grundlegenden Einblick zu bieten können einige der umgesetzten, bzw. angestrebten Features für diese Simulation beispielhaft beschrieben werden.\\
Klassischerweise im Kontext der 3D-Simulation per se umfasst dies:
\begin{enumerate}
\item Rigide physikalische Objekte
\item passiv physikalische rigide Objektkollision
\end{enumerate}

Im Kontext eines Videospiels können weitere Aspekte hinzugefügt werden. Einige sind dabei abhängig vom jeweiligen zu realisierenden Videospiel:
\begin{enumerate}
\item Projektile
\item Nicht Spieler Charactere (NPC)/Gegner
\item Spieleravatare\\
Von einem Benutzer Steuerbare Entitäten, welche diverse Aktionen ausführen können.
\item Terrain
\item diverse benutzbare Gegenstände/Items und Inventare (Werkzeuge/Waffen)
\end{enumerate}



TODO REMOVE

\begin{defi}[Beispieldefinition]
Gegeben ... wird ... definiert als ..
\end{defi}

TODO REMOVE

\section{Kapitel}

Grr f"uhrt in \cite{grr} das Konzept der Geheimniszerlegung ein 
(Beispielzitat).

Tse f"uhrt in \cite{tse} das Konzept der Geheimniszerlegung ein 
(Beispielzitat).

\section{Zusammenfassung}

Das Kollisionsproblem wird in die Teilprobleme Vorfilterung und modellgenauer Kollision
geteilt. Zur Vorfilterung der Objektpaare werden zwei Algorithmen Algorithmen untersucht, von denen beide für die meisten der Verwendungszwecke eine deutliche Verbesserung gegenüber naiven Ansätzen darstellen. Sie ermöglichen so für Echtzeitfähigkeit eine deutlich höhere Anzahl simulierter Objekte, vor allem wenn Kollisionen nur sporadisch auftreten. Für die meisten Anwendungsfälle scheint die Auswahl der Algorithmen Praxistauglich zu sein.\\
Für die Realisierung von modellgenauen Kollisionen werden zwei Verfahren vorgestellt.
Eines davon ist jedoch nur in einem beschränkten Kontext definiert und daher mit dem anderen nur schwer Vergleichbar. Sie treten letztendlich als Werkzeuge für konkrete Anwendungsfälle der Kollisionserkennung und nicht als allgemeine Lösung des Problems hervor. Insbesondere der GJK-Algorithmus zeigt weiteres Optimierungspotential. Wir erfahren im Kontext des GJK auch, dass, wenn nicht allgemein darauf geachtet wird, numerische Grenzen von herkömmlichen Datentypen, und nattürliche Grenzen, wie das Nyquist-Shannon-Theorem, in den Verfahren schnell erreicht werden können.\\
Bei Verfahren, welche das 3D-Kollisionsproblem von rigiden Objekten im simulierten Raum behandeln, erfahren wir in daher Widerstand bei Steigerung der qualitativen Anforderungen, jedoch kaum bei den Quantitativen.\\
Es werden bestimmte nicht-triviale Schwierigkeiten des Kollisionsproblems erkennbar und so können einige Fehler in Produkten aus der Industrie relativiert werden. Für einige Kritikpunkte, insbesondere hinsichtlich Performance bei mehreren Objekten, kann jedoch auch eine Rechtfertigung der Kritik gezeigt werden.


\newpage

\begin{thebibliography}{12}
%\bibitem[HaKT1 98]{HaKT1 98} \footnote{In die 
%Bibliographie sollte s"amtliche benutzte Literatur 
%rein, auch nicht beim eigenen Vortrag angegebene, aber benutzte Papiere 
%und B\"ucher. Gleichzeitig sollte aber alles in der Literaturliste angegebene
%mindestens einmal im Artikel zitiert werden, sonst nicht auflisten.}
        %Michael Harkavy, J. D. Tygar, Hiroaki Kikuchi: {\sl Multi-round 
        %Anonymous Auction Protocols}; 1st IEEE Workshop on Dependable and 
        %Real-Time E-Commerce Systems, 1998.

%Here go sources, which both grr and hel might need
% for example


@misc{csgoprice,
  title = {{Online-Listung von Preisgeldern im E-Sport auf www.esportsearnings.com},
  howpublished = {\url{https://www.esportsearnings.com/games/245-counter-strike-global-offensive}},
  note = {Accessed: 2020-02-12}
}

\bibitem[3]{buyminecraft}
		Online-Artikel im Gamepedia Minecraft Wiki über den Microsoft Kauf von Mojang
		\url{https://minecraft-de.gamepedia.com/Microsoft-Mojang-Kauf}; Zuletzt aufgerufen: 2020-02-12

\bibitem[4]{skyrimwallglitch}
		Wall-Glitch in The Elder Scrolls V:Skyrim
		\url{https://youtu.be/YpJB7aC2kg0?t=85}; 
		alternativ: \url{https://youtu.be/9ywKvhznAE0?t=75};
		Zuletzt aufgerufen: 2020-02-12

\bibitem[5]{floatdistribution}
		Moler, Cleve; \sl{Floating Point Numbers}; 2014-07-07;
		\url{https://blogs.mathworks.com/cleve/2014/07/07/floating-point-numbers/};
		Zuletzt aufgerufen: 2020-02-15
		
\bibitem[5.1]{floatdistributionexample}
		Lague, Sebastian; \sl{Coding Adventure: Solar System}; 2020-04-17
		\url{https://youtu.be/7axImc1sxa0?t=640};
		Timestamp: 10:40;
		Zuletzt aufgerufen: 2020-08-25

\bibitem[6]{gjk}
		Gilbert E., Johnson D., Keerthi S.;\sl{A Fast Procedure for Computing the Distance Between Complex Objects in Three-Dimesional Space};
		IEEE Journal of Robotics and Automation, Vol. 4, No. 2, Seite 193-203; 1988-04-02

\bibitem[7.1]{gjk-blog}
		Muratoro Casey; \sl{Implementing GJK - 2006}
		2006;
		\url{https://caseymuratori.com/blog_0003}; Zuletzt aufgerufen: 2020-04-20

\bibitem[7.2]{gjk-vid}
		Muratori Casey ; \sl{Implementing GJK - 2006};
		\url{https://www.youtube.com/watch?v=Qupqu1xe7Io} ; Hochgeladen am: 2016-05-12; Zuletzt aufgerufen: 2020-03-04

\bibitem[8]{gdc-physics}
		Catto Erin;  \sl{Physics for Game Programmers; Continuous Collision};
		Game Developers Conference(GDC) 2013;
		\url{https://gdcvault.com/play/1017644/Physics-for-Game-Programmers-Continuous}; Zuletzt aufgerufen: 2020-04-20;
		\url{https://www.youtube.com/watch?v=7_nKOET6zwI&t=3097s}; Zuletzt aufgerufen: 2020-04-20
		
\bibitem[9]{convex_part_0}
		D. Artigas , S. Dantas , M.C. Douradod, J.L. Szwarcfiter; \sl{Partitioning a graph into convex sets};
		Discrete Mathematics 311 (2011) 1968–1977;

\bibitem[TseDef\_80]{tse}
        Seitz Default: {\sl How to Cite something 2}; 
        Communications of the ACM 22/11 (1979), S. 612-613.

\end{thebibliography}
\end{document}





