\documentclass[11pt,twoside,a4paper]{article}

\usepackage{a4wide,amsmath,amssymb}

% Mann will direkt Umlaute eingeben können statt \"a, \"o, \"u usw.
% Entweder:
\usepackage[utf8]{inputenc}
% oder:
%\usepackage{umlaut}
\usepackage[german]{babel}

\usepackage{textcomp}
\usepackage{graphicx}
\usepackage{subcaption}

\usepackage{hyperref}


% Trennvorschl"age (in {} einfuegen, wenn nicht automatisch getrennt wird:
% z.B. Authen-ti-ka-tions-sys-tem)
%\hyphenation{}

%\hyphenation{min-des-tens}
%\hyphenation{Kol-li-sions-er-ken-nung}


%-------------------------- Formatsachen --------------------------%

% Bild-, Tabellenunterschriften veraendern:
% Nummer fett, kleinerer Text fuer Bildunterschrift
%\usepackage[bf,small]{caption}


%\usepackage{mathpazo}  % -- Palatino als Zeichensatz -- einfach diese
					   % Zeile auskommentieren, falls nicht installiert
%\usepackage{mathptmx}  % -- Times als Zeichensatz

% Zum Unterscheiden von Entwurfs- und endgueltiger Fassung
%\usepackage{draftcopy}
%\draftcopySetGrey{0.90}   %   90% = sehr helles Grau
%\draftcopyName{ENTWURF}{155}   % statt ``DRAFT''
%\draftcopySetScale{1}

%--------------- Zeilen- und Absatzabstaende ----------------------%
%\setlength{\parindent}{0em}
%\setlength{\parskip}{\medskipamount}    % Abstand zwischen Abs"atzen

\begin{document}

\title{HSP-Projektarbeit im Master Informatik \\
\small Kollisionsdetektion in in Echtzeit gerenderten 3D-Simulationen mit Fokus auf die Verwendung in Videospielen}
\author{Robert Graf, Lukas Hermann\\
%  (\texttt{fridolw@in.tum.de})\\[5mm]
%  Seminar "`Internetrouting"' , \\
  Ostbayerische Technische Hochschule Regensburg\\
  \\
  Projektbetreuung: Prof. Dr. Klaus Volbert
}
  
\date{WS\, 2019/2020 (Version vom \today)}


\maketitle

\newpage
\tableofcontents


\abstract{Das Projekt umfasst die Erstellung einer 3D Simulation, so wie die Implementierung von Algorithmen zur Kollisionserkennung von 3D Objekten. In diesem Bericht werden vollzogene T"atigkeiten und H"urden dargestellt.}


\section{Einleitung}

3D Kollisionserkennung und -behandlung wird in Bereichen wie z.B. der Robotik, Fabrikation, Animation oder Echtzeitgraphik ben"otigt.
Insbesondere die Unterhaltungsindustrie im Bereich der Videospiele sieht sich sehr oft dem Problem der Kollisionserkennung und -behandlung gegen"ubergestellt, um die Simulation physikalischer Prozesse, oder die Illusion davon, in ihren Produkten zu erzeugen.
Oft jedoch scheint eine gewisse Diskrepanz zwischen der Erwartung der Konsumenten und der Umsetzung im Produkt zu bestehen. Physikalische Prozesse, insbesondere Kollision, scheint oft nicht akkurat Umgesetzt zu werden. Die Konsequenz daraus sind Bugs, Glitches und inimmersives Verhalten.
Da das Problem der 3D Kollision von der Wissenschaft schon seit einiger Zeit gut verstanden scheint, erscheint es umso merkw"urdiger, dass eine Milliardenindustrie an dieser Stelle immernoch Abstriche in der Entwicklung zu machen scheint.
Um Gr"unde hierf"ur herauszufinden wird in diesem Projekt versucht eine Echtzeit-3D-Simulationsumgebung zu erstellen, die Kollisionen von 3D-Objekten miteinander erkennt.
Es wird sich erhofft dabei die generellen, unoffensichtlichen H"urden zu erkennen, mit denen die Industrie zu k"ampfen hat.


\section{Kontext \& Terminologie}
Das L0-Level (siehe \ref{l0}) beschäftigt sich mit der Datenrepräsentation von Modellen, Zeit und Raum.\\

\subsubsection{Zeit}
Es existieren die Realzeit der echten Welt und die Simulationszeit. Die Zeiten können prinzipiell asynchron ablaufen.\\
In einer Echzeitsimulation müssen beide zeiten jedoch synchronisiert werden.\\
Es genügt dabei, diese Synchronisation in kurzen Zeitabschnitten herzustellen.\\
Diese zeitlichen Abstände werden oft Ticks genannt (siehe \ref{sec:tick}).\\
Gefordert sind hierbei Tickraten von ca. $60 Ticks/s \Rightarrow 16.6ms /Tick$.\\
Realzeit wird in Microsekunden-Genauigkeit vom ausführenden Betriebssystem zu Beginn jedes Ticks erhalten. Das Intervall der vergangenen Simulationszeit kann so durch den Abgleich mit der erhaltenen Zeit des vorherigen Ticks errechnet werden. Dieser Anzahl Microsekunden wird dann in einen Floating-Point-Wert in Sekunden umgewandelt, welche als Zeitfaktor in physikalischen Berechnungen verwendet werden kann.

\subsubsection{Raum}
Der geforderte Raum ist 3-dimensional. Die Darstellung der Positionen im Raum erfolgt über 3-dimensionale Vektoren in der Einheit von Metern.\\
Es wird daher ein Datentyp für Dezimalbrüche verwendet um kleinere Raumanteile zu erfassen.
Floating-Point-Dezimalbrüche würden sich anbieten, jedoch tritt für große Räume ein Genauigkeitsproblem auf Grund der Werteverteilung in Floating-Point Datentypen auf.\\
Sei $\mathbb{F} \subset \mathbb{R}$ mit $\mathbb{F}$ als Floating-Point-Datentyp, so ist die Verteilung der verfügbaren Werte des Floats dichter je näher am Ursprung($0.0$) \cite{floatdistribution}.\\
Für eine Darstellung im Raum $\mathbb{F}^3$ existiert dabei das selbe Problem in 3 Dimensionen. Physikalische Prozesse können daher inkonsistent in Abhängigkeit zum Ort im Raum sein.
Es wird allerdings an dieser stelle Konsistenz der simulierten Prozesse unabhängig vom Ort im Raum gefordert.\\
Lösungen des Problems sind
\begin{enumerate}
	\item Relativierung interagierender Objekte zueinander.\\
		Funktioniert unter der Annahme, dass Entfernungen zwischen interagierenden Objekten gegenüber der Gesamtgröße des Raums relativ klein sind.
	\item Aufteilung des Raums und Positionswerte von Objekten relativiert zu einem nahen Raumanteilsursprung\\
		Sorgt dafür das Objektpositionswerte nicht dem Ungenauigkeitsproblem verfallen, wenn Objekte sich weit vom Raumursprung befinden. Dafür muss ein Positionsdatentyp $\mathbb{S}$ definiert werden, welcher den betreffenden Raumanteil pro Positionswert mitführt $\mathbb{S}:\mathbb{Z}\times\mathbb{F}$.
\end{enumerate}

Der Raum wird demnach mit Vektoren $s\in\mathbb{S}^3$ dargestellt.

\subsubsection{Objekte}
Über die Grafikbibliothek OpenGL können Objekte graphisch dargestellt werden. Die Bibliothek ermöglicht die Darstellung folgendermaßen:\\
Es wird eine Ansammlung an Ecken (eng. vertices) über 3D-Vectoren (32Bit-floating-point) gegeben. Weiter eine Ansammlung von 3er Gruppen an Integers zu der Eckenansammlung um Dreiecke zu spezifizieren. Die Dreiecke Bilden dann den Körper.\\
Da diese Art der Repräsentation von OpenGL auf diese Weise im allgemeinen verwendet wird und von der Seite der Physiksimulation keine konkreten Anforderungen gestellt sind, wurde entschieden die Objektrepresentationen gleich zu halten, um dynamische Unformungen zwischen Repräsentationen zu vermeiden.\\
Jedes Objekt hat einen eigenen Ursprung, auf den sich die Vertexdaten beziehen.\\
\\
Im Folgenden werden weitere Attribute von Objekten aufgezählt. Dabei werden verschiedene Arten der Objektrepräsentation dargestellt.
\begin{itemize}
	\item[R0]
		\begin{itemize}i
			\item Ein Positionswert $p \in \mathbb{S}^3$, der den Objektursprung zum Raumursprung absolut beschreibt
			\item Geschwindigkeitswert $v \in \mathbb{S}^3$ in Form eines Vektors, der die lineare Bewegung im Raum darstellen kann.
		\end{itemize}
		Diese Repräsentation ist für uns die Basis. Es gibt auch Entitäten in der Spielesimulation, welche keine 3 Dimensionalen Objekte sind (zum Beispiel Geräusche, unter Umständen sogar Projektile, welche durch einen Punkt dargestellt werden). Diese besitzen daher nur Position und Geschwindigkeit.
		Für 3-Dimensionale Physiksimulation mit 3D Modellen ist das jedoch ungenügend, da Objekte sich beispielsweise nicht drehen, sondern in einer Konkreten ausrichtung verharren.
	\item[R1]
		\begin{itemize}
			\item Inklusive R0
			\item Ausrichtung des Objektes im Raum (Rotation)
			\item Rotationsgeschwindigkeit (Tickweise Rotation)
			\item Skalierung
			\item Sklaierungsgeschwindigkeit (Wachsen \& Schrumpfen ; Tickweise Skalierung)
		\end{itemize}

	\item[R2]
		\begin{itemize}
			\item Animation des Objektes
			\item Inklusive R0, unter Umständen auch R1 als Fallback
		\end{itemize}
\end{itemize}

Mit R1 und R2 lassen sich schon akzeptablere Simulationen erstellen.
	


\subsection{Tick}

Die Simulation behandelt das Verstreichen von Zeit in Zeitschritten, während dem der interne Zustand der Simulation, bzw. der simulierten Objekte, zu einem zeitlich neuen Zustand aktualisiert wird.
Dieser Zeitschritt wird oft als Tick bezeichnet.\\
Es ist besonders anzumerken, das der Begriff des Ticks sich ausschließlich auf das Voranschreiten der Simulation bezieht und nicht dem Anzeigen einer Szene. Die Äquivalente Bezeichnung im Kontext der Grafik wird als Frame bezeichnet, in welchem eine Szene (ein Grafischer Zustand der Simulation) gerendert wird. Es besteht Verwechslungsgefahr. Von beiden Größen können Raten $r_{tick}, r_{frame}$ angegeben werden (üblicherweise in $\frac{1}{s}$). Praktisch kann eine Grafikengine durch Inter- oder Extrapolation höhere Frameraten relativ zur Tickrate annehmen ($t_{frame}\geq t_{tick}$).\\
Wir definieren die Menge der Ticks $\delta:T_r^2; +\delta:={\delta_1, \delta_2, ...}$ anhand ihrer Start- und Endzeitpunkte in Echtzeit $\delta_i := (\delta_{i0}, \delta_{i1})$ und erweitert die zu einem Tick gehörenden Zeitpunkte als $\delta_{id}; d \in ]0,1]$.\\
Es gilt außerdem die Kontinuität der Zeit auch bei Ticks $\delta_{j1} + \epsilon_t = \delta_{(j+1)0}$\\
Durch die Abbildung $\mathcal{T}$ erhält der Tick eine Entsprechung in Simulationszeit.\\
Ist im aktuellen Kontext nur ein Tick $i$ von belang wird auch die Terminologie $t_d = \delta_{id}$, also $t_0$ für den Tickbeginn und $t_1$ für das Ende, verwendet.\\
Man kann weiter die Sequenz $\Upsilon_{\delta i} = \langle t_0, ...,  t_1\rangle$ als die zusammenhängende Partition der Simulationszeitsequenz $T_s$ denotieren, welche die Zeitpunkte eines Ticks in Simulationszeit darstellt.
Die Größe der Zeitdifferenz $t_1 - t_0$ unterliegt meist Einschränkungen. Bestimmte Simulationsalgorithmen wie z.B. die Methode der kleinen Schritte erfordert für eine bestimmte Genauigkeit eine maximale Schrittgröße. Die verfügbare Rechenleistung hingegen beschränkt die Tickrate nach oben. Reicht die Berechnungszeit während eine Ticks nicht um den Status der Simulation von $t_0$ auf $t_1$ zu aktualisieren, läuft die Simulation langsamer als die reale Zeit. Die Echtzeitanforderung ist dann verletzt. Oft wird die Tickrate als Konstante festgelegt, in diesem Projekt ist jedoch nur eine Mindestrate festgelegt.

\subsection{Hitbox}
\label{sec:hitbox}
Hitboxen sind Approximationen für Modelle in einer Simulation. Sie ersetzen das konkrete Modell dabei für Kollisionen gegenüber der Kollisionserkennung komplett.\\
Der Term Hitbox suggeriert die verwendung einer Box/eines Quaders zur Approximation. Das ist historisch bedingt. Der Term ist allerdings auch für andere Approximationsformen etabliert.\\
Die Diskrepanz zwischen Hitbox und Model wirkt sich negativ auf den physikalischen Realismus aus. Trotzdem wird von Entwicklern so viel wie kontextuell vertretbar approximiert, um Rechenzeit zu sparen und Echtzeitanforderungen zu genügen.

\begin{figure*}
	\begin{subfigure}[t]{0.45\textwidth}
		\centering
		\includegraphics[width=1\textwidth]{./res/csgo_hitbox.png}
		\caption{Hitbox des Spieler-Modells aus dem Videospiel Counter Strike: Global Offensive; sichtbares Modell(links), mit eingeblendeter Hitbox (rechts)}
%%TODO source for pic
		\label{fig:chitbox}
	\end{subfigure}
~
	\begin{subfigure}[t]{0.2\textwidth}
		\centering
		\includegraphics[width=1\textwidth]{./res/pig_hitbox.png}
		\caption{Hitbox eines NPC-Modells (Schwein) aus dem Videospiel Minecraft; Hitbox in weiß}
		\label{fig:mphitbox}
	\end{subfigure}
~
	\begin{subfigure}[t]{0.2\textwidth}
		\centering
		\includegraphics[width=1\textwidth]{./res/wither_hitbox.png}
		\caption{Hitbox eines NPC-Modells (Wither) aus dem Videospiel Minecraft; Hitbox in weiß}
		\label{fig:mwhitbox}
	\end{subfigure}

	\caption{Güten von Hitboxen}
	\label{fig:hitbox}
\end{figure*}

Die Abbildungen~\ref{fig:hitbox} zeigen Hitboxen in 2 verschiedenen Spielen.\\
\ref{fig:chitbox} zeigt das Spielermodell aus dem Spiel Counter-Strike: Global Offensive (CSGO). Links ist dabei das sichtbare Modell zu sehen, während rechts die Hitboxen eingeblendet sind. Die Hitboxen, welche hier nichmal mehr Boxen sind, sondern Ellipsoiden, decken das sichtbare Modell relativ genau ab. Ebenfalls zu erkennen ist die Partition in einzelne Hitboxen, zu sehen an den verschiedenen Farben der Hitboxen im rechten Bild.\\
Einzelne Details des Spielermodells, wie Riemen und Taschen an der Ausrüsung, sind nicht essentiell und werden daher auch physikalisch nicht abgebildet.\\
\ref{fig:mphitbox} und \ref{fig:mwhitbox} zeigen Hitboxen aus dem Spiel Minecraft bei zwei NPCs (Non-Player-Character (zu Deutsch: Nicht-Spieler-Charakter)). Es ist dort klar zu erkennen, dass die Hitboxen nicht sehr genau mit dem sichtbaren Modell übereinstimmen. Mehr noch: Die Minecraft-Hitboxen sind Koordinatenachsenparallel, d.h. Kanten verlaufen immer entlang der Koordinatenachsen der Raumrepräsentation.\\
Es wird versucht die Unterschiede zu rechtfertigen:\\
CSGO ist ein Shooter. Schnelle Reaktion und genaues Zielen sind ein hauptbestandteil des Produkts. Zudem ist CSGO ein hochkompetitiver E-Sport, der professionell gespielt wird. Es geht dabei um Preisgelder im siebenstelligen Bereich \cite{csgoprice}. Akkurate und, aus der perspektive des Spielers deterministische Hitboxen sind daher essenziell für das Produkt.\\
Die Partition der Hitboxen in CSGO ergibt sich direkt aus einer Anforderung der Anwendung, Schusstreffer auf verschiedene Teile des Spielermodells unterschiedlich zu bewerten. Beispielsweise verursacht der Treffer am Kopf am meisten Schaden. CSGO modelliert die unterschiedlichen Treffbaren teile des Modells also über mehrere Hitboxen.\\
Minecraft ist ein Sandbox Aufbauspiel. Ziel des Spiels ist der Bau von beliebigen Gebäuden, Tunneln, die Kreation von Maschinen oder das Erkunden von Gebieten.\\
In Minecraft steckt auch eine erhebliche Summe Geld. Am 15. September 2014 kaufte Microsoft die Entwicklerfirma und die rechte am Spiel für ca. 2,5 Milliarden Dollar \cite{buyminecraft}.\\
Das Kampfsystem in Minecraft forciert keine schnellen und genauen Treffer auf Gegner. Die gesamte Spielwelt ist aus sichtbaren Achsenparallel aufgestellten Würfeln gegeben, sind also Deckungsgleich mit entsprechenden achsenparallelen Hitboxen. Minecraft macht es sich einfach, da keine konkreten Anforderungen hinsichtlich Genauigkeit bestehen. Tatsächlich wird eine künstlich kleinere Hitbox manchmal sogar eingesetzt um einen Treffer zu erschweren (vgl. Abbildung \ref{fig::mwhitbox}).\\


\subsection{Double Buffering}
\label{sec:doublebuffer}
Beim grafischen Rendern einer Szene werden übicher weise mehrere (hier 2) Buffer angelegt, in welche die Szene gezeichnet wird.\\
Dabei ist ein Buffer der Renderbuffer, zu dem aktiv gezeichnet wird,\\
der andere ist der Displaybuffer, welcher zur Anzeige aktivgeschaltet ist und dann in einem Fenster angezeigt werden kann.\\
Nach einem volzogenen Rendering-Vorgang werden die Rollen der Buffer vertauscht und der nächste Rendering-Schritt durchgeführt. Dieser Zyklus wird Grafiktick genannt (siehe \ref{sec:tick})\\
Dieses Verfahren wird verwendet um Race-Conditions auf den Buffer zu vermeiden, was sichtabre flickernde Fragmente im Ausgabebild erzeugen kann.


\section{Verwendungszwecke}

Es gibt in Videospielen mehrere Verwendungszwecke für Kollisionserkennung in der Simulation.
Einige Beispiele sind:
\begin{enumerate}
	\item physikalische Simulation\\
		Objekte, die sich gegenseitig durch Kollision beeinflussen.
	\item Ermittlung eines Fokusobjektes\\
		In 3D-Spielen wird oft ein Fadenkreuz in die Mitte des Bildes gelegt, welches als Auswahlwerkzeug dient. Um dies zu realisieren muss ermittelt werden, auf welches Objekt das Fadenkreuz zeigt. Das ist beispielsweise durch die Kollision mit einem Strahl in Blickrichtung zu bewerkstelligen.
	\item Relative Interaktion\\
		Beispiel: In einem bestimmten Bereich um den Spieler werden Gegenstände ins Inventar des Spielers gebracht. Realisierbar durch Kollidierenden Körper um den Spieler. Wenn Kollision, dann Aufnahme ins inventar.
	\item Objektplazierung
		Unter Gravitation: Objekte sollen auf den Boden plaziert werden. Ermittlung der Höhe.
\end{enumerate}

Die verschiedenen Verwendungszwecke haben unterschiedliche Anwendungen an Kollisionlogik in Punkten Kollisionsgenauigkeit und welche Informationen zur Kollision benötigt werden.\\
Beispielsweise: Für die Ermittlung des Fokusobjektes soll das Objekt der Kollision ermittelt werden. Der Konkrete treffpunkt oder die Auftrittszeit innerhalb des Ticks ist für diese Aufgabe irrelvant.\\
\\
Anforderungen an die Kollisionserkennung entstehen daher aus vielen Szenarien. Kollisionserkennung ist demnach nicht nur ein einziges Verfahren im System. Kollisionserkennung umfasst auch den Kontext und die Auswahl des passenden Verfahrens mit Blick auf Performanz und Genauigkeit.


\section{Problemdefinition}
Kollisionserkennung (\textit{eng. collision detection}) ist die Erkennung der Überschneidung von zwei aus $k\in\mathbb{N}$ sich bewegenden Objekten in einem Raum.
Gesucht wird dabei der gezeitete Übergang vom Zustand der Nicht-Kollision zur Kollision.\\
Es gilt außerdem die Echtzeitanforderung (vgl.~Abschnitt \ref{sec:tick}).
\\
\subsection{Problemgrenzen}
Kollisionserkennung scheint, je nach Anwendungsfall, ein spezifisches Problem zu sein. Die anerkannte Terminologie für Kollisionserkennung scheint semantisch nicht eindeutig. Daher müssen für diese Studie klare Problemgrenzen und spezifische Begriffe definiert werden.\\
\\
Die erste Grenze ist temporal. Ein Tick (vgl.~Abschnitt \ref{sec:tick}) grenzt einen Simulationsschritt ein. Das Problem kann dabei auf die Abhandlung von einzelnen Simulationschritten beschränkt werden. Algorithmen müssen daher dynamisch bezüglich der Zeitgrenzen sein (Tickintervallgröße nicht konstant), diese dynamischen Grenzen aber nicht überschreiten können.\\
\\
Weiter werden verschiedene Level des Problems definiert:
\begin{itemize}
	\label{l0}
	\item[L0] Physikalische Repräsentation\\
		Überlegungen auf dieser Ebene behandelt die Auswahl möglicher mathematisch-technologischer Repräsentationen der Kollisionsphysik, welche wiederum Repräsentationen des Raums, der Zeit und der zu kollidierenden Objekte fordert.\\
		In dieser Studie wird sich damit nicht aktiv beschäftigt. Eine Repräsenation ist nötig und kommt daher passiv durch die Anforderungen der verwendeten Algorithmen und Technologien zu Stande.\\
			
	\label{l1}
	\item[L1] Suchraumfilter\\
		Annahme: Kollisionen finden zwischen Paaren von Objekten statt.
		Sei die Menge der Objekte im Raum $k\in\mathbb{N}$ so ist die Menge der Paare von Objekten, und damit die Menge möglicher Kollisionspaare, $k^2$.\\
		Bei naiver Überprüfung jedes Paares beträgt die Komplexität in dieser Ebene demnach $O(k^2)$ (vgl. \cite[Abschn. 2]{cd2D}).\\
		Verbesserungen dieser Komplexität können jedoch durch Vorfilterung des Suchraumes erreicht werden.\\
		Ziel ist dieses Levels ist es, eine Menge von Objektpaaren mit minimaler Anzahl von False-Positives zu ermitteln.

	\label{l2}
	\item[L2] Paarweise Kollisionserkennung\\
		Aus \ref{l1} vorliegende Objektepaare müssen auf tatsächliche Modellkollision Überprüft werden, da in L1 noch False-Positives enthalten sein können.
		Aus dieser genaueren Modellkollision können meist auch zusätzliche Informationen über die Kollision ermittelt werden (genaue Zeit, Ort, beteiligte Objektmerkmale), welche für die Kollisionsantwort benötigt werden.

	\label{l3}
	\item[L3] Kollisionsantwort/-auflösung\\
		In einer physikalischen Simulation hat eine Kollision meist eine physikalische Konsequenz, z.B. Geschwindigkeitsveränderung oder Verformung von Objekten. Die Konsequenz betrifft dabei nicht nur das beteiligte Objekt, sondern auch den gesamten Simulationsverlauf (z.B.~neue Kollisionsmöglichkeiten nach Kursänderung eines Objektes).\\
		Die L3-Schicht befasst sich mit der Umsetzung der Konsequenzen einer Kollision.
		Obwohl sich in dieser Arbeit mit den Konsequenzen der Kollision eigentlich nicht beschäftigt werden soll, muss hierzu ein Kontext hergestellt werden, vor allem um Anforderungen an andere Schichten zu ermitteln.
\end{itemize}





\section{L1 : Suchraumverkleinerung/Kollisionspaarermittlung}
%GRR read:
% I mark my annotations with these GRR comments. Not in text.
% To make the text readable in texmaker editor (comments are grey, text is a lighter grey, yay!) I recommend changing color settings of your editor to make comments brighter and a color
% I also have taken the liberty of making some changes myself, when I
% * was at least 95% sure I would not change a meaning
% * only to improve meaning, and compactness
% * to apply mathematics after already established conventions
% All other things are added as comments, especially, when I am not sufficiently sure it would break established meaning
% All comments are recommendations, do as you please.
% Regards, Meph

Da die exakte Berechnung einer Kollision zwischen zwei Objekten $o_0, o_1 \in \obj$ signifikant viel Rechenleistung benötigt //TODO (Test und referenz auf l2), ist eine Vorfilterung der möglichen Objektpaare sinnvoll, die die meisten Paare schon vorher ausschließt.\\
Zu diesem Zweck werden zu jedem Objekt Hüllkörper angegeben, die die Objekte durch die Eigenschaft $B_o \supseteq K_o$ abstrahieren, da gilt $K_{o0} \cap K_{o1} \Rightarrow B_{o0} \cap B_{o1}$, d.h.~ die Menge aller Paare mit exakter Kollision ist in jedem Fall in der Menge aller Paare enthalten $ \{ (o_0, o_1) | B_{o0} \cap B_{o1} \} \supseteq \{ (o_0, o_1) | K_{o0} \cap K_{o1} \} $.
Die weiteren Eigenschaften der Hüllkörper sollen den Filterungsprozess erheblich vereinfachen, da Kollisionen unter diesen u.U. leichter zu errechnen sind. Hier wird als Hüllkörper eine AABB verwendet, deren hier praktische Eigenschaften in Abschnitt \ref{sec:AABB} beschrieben sind.\\
Im Kontext des gesamten Abschnitts \ref{sec:l1} wird als Abstraktionsgrad für ein Objekt solch eine AABB verwendet. Für bestimmte Zwecke, meist logische Kollisionen (vgl. \ref{sec:usages}), ist diese abstrakte Kollision sogar schon ausreichend, um eine Reaktion einzuleiten.\\
Um verschiedene Arten der Kollision behandeln zu können und die Objektpaare dem richtigen Algorithmus für die nächste Stufe zuführen zu können, wurde im Rahmen des Projekts eine Komponente erstellt, die dieses Problem mit austauschbarem Vorfilterungsalgorithmus löst. (Ungenügende Ansätze der Behandlung verschiedener Kollisionstypen unter Verwendung eines nicht austauschbaren Vorfilterungsalgorithmus haben schon vor dem Projektstart in der Codebasis existiert). \\
Logische und physikalische Kollisionen werden als (lokale) Interaktionen zusammengefasst und werden in 2 Kategorien eingeteilt:
\begin{enumerate}
\item Symmetrische Interaktionen:\\
Beide Objekte des Objektpaares gehören der selben Klasse an (z.B: die Kollision rigider Objekte, wie sie in Abschnitt~\ref{sec:l2} beschrieben wird).
\item Asymmetrische Interaktionen:\\
Objektpaare setzen sich aus 2 Objekten verschiedener Klassen zusammen (z.B. Projektil und treffbares Objekt). Kollisionen innerhalb der selben Klasse treten hier nicht auf (Projektile interagieren nicht mit anderen Projektilen).
\end{enumerate}

In jeder o.g. Kategorie können mehrere Interaktionstypen enthalten sein. Beispiele für 2 asymmetrische Interaktionstypen sind Interaktion von Projektil mit treffbarem Objekt oder Trigger-Bereich mit triggerndem Objekt. Jeder Typ wird hier einzeln betrachtet und ihm kann so ein eigener Vorfilterungsalgorithmus zugewiesen werden, der entsprechend zugeschnitten und problemspezifisch optimiert werden kann.
Die Aufgabe des Vorfilterungsalgorithmus unterscheidet sich je nach Kategorie der Interaktion etwas:
\begin{enumerate}
\item Symmetrische Interaktionen: Aus einer Menge von Objekten $\obj$ werden Objektpaare ausgewählt, deren AABBs sich überschneiden. Ein beliebiges der beiden Objekte muss dann über die Interaktion mit dem Partner informiert werden, d.h. die Objekte sind vertauschbar.
Als untere Schranke kann man $\Omega (|\obj|+|C|)$ angeben, wobei $|C|$ die Anzahl der tatsächlichen Überschneidungen ist. Die obere Schranke $\mathcal{O}(|\obj|^2)$ erhält man durch naives Ausprobieren jeder Kombination.
\item Asymmetrische Interaktionen: Es gibt 2 Mengen $\obj_M, \obj_S$,die Menge der Master-Objekte und die Menge der Slave-Objekte. Daraus werden alle Objektpaare mit jeweils einem Mitglied aus jeder Menge ausgewählt, genau dann, wenn deren AABBs sich überschneiden. Das Master-Objekt ist dabei der aktive Part, der vom Algorithmus bei einer Interaktion informiert wird. Die Zuweisung ist also zur Compilezeit festgelegt. Wenn also ein Filteralgorithmus 2 interne Mengen unterschiedlich behandelt, aber die Zuweisung dieser 2 Mengen an Master/Slave-Menge beliebig sein soll, muss dieser diese Flexibilität selbst extra implementieren. Als untere Schranke kann man $\Omega(|\obj_M|+|\obj_S|+|C|)$ angeben. Die obere Schranke $\mathcal{O}(|\obj_M|+|\obj_S|)$ erhält man durch naives Ausprobieren jeder Kombination.
\end{enumerate}

Für die Laufzeit bei praktischen Problemen sei angemerkt, dass die Konstante oft eben so wichtig zu betrachten ist wie die asymptotische Komplexität (z.B. siehe Abb.~\ref{fig:symmetricComparison} C)\\
Im Folgenden werden 2 Algorithmen, die im Rahmen des Projekts implementiert werden, näher beschrieben.


\subsection{Box Sort}
\label{sec:boxsort}
//TODO boxsort box sort einheitlich\\
Bei Box-Sort \cite{houthuys1987box}
%GRR citation wrongly displayed, seemingly not in bibliography either??????????????????????????????
 wird eine binäre Baumstruktur aufgebaut, in der anschließend nach einem Raumbereich (einer AABB, der Query-Box
\newcommand{\qb}{\operatorname{QB}}
 $\qb$) gesucht werden kann, sodass alle damit überlappenden einsortierten AABBs gefunden werden. Aufgrund der Mehrdimensionalität und dem Einsortieren von Bereichen statt Punkten müssen Anpassungen der Baum-Datenstruktur vorgenommen werden.\\
Ein Knoten referenziert genau eine einsortierte AABB. Als Sortierschlüssel wird dort das untere Extremum der einsortierten AABB in einer bestimmten Dimension, $x_{min},y_{min}$ oder $z_{min}$  (Definition vgl. \ref{sec:aabb}) gewählt, es gilt also $\operatorname{key} \in BB_{min}, \operatorname{key} \in \mathcal{S} $. 
Ist das obere Extremum der Query-Box in dieser Dimension kleiner als der Schlüssel, kann die Baumhälfte, die höhere Werte beinhaltet, ausgeschlossen werden und muss nicht mehr durchsucht werden. \\
Die Sortierdimension ist hier für jeden Knoten individuell wählbar.\\
Um die untere Baumhälfte ausschließen zu können, müssen jedoch eine weitere Information vorliegen: Die maximale Ausdehnung der AABBs im unteren Teilbaum nach oben (in der vorher festgelegten Sortierdimension). Der Knoten kann diese jedoch nicht festlegen, da die Sortierung nur anhand der unteren Extrema erfolgt. Deshalb muss jeder Knoten, der darunter liegt, berücksichtigt werden und der Maximalwert aller ermittelt werden. Dies geschieht beim Baumaufbau und wird danach im Knoten vermerkt.\\
Wird zu beiden Teilbäumen auch der jeweils andere Extremwert gespeichert, entsteht eine eindimensionale AABB und beide Teilbäume können identisch behandelt werden. Dann kann für jeden Teilbaum individuell ein Überlappungstest bestimmen, ob sich $\qb$ mit dem Teilbaum überschneidet. So können unter bestimmten Umständen sogar beide Teilbäume ausgeschlossen werden. Ein Beispiel dafür sieht man in Abb.~\ref{fig:boxsortNode}.\\

\begin{figure}
    \centering
    \includegraphics[width=1.0\textwidth]{./res/BoxsortNode.png}
    \caption{2D-Beispiel für einen Knoten (mittig, hellgrau) im Suchbaum bei Box-Sort. x-Achse zur Sortierung. Teilbäume nur durch darin enthaltene AABB(unten, dunkelgrau) dargestellt, Query Boxes (oben, gelb).}
    \label{fig:boxsortNode}
\end{figure}

Man sieht in Abb.~\ref{fig:boxsortNode}, dass QB1 sich komplett links der Bereiche der beiden Teilbäume befindet. Der Suchlauf endet für QB1 somit direkt bei dem dargestellten aktuellen Knoten. Ebenso bei QB2, die sich mit keinem Bereich der Teilbäume überschneidet, sie befindet sich diesmal aber zwischen beiden. Solch eine Lage muss im Allgemeinen nicht zwangsläufig existieren, denn die Bereiche der beiden Teilbäume können auch überlappen. QB3 hingegen überlappt mit dem rechten Teilbaum. Für die Suche scheidet in diesem Fall also nur der linke Teilbaum aus. Diese wird im rechten Teilbaum rekursiv fortgesetzt. Man kann auch Query-Boxes konstruieren, sodass eine Überlappung mit beiden Teilbäumen auftritt. Dann muss die Suche in beiden rekursiv fortgesetzt werden.\\

Der eigentliche Sortiervorgang zum Aufbau des Baums wird mit einer Variante von Quicksort durchgeführt. Dabei wird das Pivot-Element zum Knoten und die restlichen Elemente werden entsprechend des Schlüssels in die beiden Teilbäumen aufgeteilt. Bei der Zuteilung wird nebenbei die 1D-AABB des Teilbaums ermittelt, indem die Ausdehnung bei jedem Zuteilungsschritt ggf. vergrößert wird, um das aktuelle Element zu berücksichtigen.\\
Als Heuristik für das Finden des Pivot-Elements wird der Median einer zufälligen Auswahl von bis zu 9 Elementen gewählt. Dies ist ein einfacher Kompromiss zwischen rechenaufwändigeren Verfahren, die genauer wären und der zufälligen Wahl des Pivot-Elements, die aber einen weniger idealen, tieferen Baum liefert, dessen Durchlauf deshalb länger dauern würde. Die Sortierdimension wird an der Wurzel des gesamten Binärbaums auf die x-Achse festgelegt. 
Beim Einsortieren in die beiden Teilbäume wird eine Heuristik berechnet, die im jeweiligen Teilbaum die beste Sortierdimension abschätzt. Dafür wird für jede Dimension die durchschnittliche Größe der Objekte im Teilbaum in dieser Dimension berechnet. Damit kann man die Dichte der Objekte in jeder Dimension abschätzen, wenn man den Gesamtbereich kennt, den die AABBs einnehmen. Gewählt wird die Dimension niedrigster Dichte.\\
Es müssen für den Boxsort-Algorithmus vor dem Start erst alle Objekte gesammelt werden. Im asymmetrischen Fall muss dann entschieden werden, aus welcher Menge (Master/Slave //TODO Mathe) der Objekte der Baum aufgebaut wird. Die Menge der Objekte für den Baumaufbau sei //TODO Mathe. Der Benutzer kann Einfluss auf die Wahl nehmen. Die Basis der dann automatisch getroffenen Wahl ist ein Vergleich der Anzahl von Elementen je Klasse. Der Grund dafür wird deutlich, wenn die Laufzeit im erwarteten Fall betrachtet wird: $\mathcal{O}(|\obj_T|*log(|\obj_T|))$ für den Baumaufbau (folgt aus Quicksort). Wenn eine ausreichende Verteilung im Raum vorliegt und die Größe von QB in der selben Größenordnung ist wie Elemente aus $\obj_T$, dann benötigt die Suche $\mathcal{O}(log(|\obj_T|))$ Zeit \cite[//TODO Abstract; überhaupt angeben?]{houthuys1987box}. Insgesamt wird also für das Finden aller Überlappungen $\mathcal{O}(|\obj_T|*log(|\obj_T|)+|\obj_Q|*log(|\obj_T|))$ Zeit benötigt. Anhand dieser Form steht die Vermutung nahe, dass die Wahl von $\obj_T$ immer auf die kleinere Menge fallen sollte. In der Praxis ist vor den Termen $|\obj_T|*log(|\obj_T|)$ und $|\obj_Q|*log(|\obj_T|)$ aber ein jeweils anderer Vorfaktor, der deshalb auch beachtet werden muss. Messungen ergaben einen nahezu konstanten Vorfaktor beim Baumaufbau. Für den Abfrage-Durchlauf  ist der Faktor jedoch stark abhängig von der konkreten Verteilung der eingegebenen AABBs. Messungen ergeben einen um den Faktor von ca.~3 geringeren Vorfaktor unter stark idealisierten Bedingungen, einen nahezu identischen unter moderaten Bedingungen und einen ca. 3-mal größeren unter erschwerten Bedingungen (Durchschnittliche Kollisionen pro Objekt>3). Den Benchmark-Ergebnissen in ~\ref{sec:benchmark} kann man entnehmen, dass wenn die beiden Mengen die gleiche Verteilung aufweisen die kleinere Menge meist die bessere Wahl ist, vor allem bei geringer Dichte der Objektverteilung. Um trotzdem offen für zukünftige problemspezifische Erkenntnisse zu sein, bleibt die Option, die größere oder die kleinere Menge zu wählen. Dazu kann ein Präferenz-Faktor angegeben werden, der vor der Entscheidung mit der Anzahl der Master-Elemente multipliziert wird. Mit Extremwerten kann der Benutzer einstellen, dass nur eine der beiden Mengen für den Baum verwendet werden darf. Dies wird für Fälle gebraucht, wo die o.g. Bedingungen für die Komplexität der Suche im Baum nur für eine der beiden Mengen zutreffen.\\
 
 
Im symmetrischen Fall wird jedes Objekt sowohl in den Baum einsortiert als auch als QB in dem Baum gesucht. Dabei muss das Ergebnis des Paares $(qb, qb) : qb\in\{AABB_o | o \in \obj\}$ aus den Ergebnissen nachträglich wieder entfernt werden. Außerdem muss dafür gesorgt werden, dass bei einem interagierenden Objektpaar nur ein Aufruf erfolgt, obwohl beide Richtungen als Suchergebnis gefunden werden.\\

\subsection{Spatial Hashing}
\label{sec:spatialHashing}
Die Grundidee besteht hier dabei, den Raum in viele Bereiche (sog. Chunks) aufzuteilen und Kollisionspaare nur innerhalb dieser Teilbereiche zu suchen. Sind 2 Objekte weit voneinander entfernt und somit nicht im selben Chunk, ist eine Überschneidung der AABBs nicht möglich und muss deshalb auch nicht betrachtet werden. Wegen der Einfachheit bietet es sich an, regelmäßige Interavalle, durch achsenparallele Linien bzw. Ebenen getrennt zu verwenden, meist mit identischer Größe in jeder Dimension. Ein Beispiel für 2 Dimensionen sieht man in Abb.~\ref{fig:spatialHashing}.

\begin{figure}
    \centering
    \includegraphics[width=0.5\textwidth]{./res/spatialHashingAABB.png}
    \caption{2D-Beispiel von Spatial Hashing von AABBs (stark gefärbte Rechtecke). Die leicht gefärbten Bereiche sind Chunks, die ein Objekt der entsprechenden Farbe(n) beinhalten.//TODO Quelle}
    \label{fig:spatialHashing}
\end{figure}

Man sieht in Abb.~\ref{fig:spatialHashing}, dass Objekte mit mehreren Chunks überlappen können. Ein Objekt muss sich in jedem Chunk anmelden, das sich mit seiner AABB überschneidet, zu sehen z.B. beim grünen Objekt, das sich in den 4 Chunks rechts oben anmelden muss. Gibt es keine weiteren Objekte in den Chunks, wo sich das Objekt angemeldet hat, müssen keine potentiellen Kollisionen berechnet werden.\\

Im symmetrischen Fall, wenn dies für alle Objekte der Fall ist, ist die optimale Komplexität $\mathcal{O}(|\obj|)$  bei konstanter Maximalgröße von Objekten erreicht.\\
Im asymmetrischen Fall können mehrere Objekte im selben Chunk angemeldet sein, ohne dass eine Kollision auftreten muss, solange die Objekte alle vom selben Typ sind. Ist diese Bedingung für alle Chunks wahr ist die optimale Komplexität von $\mathcal{O}(|\obj_M|+|\obj_S|)$ bei konstanter Maximalgröße von Objekten erreicht. \\
Werden mehrere Objekte, die für eine Kollision kompatibel sind, in einem Chunk angemeldet, so müssen alle potentiellen Kollisionspaare überprüft werden (hier mit naivem $\mathcal{O}(|\obj|^2)$ bzw. $\mathcal{O}(|\obj_M|*|\obj_S|)$ Algorithmus). Die Überprüfung kann entweder direkt bei der Anmeldung erfolgen (hier implementiert) oder erst nach Eintragung aller Objekte erfolgen, was die Auswahl an für die weitere Auflösung verwendbaren Algorithmen erhöhen würde (???wie z.B. Verwendung Box Sort innerhalb eines Chunks).\\
 %GRR Auswahl weitere Auflösung Algorithmen ???? was ist gemeint?
Ein potentielles Problem kann in Abb.~\ref{fig:spatialHashing} bei der Überschneidung des roten und blauen Objekts betrachtet werden. Erfolgt eine Überlappung in mehreren Chunks, besteht die Gefahr der mehrfachen Behandlung von Kollisionen. Deshalb wird während der Partnerfindung ein Hashtable mit schon behandelten Objekten mitgeführt, womit effizient Duplikate vermieden werden. Auf diese Maßnahme kann verzichtet werden, wenn sich das Objekt nur in einem Chunk befindet.\\
Aus Einfachheitsgründen wurde als Chunk genau eine kubische Einheit (Seitenlänge 1) im System $\mathcal{S}^3$ gewählt. Ein Chunk lässt sich also eindeutig durch Koordinaten aus $\mathcal{I}^3$ identifizieren. Um das Anlegen vieler leerer Chunks zu vermeiden und trotzdem schnellen Zugriff zu haben, wird ein Hashtable zur Verwaltung der Chunks verwendet. Dazu wurde eine passende Hashfunktion gesucht und eingebunden, die plattformunabhängig aber dennoch mit sehr geringer Laufzeit $HASH: \mathcal{I}^3 \mapsto \mathcal{I}_{64}$ mit $\mathcal{I}_{64}$ als 64-Bit Integer einem Hashwert zuordnet.\\
Der Idealfall an dem Spatial-Hashing asymptotisch optimal wird wurde bereits beschrieben. In der Praxis tritt dies jedoch nur selten auf. Problematisch können dort z.B. große Objekte werden, denn die Anzahl Chunks, wo eine Anmeldung erforderlich ist (und damit auch die Laufzeit) steigt dann linear mit dem Volumen. Ein weiterer Problemfall ist, wenn zu viele Objekte in einem Chunk sind. Denn selbst wenn sie sich nicht überschneiden, ist dann eine $\mathcal{O}(|\obj|^2)$ bzw. $\mathcal{O}(|\obj_M|*|\obj_S|)$ Auflösung erforderlich ($\obj, \obj_M, \obj_S$ hier Mengen von Objekten im Chunk). Ob dieser Algorithmus für den Einsatz sinnvoll ist, hängt also davon ab, ob diese Probleme bei dem bestimmten Interaktionstyp auftreten, wo er eingesetzt werden soll.\\


\subsection{Benchmark und Vergleich der Verfahren}
\label{sec:benchmark}
Um die Performanz der Algorithmen bei verschiedenen Szenarien objektiv bewerten zu können, wurde im Rahmen des Projektes eine Benchmark-Umgebung implementiert. Getestet wurden verschiedene Szenarien und Parameter. Gemeinsam haben alle Szenarien die Testparameter Problemgröße (gesamte Objektanzahl) und absolute Größe der Objekte (aus $\mathcal{S}$ d.h. der durchschnittliche Wert für die Größe in der Dimension, wo das Objekt am größten ist).\\

Das Standardszenario ist eine Gleichverteilung von Objekten im Raum. 
Ein zusätzlicher Testparameter hier ist die Dichte der Objekte, welche das Verhältnis der Durchschnittsgröße (längste Ausdehnung in einer Achse) zur Durchschnittsdistanz (der Mittelpunkte) darstellt. Die asymmetrische Variante hat zusätzlich noch das Verhältnis der Menge an Objekten beider Klassen als Testparameter. Die Größe des Erzeugungsbereichs ist bei beiden Klassen identisch. Die Objektanzahl bezieht sich bei allen asymmetrischen Szenarien auf die Gesamtzahl der Objekte (also $|\obj_M| + |\obj_S|$)\\
Ein weiteres rein asymmetrisches Szenario "Shotgun" modelliert eine Menge an Projektilen und Zielen. Die Ziele sind weiterhin gleichverteilt, die Projektile modellieren jedoch eine Schrotladung, bei der viele Projektile relativ nahe zusammen in eine sehr ähnliche Richtung fliegen. Dies bedeutet, dass die AABBs der Projektile sich weitgehend überlappen (AABBs werden über den gesamten Raum ausgedehnt, durch den sich Objekte im Laufe eines Ticks bewegen, siehe \ref{sec:bounding_volume}). Sie werden aber alle nur in einem Teilbereich erzeugt, der nicht dem gesamten Erzeugungsbereich der Zielobjekte entspricht. Ein Testparameter beschreibt die erwartete Anzahl Ziele im Erzeugungsbereich der Projektile. Das Verhältnis zwischen der Menge der Zielobjekte und der Projektile ist ein weiterer Parameter.\\
Die Benchmark-Umgebung setzt den Seed des verwendeten Zufallsgenerators abhängig von den Szenarioparametern, aber nicht abhängig vom verwendeten Algorithmus. So müssen verschiedene Algorithmen die exakt selben Probleme lösen. Um Messstörungen herauszufiltern, wird jedes Experiment einige Male wiederholt und der Median für die weitere Untersuchung verwendet. Eines dieser Ergebnisse wird im Folgenden als Sample bezeichnet. Um die Probemenge zu erhöhen wurde ein Dummy-Parameter (mit 4 möglichen Werten) eingefügt, der den Seed ändert aber sonst keine Auswirkungen hat. Dies führt dazu, dass 4 Samples zu jeder Kobination von wirksamen Parametern existieren.\\
%//TODO Absatz hier?
Für die im Folgenden gezeigten Daten wurden alle Benchmarks auf einem PC mit Intel i7-4790K CPU bei 4,5GHz mit 2400MHz Dual-Channel RAM ausgeführt.\\
Zunächst soll bei Boxsort die Wahl der Objektmenge ($\obj_M$ oder $\obj_S$) für den Baum untersucht werden (in Abschnitt~\ref{sec:boxsort} theoretisch betrachtetes Problem). Zu diesem Zweck wird das Verhältnis der Ausführungszeiten beider Möglichkeiten betrachtet. Ist es größer als 1, bedeutet dies hier, dass es mehr Ausführungszeit benötigt hat, die Master-Klasse für den Baum zu verwenden, d.h. die Slave-Klasse ist dann die bessere Wahl für den Baum.\\

\begin{figure}
    \centering
    \includegraphics[width=1.0\textwidth]{./res/boxsortChoice-reference.png}
    \caption{
%GRR Diagrammtyp Histogramm ?
    Einfluss der Wahl der Objektmenge für den Baumaufbau beim Box Sort, wenn 2 gleich große Mengen getestet wurden, die beide statistisch gleichverteilt über den selben Raum sind (Referenzfall).}
    \label{fig:boxsortCoice-reference}
\end{figure}

Als Referenz soll zunächst die Wahl bei einer Gleichverteilung betrachtet werden, zunächst mit einer identischen Menge an Objekten beider Klassen. Abb.~\ref{fig:boxsortChoice-reference} zeigt, dass in diesem Referenzfall kein signifikanter Unterschied zwischen beiden besteht und die sichtbaren Unterschiede wahrscheinlich vom Zufall stammen. Zu sehen ist, dass die stärkeren Abweichungen im Randbereich häufiger bei niedrigen Objektzahlen und damit auch niedrigen absoluten Laufzeiten auftreten.\\


\begin{figure*}
	\begin{subfigure}[t]{0.50\textwidth}
		\centering
		\includegraphics[width=1\textwidth]{./res/boxsortChoice-uniform-A.png}
		
		\label{fig:boxsortChoice-uniform-A}
	\end{subfigure}
\hfill
	\begin{subfigure}[t]{0.50\textwidth}
		\centering
		\includegraphics[width=1\textwidth]{./res/boxsortChoice-uniform-B.png}

		\label{fig:boxsortChoice-uniform-B}
	\end{subfigure}
\vskip\baselineskip
	\begin{subfigure}[t]{0.50\textwidth}
		\centering
		\includegraphics[width=1\textwidth]{./res/boxsortChoice-uniform-C.png}

		\label{fig:boxsortChoice-uniform-C}
	\end{subfigure}
\hfill
	\begin{subfigure}[t]{0.50\textwidth}
		\centering
		\includegraphics[width=1\textwidth]{./res/boxsortChoice-uniform-D.png}

		\label{fig:boxsortChoice-uniform-D}
	\end{subfigure}

	\caption{Untersuchung der Auswirkungen verschiedener Parameter auf das Verhältnis der Laufzeiten des Boxsort Algorithmus //TODO}
%GRR todo
	\label{fig:boxsortChoice-uniform}
\end{figure*}


Wenn die Objektzahl je Klasse nicht mehr identisch ist, können Vermutungen aus Abschnitt~\ref{sec:boxsort} untersucht werden.
%GRR genauer referenzieren, fusionieren oder nochmal darlegen.
Mehrere Untersuchungen dazu werden in Abb.~\ref{fig:boxsortChoice-uniform} gezeigt. Abb.~\ref{fig:boxsortChoice-uniform} A zeigt alle Samples, gefärbt nach Anteil der Master-Objekte. Zu erkennen ist, dass bei den getesteten Parametern der Baum aus der kleineren Menge gebildet werden sollte. Um genauer zu betrachten, welche Parameterkombinationen zu welchem Unterschied führt, werden zunächst in Abb.~\ref{fig:boxsortChoice-uniform} B die Samples auf diejenigen beschränkt, wo der Anteil Master-Objekte 90\% beträgt. Die farbliche Aufteilung ist dort anhand der Dichte. Erkennbar ist, dass der Wertebereich stark von der Dichte abhängig ist, wobei die geringste Dichte die größte Streuung aufweist. Um dies weiter zu untersuchen, zeigt Abb.~\ref{fig:boxsortChoice-uniform} C davon nur die geringste getestete Dichte, farblich nach Objektzahl getrennt. Zu sehen ist dort, dass die Unterschiede bei weniger Objekten gegen 0\% gehen, bei vielen Objekten gegen etwa +150\%. Die Beobachtung bei vielen Objekten passt zu den in Abschnitt~\ref{sec:boxsort} beobachteten Vorfaktoren, was auch bei vielen Objekten erfolgte. Für den Fall höherer Dichte scheinen die dort ermittelten Werte nicht mit den im Benchmark gemessenen Werten übereinzustimmen, die in Abb.~\ref{fig:boxsortChoice-uniform} D zu sehen sind. Dies weist auf einen weiteren versteckten Faktor hin. Man könnte weitere Tests für höhere Dichten durchführen, dort ist Box Sort jedoch keine sinnvolle Wahl mehr, da für praktische Objektzahlen dann selbst der naive Algorithmus noch schneller wäre. Dort wäre außerdem die absolute Laufzeit für Echtzeitanwendungen zu hoch.\\

\begin{figure*}
	\begin{subfigure}[t]{0.55\textwidth}
		\centering
		\includegraphics[width=1\textwidth]{./res/boxsortChoice-shotgun-A.png}
		
		\label{fig:boxsortChoice-shotgun-A}
	\end{subfigure}
~
	\begin{subfigure}[t]{0.55\textwidth}
		\centering
		\includegraphics[width=1\textwidth]{./res/boxsortChoice-shotgun-B.png}

		\label{fig:boxsortChoice-shotgun-B}
	\end{subfigure}
~
	\begin{subfigure}[t]{0.55\textwidth}
		\centering
		\includegraphics[width=1\textwidth]{./res/boxsortChoice-shotgun-C.png}

		\label{fig:boxsortChoice-shotgun-C}
	\end{subfigure}
~
	\begin{subfigure}[t]{0.55\textwidth}
		\centering
		\includegraphics[width=1\textwidth]{./res/boxsortChoice-shotgun-D.png}

		\label{fig:boxsortChoice-shotgun-D}
	\end{subfigure}
~
	\begin{subfigure}[t]{0.55\textwidth}
		\centering
		\includegraphics[width=1\textwidth]{./res/boxsortChoice-shotgun-E.png}

		\label{fig:boxsortChoice-shotgun-E}
	\end{subfigure}
~
	\begin{subfigure}[t]{0.55\textwidth}
		\centering
		\includegraphics[width=1\textwidth]{./res/boxsortChoice-shotgun-F.png}

		\label{fig:boxsortChoice-shotgun-F}
	\end{subfigure}
~
	\begin{subfigure}[t]{0.55\textwidth}
		\centering
		\includegraphics[width=1\textwidth]{./res/boxsortChoice-shotgun-G.png}

		\label{fig:boxsortChoice-shotgun-G}
	\end{subfigure}

	\caption{Untersuchung der Auswirkungen verschiedener Parameter auf das Verhältnis der Laufzeiten des Boxsort Algorithmus //TODO}
	\label{fig:boxsortChoice-shotgun}
\end{figure*}

Die Objekte sind jedoch nicht immer gleichverteilt. Als nächstes wird deshalb das Szenario "Shotgun" betrachtet. In Abb.~\ref{fig:boxsortChoice-uniform} A ist zu sehen, dass wie bei der Gleichverteilung meist die kleinere Menge verwendet werden sollte. Die Menge der Samples, wo dies nicht zutrifft ist jedoch größer als bei der Gleichverteilung. Abb.~\ref{fig:boxsortChoice-uniform} B zeigt nur die Samples, wo der Projektilanteil 90\% beträgt. Zu erkennen ist, dass eine höhere Objektanzahl die Abweichung verstärkt. In Abb.~\ref{fig:boxsortChoice-uniform} C sieht man\\

\begin{figure*}
	\begin{subfigure}[t]{0.55\textwidth}
		\centering
		\includegraphics[width=1\textwidth]{./res/symmetricComparison-A.png}
		
		\label{fig:symmetricComparison-A}
	\end{subfigure}
~
	\begin{subfigure}[t]{0.55\textwidth}
		\centering
		\includegraphics[width=1\textwidth]{./res/symmetricComparison-B.png}

		\label{fig:symmetricComparison-B}
	\end{subfigure}
~
	\begin{subfigure}[t]{0.55\textwidth}
		\centering
		\includegraphics[width=1\textwidth]{./res/symmetricComparison-C.png}

		\label{fig:symmetricComparison-C}
	\end{subfigure}
~
	\begin{subfigure}[t]{0.55\textwidth}
		\centering
		\includegraphics[width=1\textwidth]{./res/symmetricComparison-D.png}

		\label{fig:symmetricComparison-D}
	\end{subfigure}
~
	\begin{subfigure}[t]{0.55\textwidth}
		\centering
		\includegraphics[width=1\textwidth]{./res/symmetricComparison-E.png}

		\label{fig:symmetricComparison-E}
	\end{subfigure}
~
	\begin{subfigure}[t]{0.55\textwidth}
		\centering
		\includegraphics[width=1\textwidth]{./res/symmetricComparison-F.png}

		\label{fig:symmetricComparison-F}
	\end{subfigure}

	\caption{Alle Benchmark-Daten für das gleichverteilte Szenario bei symmetrischen Interaktionen, aufgeteilt nach Dichte. Die Tabelle F zeigt die Anzahl Interaktionen bei jeweils einer Beispiel-Aufgabenstellung mit 1024 Objekten und verschiedenen Dichten.}
	\label{fig:symmetricComparison}
\end{figure*}

In Abb.~\ref{fig:symmetricComparison} wird nun die Performance von allen Algorithmen verglichen, zunächst im Fall symmetrischer Interaktionen. Das einzige hierbei betrachtete Szenario ist eine Gleichverteilung der Objekte. Nahe liegt die Annahme, dass die Größe von Objekten in //TODO S keinen Einfluss auf den naiven Algorithmus und Box Sort haben, da diese nur mit relativen Größen und Vergleichen arbeiten. Während der Auswertung wurde jedoch festgestellt, dass dies nicht der Fall war. Vergleichsoperationen und Rechnungen in //TODO S verwendeten Branches, die von der Größe der Unterschiede zwischen den Operanden abhängen. Um diese Variable zu eliminieren wurden Techniken des Branchless Programming eingesetzt. Dies hat sich zudem als Optimierung um 0-50\% je nach Umständen herausgestellt und wurde somit beibehalten. Die Graphen von Box Sort und dem naiven Algorithmus behandeln die Samples mit verschiedenen Objektgrößen identisch. Sie sind als mehrere Linien im Diagramm eingetragen, genau wie zu jedem Parametersatz die 4 Samples, die sich nur im Seed für die Generierung der Aufgabenstellung unterscheiden. Dargestellt ist die Laufzeit in Abhängigkeit von der Objektanzahl. Die Samples, immer um Faktor 2 auseinander, werden linear interpoliert zu einem Graph verbunden. Zu sehen ist, dass Spatial Hashing empfindlich auf die Größe der Objekte relativ zur Größe der Chunks reagiert. Sind die Objekte z.B. durchschnittlich 10-mal größer als Chunks, ist die Performance besonders schlecht. Dies ist intuitiv verständlich, da sich jedes Objekt bei durchschnittlich 1000 Chunks anmelden und dort nach Interaktionspartnern suchen muss. Der naive Algorithmus zeigt wie zu erwarten kaum eine Abhängigkeit von der Dichte und hat die kleinste Konstante, was vor allem bei kleiner Menge and Objekten die beste Performance liefert. Box Sort hat eine konsistent gute Performance, sodass er nur bei extrem hoher Dichte oder geringer Menge an Objekten eine längere Laufzeit als der naive Algorithmus benötigt. Spatial Hashing hat in vielen Situationen vor allem bei größeren Objektmengen eine noch bessere Performance, ist aber weniger flexibel bei der Objektgröße. Neben dem offensichtlichen Problem bei großen Objekten ist auch eine zu kleine Objektgröße bei gegebener Dichte für Spatial Hashing nachteilig, da sich dann mehr Objekte innerhalb eines Chunks befinden und innerhalb eines Chunks der naive Algorithmus verwendet wird. Je nach Dichte ist eine andere Objektgröße ideal. Bei geringer Dichte überwiegt der Laufzeitanteil der Anmeldung in(und damit Erstellung von) Chunks, sodass dort möglichst kleine Objekte in weniger Chunks enthalten sind. Wenn die Dichte höher wird, nähert sich die ideale Objektgröße an die Chunkgröße an. Für extreme Dichten ist Spatial Hashing genau wie Box Sort ungeeignet.\\


\begin{figure*}
	\begin{subfigure}[t]{0.55\textwidth}
		\centering
		\includegraphics[width=1\textwidth]{./res/asymComparison-A.png}
		
		\label{fig:asymComparison-A}
	\end{subfigure}
~
	\begin{subfigure}[t]{0.55\textwidth}
		\centering
		\includegraphics[width=1\textwidth]{./res/asymComparison-B.png}

		\label{fig:asymComparison-B}
	\end{subfigure}
~
	\begin{subfigure}[t]{0.55\textwidth}
		\centering
		\includegraphics[width=1\textwidth]{./res/asymComparison-C.png}

		\label{fig:asymComparison-C}
	\end{subfigure}
~
	\begin{subfigure}[t]{0.55\textwidth}
		\centering
		\includegraphics[width=1\textwidth]{./res/asymComparison-D.png}

		\label{fig:asymComparison-D}
	\end{subfigure}
~
	\begin{subfigure}[t]{0.55\textwidth}
		\centering
		\includegraphics[width=1\textwidth]{./res/asymComparison-E.png}

		\label{fig:asymComparison-E}
	\end{subfigure}
~
	\begin{subfigure}[t]{0.55\textwidth}
		\centering
		\includegraphics[width=1\textwidth]{./res/asymComparison-F.png}

		\label{fig:asymComparison-F}
	\end{subfigure}

	\caption{Alle Benchmark-Daten für das gleichverteilte Szenario bei asymmetrischen Interaktionen, aufgeteilt nach dem Anteil der Objekte, die der Master-Klasse angehören. Zum Vergleich wurde die gleiche Art Graph aus allen Daten bei symmetrischen Interaktionen erzeugt.}
	\label{fig:asymComparison}
\end{figure*}

Die Szanarien mit asymmetrischen Interaktionen haben einen Parameter mehr, sodass eine kompaktere Version der Graphen benötigt wird, um effizient alle Daten darzustellen. Dazu wird eine Annahme über den Verwendungszweck gemacht, der für die meisten Games und viele Echtzeit-Simulationen zutrifft: Es steht eine bestimmte Zeit zur Verfügung, in der alle Berechnungen abgeschlossen sein müssen. Hier wurde also untersucht, für wie viele Objekte in 10ms alle Interaktionen berechnet werden können. Der Zeitrahmen ist dabei weniger strikt wie z.B. in professionellen Counter-Strike Matches aber strikter als in Games wie Minecraft. 10ms erscheint eher im strikten Bereich, aber es muss auch beachtet werden, dass die Ausführung dieses Algorithmus nicht das Einzige ist, was in einer Simulation Rechenzeit benötigt. Die maximale Objektanzahl wurde durch lineare Interpolation der getesteten Samples ermittelt. Für Fälle, wo die Anzahl außerhalb des getesteten Bereichs liegt (größer 32768 oder kleiner 128), wird stattdessen extrapoliert. Wenn ein Großteil der Samples außerhalb liegt, ist in der Legende "Extrapoliert" vermerkt.\\
Abb.~\ref{fig:asymComparison} zeigt das gleichverteilte Szenario, unterteilt nach dem Anteil der Objekte, die der Master-Klasse angehören. Zum Vergleich wurden alle Daten des symmetrischen Falls des gleichverteilten Szenarios auf die gleiche Art ausgewertet (Abb.~\ref{fig:asymComparison} F). Die Dichte ist nun auf der x-Achse. Im Vergleich zum symmetrischen Fall ist die erreichte Anzahl beim naiven Algorithmus höher, da nicht alle Objekte mit allen anderen getestet werden müssen, sondern nur Paare mit jeweils einem Objekt aus einer Klasse //TODO (O(n*m)). Dies ist besonders deutlich bei einer 10\% - 90\% Aufteilung. Box Sort hat bei asymmetrischen Interaktionen ebenfalls einen Vorteil, der noch stärker ausgeprägt ist. Dies stammt von dem Problem bei symmetrischen Interaktionen, dass aufgrund der Funktionsweise alle Interaktionen in beide Richtungen gefunden werden und die eine Richtung herausgefiltert werden muss, die Effizienz ist dort also nur 50\%. Dies ist bei asymmetrischen Interaktionen nicht der Fall. Bei Spatial Hashing ist die Laufzeit zum Anmelden in und Anlegen von Chunks bei asymmetrischen Interaktionen nicht positiv beeinflusst. Deshalb ist selbst unter ansonsten idealen Umständen (geringe Dichte) die Performance gegenüber Box Sort nicht mehr überlegen.\\
Abb.~\ref{fig:asymComparison} A und Abb.~\ref{fig:asymComparison} E bzw. Abb.~\ref{fig:asymComparison} B und Abb.~\ref{fig:asymComparison} D weisen starke Ähnlichkeiten zueinander auf. Dies ist nicht verwunderlich, da sie nur dem Tausch der Master-Objekte mit Slave-Objekten entsprechen. Die Graphen für Box Sort mit Baum aus Master-Objekten und Slave-Objekten sind entsprechend vertauscht. Eventuelle Unterschiede zwischen den jeweiligen Teilabbildungen könnten auf verschiedene Ergebnisse der Pseudozufälligen Aufgabengenerierung zurückzuführen sein. Geringfügige Unterschiede der Trends der Graphen könnten mit Caching in der CPU zusammenhängen, denn es werden zuerst alle Master-Objekte generiert, dann die Slave-Objekte, der Cache-Inhalt ist also beim Start der Algorithmen nicht ebenfalls vertauscht.\\
Diese Symmetrie ist beim Shotgun-Szenario nicht vorhanden, da dort die beiden Mengen nicht die gleiche statistische Verteilung aufweisen. Abb.~\ref{fig:shotgunComparison} zeigt die Daten dieses Szenarios. Die x-Achse ist nun der Erwartungswert der Anzahl treffbarer Objekte in dem Bereich, wo Projektil-Objekte erzeugt werden. Es wurden 4 Werte getestet, jeweils um den Faktor 8 verschieden. Zu sehen ist eine viel höhere Abweichung der Samples auch bei Werten, die im Gleichverteilungs-Szenario keine solche aufweisen. Diese stammen wahrscheinlich daher, dass die konkret generierte Aufgabenstellung eine hohe Abweichung der Anzahl der Interaktionen aufweist. Dies ist wegen dem kompakten Bereich der Fall, worin alle Projektil-Objekte generiert werden. Nur eine geringe Anzahl treffbarer Objekte ist darin enthalten, was statistisch zu einer hohen Standardabweichung relativ zum Erwartungwert führt. Die Objekte in dem Bereich interagieren dann mit einem großen Teil aller Projektil-Objekte.\\
Allgemein ist beim Shotgun-Szenario Box Sort wieder überlegen. Den Baum aus Projektil-Objekten aufzubauen hat hier einen Vorteil: die meisten treffbaren Objekte können wegen der Kompaktheit des Projektil-Erzeugungsbereichs schon beim root-Knoten ausgeschlossen werden.\\

\begin{figure*}
	\begin{subfigure}[t]{0.55\textwidth}
		\centering
		\includegraphics[width=1\textwidth]{./res/shotgunComparison-A.png}
		
		\label{fig:shotgunComparison-A}
	\end{subfigure}
~
	\begin{subfigure}[t]{0.55\textwidth}
		\centering
		\includegraphics[width=1\textwidth]{./res/shotgunComparison-B.png}

		\label{fig:shotgunComparison-B}
	\end{subfigure}
~
	\begin{subfigure}[t]{0.55\textwidth}
		\centering
		\includegraphics[width=1\textwidth]{./res/shotgunComparison-C.png}

		\label{fig:shotgunComparison-C}
	\end{subfigure}
~
	\begin{subfigure}[t]{0.55\textwidth}
		\centering
		\includegraphics[width=1\textwidth]{./res/shotgunComparison-D.png}

		\label{fig:shotgunComparison-D}
	\end{subfigure}
~
	\begin{subfigure}[t]{0.55\textwidth}
		\centering
		\includegraphics[width=1\textwidth]{./res/shotgunComparison-E.png}

		\label{fig:shotgunComparison-E}
	\end{subfigure}

	\caption{Alle Benchmark-Daten für das Shotgun-Szenario, aufgeteilt nach dem Anteil der Objekte, die der Projektil-Klasse angehören.}
	\label{fig:shotgunComparison}
\end{figure*}

//TODO Ausblick?:


%GRR Recommendations:
% * pack die Benchmarks zu ihren Themen dazu, passt besser, und du kannst einfacher spezifisch zurückrefernzieren, was da vorher theoretisch mal behauptet wurde, was jetzt stimmt, etc, etc.
% für mich als leser ist weniger lücke, besserer cache
% * mach am Schluss dediziert einen Vergleich zwischen den Verfahren, wobei du auf die jeweiligen benchmarkergebnisse refenzierst. so sind die benchmarks in ihrer schublade, und das eigentliche ergebnis deiner benchmarks, nämlich der vergleich der Algorithmen, ist dedizierter plaziert, was ja, denke ich, deiner intention ensprechen sollte
% * bitch in deinem Fazit auf jeden fall, wie viel aufwand die benchmarks waren, da sie ja dann ggf. nichtmehr ein extrakapitel sind und das dann weniger auffällt, erwähne auch gerne, wie sehr du dich da in der bearbeitungszeit verschätzt hast
% spezifizität würde ich mir mehr wünschen. an stellen ist nur sehr schwer eindeutig auszumachen was du meinst und ich muss sachen annehmen um zu verstehen, die aber falsch sein können, etc. Dazu gehören vor allem Definitionen von Begriffen. Wenn du vor hast einen Begriff mehrfach zu verwenden kann mann dafür auch einen dedizierten satz opfern. Wenn der Begriff dann definiert ist, bitte keinen Deckungsgleichen begriff einführen. Ich hab an stellen die Einschätzung, dass einige nicht definierte Begriffe, die zur variablen Wortwahl beitragen sollen, so zwischen absätzen ihre Semantik ändern.


\section{Paarweise Modellkollisionen}
Die paarweise Kollisionserkennung bezieht sich auf die Kollision von zwei 3D-Modellen. In diesem Schritt ist die Genauigkeit der Kollision bei steigender Modellkomplexität im Fokus.
Ziel ist es die Diskrepanzen zwischen der grafischen Anzeige von Modellen und der physikalischen Simulation zu deckeln.
\subsection{Input}
Aus dem vorhergehenden Schritt der Kollisionspaarermittlung werden für den Schritt der paarweisen Kollisionserkennung bis zu  $|OBJ|^2$ Objektpaare $C_{guess}\subseteq OBJ^2$ erhalten, für die eine Kollisionsvermutung während eines Ticks gilt. Man geht weiter davon aus, dass die Menge der tatsächlichen Kollisionen $C_{definitive}\subseteq OBJ^2$ in der Menge der vermuteten Kollisionspaare  vollständig enthalten ist $C_{definitive}\subseteq C_{guess}$, d.h. keine tatsächlichen Kollisionspaare im Paarermittlungsprozess verlorengehen. Es ist eine Aufgabe dieses Schritts, die Reduzierung von $C_{guess}$ auf $C_{defintive}$ zu vollenden.

\subsection{Output}
Im Abschnitt~\ref{sec:usages} werden einige Verwendungszwecke der Kollisionsermittlung aufgelistet. 
Elemente der Aufzählung haben unterschiedliche Anforderungen hinsichtlich benötigter Information über den Kollisionsvorgang um realisiert werden zu können.

\begin{enumerate}
	\item Inklusion\\
		Die Frage ob Objekte $o_0,o_1 \in OBJ$ sich an einem konkreten Zeitpunkt während eines Ticks $t\in \Upsilon_{\delta i}$ schneiden/gegenseitig enthalten. $D(o_0, t) \cap D(o_1, t) \neq \emptyset$.
	\item Intrusion\\
		Die Frage nach den Eigenschaften
		\begin{enumerate}
		\item ob
		\item wann
		\item an welcher Raumposition
		\item mit welchen Teilen des Objektes
		\end{enumerate}
		ein Eindringen eines Objektes in ein anderes stattfindet, d.h. ein Zustandsübergang von Nicht-Kollision zu Kollision stattfindet.
		Hier beziehen wir uns auf die Zeitspanne eines Ticks $\delta_i$, während dem bei Zeitpunkt $\mathcal{T}(s)$ die erste Kollision (während $\delta_i$) auftritt. 
		$$\exists s \in [1, |\Upsilon_{\delta i}|-1], t_c= \Upsilon_{\delta i}(s):$$
		$$ D(o_0, t_c)\cap D(o_1, t_c) \neq \emptyset \wedge \forall s' < s: 
		D(o_0, \Upsilon_{\delta i}(s'))\cap D(o_1, \Upsilon_{\delta i}(s')) = \emptyset$$
		Aus der Schnittmenge $D(o_0, t_c)\cap D(o_1, t_c)$ können dann die Angeforderten Eigenschaften zur Kollision (Zeit, Ort, Beteiligte Objektteile) ermittelt werden.
\end{enumerate}
Es mag weitere Forderungen an Kollisionsalgorithmen geben. Die obige Aufzählung stellt dabei die von uns in Betracht gezogenen dar.

Für physikalische Verwendungszwecke insbesondere bei der rigiden Objektkollision scheint der Bereich der Intrusion mehr relevant. In Simulationen mit rigiden Kollisionsobjekten ist typischerweise der Zustand des Überschneidens zweier Objekte (Inklusion) nicht erlaubt und muss daher aktiv durch eine Kollisionsreaktion verhindert werden. Durch die Ermittlung des ersten Zeitpunkts des Bruchs der Nicht-Überschneidungsbedingung kann die Kollisionsreaktion passend ermittelt werden.\\
\begin{figure}
	\centering
	\includegraphics[width=0.6\textwidth]{./res/l3_col.png}
	\caption{2D Kollisionsszenario mit alternativen Bewegungsbahnen eines Projektils an Kollisionskörpern}
	\label{fig:l3col}
\end{figure}
Die Umsetzung der Auflösung von Kollisionen und Mehrfachkollisionen ist explizit aus diesem Projekt ausgeschlossen. Selbst jedoch bei der theoretischen Betrachtung von Mehrfachkollisionen während eines Ticks und Optionen zu deren Auflösung, bzw.~physikalischer Reaktion von Objekten , scheinen vorwiegend Erstkollisionen interessant zu sein. Man vergleiche hierzu Abbildung~\ref{fig:l3col}.\\
An dieser Stelle wird demnach die Annahme getroffen, dass zunächst nur Erstkollisionen durch Intrusion interessanter sind sind und zuerst ermittelbar gemacht werden sollten.


%%TODO look if one can fit that under his hat somewhere, would be rad 
%Ein interessanter Fehler bezogen auf L3 in freier Wildbahn ist in \cite{skyrimwallglitch} zu sehen. Es handelt sich dabei um einen Glitch im Spiel The Elder Scrolls V:Skyrim. Dabei wird eine gleichzeitige Kollision des Spielermodells (welches sich durch eine Ingame-Fähigkeit unüblich schnell bewegt), eines Objekts und einer Wand oder Tür provoziert. Die Kollision, bzw. die wiederholten Kollisionen zwischen dem Spielermodell, dem Objekt (Kessel/Teller/Korb) und der Wand/Tür werden nicht richtig aufgelöst, was dazu führt, dass der Spieler durch die Wand laufen kann.\\


\subsection{Intrusionsermittlung durch lineare Interpolation der Objektbewegung}
Um das Problem zu vereinfachen wird die zunächst die Rotation von Objekten vernachlässigt.\\
Genauer: In diesem Kontext erfährt ein Objekt $o\in OBJ$ während eines Ticks $\delta_i$ keine zeitliche Änderung der Rotation $\forall t \in \Upsilon_{\delta i} : rot(o, t) = rot(o, t_0)$. Dies hat mehrere Gründe:
\begin{enumerate}
\item Objekte hielten zum Start des Projekts noch keine Repräsentation für Rotation.
\item Rotation mit einzubeziehen wurde zu Anfang schon als schwierig eingestuft. Dieser Verhalt sollte sich später bestätigen.
\item Es gibt genügend Kollisionsszenarien/Verwendungszwecke für denen Rotation keine Rolle spielen muss. (Logische Kollision, Punktförmige Objekte, statische Objekte wie z.B. oft Terrain, Verwendungszwecke mit hoher Fehlertoleranz)
\item Es wird ein Experiment der Vernachlässigung von Rotation durchgeführt. Es soll dabei beantwortet werden, ob auch ohne Behandlung der Rotation eine ausreichend zufriedenstellende Illusion von physikalischem Realismus geschaffen werden kann.
\end{enumerate}

Wie in Abschnitt~\ref{sec:objects_mov} bereits beschrieben sind die zeitlichen Änderungen (Geschwindigkeit $v$ und Drehgeschwindigkeit $\omega$) eines Objektes während eines Ticks konstant. Durch die Vernachlässigung der Drehgeschwindigkeit $\omega$ kann die zeitliche Änderung eines Objektes $o$ allein auf Bezug zu Geschwindigkeit $v$ und der Zeit $t$ beschrieben werden, bzw.~alle im Objekt enthaltenen Punkte $p \in D(o, t_0)$ beschreiben lineare, parallele Flugbahnen, die in ihrer Gesamtheit das durchlaufene Volumen des Objekts während eines Ticks beschreiben.
$$l_p = \{p + (t - t_0) * v(o, t_0) | t_0 \leq t \leq t_1\}$$ 
Dies ermöglicht die Ermittlung von Kollisionen durch lineare Interpolation der Zeit.\\
Die Eigenschaft der Linearität hat dabei noch den weiteren entscheidenden Vorteil, dass bei einer Relativierung der Objekte wechselseitig zueinander, eine Operation der einer Transformation in Form einer Translation gleichkommt, die Linearität weiterhin gewahrt bleibt. Zum Vergleich: Mit aktiver Rotation beschrieben Objekte relativ zueinander Kurvenflugbahnen.\\
Es kann also wechselseitig zu beiden Objekten relativ gerechnet werden.\\
Wir definierten $D(o, t)$ als die Sammlung aller Flächen, Kanten und Ecken eines Objektes. Zur Vereinfachung benennen wir Objekte $o_x := (A_x, E_x, V_x)$ mit $A_x = A(o_x, t_0); E_x = E(o_x, t_0); V_x = V(o_x, t_0)$. Diese Merkmale sollen nun durch die Zeitdimension erweitert und dann kollidiert werden.\\
Es müssen nicht alle Merkmalskombinationen $\in \{Area, Edge, Vertex\}^2$ getestet werden. Es konnten essenzielle Merkmalskombinationen ermittelt werden, welche bei einer Erstkollision auftreten können:
		\begin{itemize}
			\item [$\{Vertex, Area\}$] Eine Ecke durchschlägt eine Fläche.\\
				$\Rightarrow$ Zu überpfüfende Paare: $(V_0\times A_1)\times (V_1\times A_0)$
			\item [$\{Edge, Edge\}$] Kanten durchschneiden sich gegenseitig.
				$\Rightarrow$ Zu überprüfende Paare: $(E_0\times E_1)\times (E_1\times E_0) = (E_0\times E_1)$
		\end{itemize}
		Alle anderen Kombinationen sind entweder in diesen beiden enthalten (z.B. $\{Vertex, Edge\}$ in 1.) oder eine Ereignis dieser beiden Szenarien muss logisch vorher passieren (z.B. 1. oder 2. muss vor $\{Edge, Area\}$ bereits passiert sein).
\ \\
		Aufgrund der Relativierung muss nur eines der Merkmale muss eine zeitliche Bewegung durchführen
		\begin{itemize}
			\item [$$\{Vertex, Area\}$$] 2 Möglichkeiten:
				\begin{itemize}
					\item[Option0:] Eckpunkt $v \Rightarrow$ Linie $l_v$
					\item[Option1:] Dreiecksfläche $a \Rightarrow$ Schiefes Prisma $ \{l_p | p \in A_p(a)\}$
				\end{itemize}
				Gewählt wird Option0, da einfachere Berechnung.
			\item [$\{Edge, Edge\}$] Kante $e \Rightarrow$ Parallelogramm $\{l_p | p \in E_p(e)\}$
		\end{itemize}
\ \\
		Geometrische Eigenschaften nun beteiligter Formen:
		\begin{itemize}
			\item [Linie] $L = \{l | x\in\mathcal{F} ; l_0, l_1 \in \mathcal{F}^3 ; l = l_0 + x * l_1; 0\le x\le 1 \}$
		\item [Dreieck] $T = \{t | x,y \in\mathcal{F}; t_0, t_1, t_2 \in \mathcal{F}; t = t_0 + x*t_1 + y*t_2; 0\le (x+y) \le 1\}$
			\item [Parallelogramm] $P = \{p | x,y \in\mathcal{F}; p_0, p_1, p_2 \in \mathcal{F}; p = p_0 + x*p_1 + y*p_2; 0\le x\le 1; 0\le y\le 1\}$
		\end{itemize}
\ \\
		Beide Szenarien können über Gleichungssysteme zur Ermittlung der Koeffizienten in konstanter Zeit überprüft werden:
		\begin{itemize}
			\item [$\{Vertex, Area\}$] $l_0 + x * l_1 = t_0 + y*t_1 + z*t_2$\\
				$x$ ist zudem hier der Koeffizient der Zeit, da die Linie in der Zeitdimension liegt.
			\item [$\{Edge, Edge\}$] $l_0 + x * l_1 = p = p_0 + y*p_1 + z*p_2$\\
				Sei die ursprüngliche Kante, aus dem das Parallelogramm generiert wurde $\{p_0+y*p_1\}$, so liegt die andere Kante $\{p_0 + z*p_2\}$ in der Zeitdimension und somit ist hier z der Zeitkoeffizient.
		\end{itemize}
\ \\
		Komplexität: $|V_0|* |A1| + |V_1|*|A_0|$ + $|E_0| * |E_1|$

		\begin{itemize}
			\item Zuverlässige Ermittlung der Erstkollision durch Findung des minimalen Zeitkoeffizienten.
			\item Mathematisch exakte Ermittlung der Zeit einer Kollision durch Zeitkoeffizienten.
			\item Mathematisch exakte Ermittlung des Orts einer Kollision durch Ortskoeffizienten.
			\item Ermittlung der Beteiligen Objektmerkmale der Erstkollision.
		\end{itemize}
Wie beschrieben kann dieses Verfahren bereits in einigen Szenarien, bei denen Rotation keine Rolle spielt, als vollständige Lösung zum Einsatz gebracht werden.\\
Es bleibt das Experiment, eine Illusion physikalischer Vorgänge mit rigider Körperkollision über den Linearen-Interpolationsalgorithmus zu realisieren.\\
Zunächst scheint dabei klar: Durch die Vernachlässigung der Rotation passieren physikungetreue Fehler, besonders, je größer die Varianz des Abstands der vom Objekt enthaltenen Punkte zum Objektursprung ist. Objekte sind dann unförmig und durchstreifen bei zeitlicher Bewegung logisch ein zu einem gewissen Grad anderes Volumen, als der lineare Interpolationsalgorithmus annimmt. Diese Diskrepanz kann sowohl zu False-Positives als auch zu False-Negatives führen.\\
Des Weiteren muss die vernachlässigte Rotation nachgeholt werden. Durch einen Extremfall dieser Nachlässigkeit fällt ein weiteres Defizit des Algorithmus auf. Ist die relative Geschwindigkeit zweier Objekte $=0$, wird aus Sicht des Algorithmus' kein Volumen durchstriffen. Es werden zwischen diesen Objekten dann gar keine Kollisionen erkannt, welche durch Rotation logisch/visuell aber auftreten.\\
Man versucht diese Problem durch statische Kollisionstests am Tickende zu lösen. Dabei wird die akkumulierte Rotationsbewegung/Rotationstransformation auf einmal ausgeführt. Vor und nach der Bewegung wird statisch eine Kollisionsabfrage berechnet.\\
Diese Anpassung bringt eigene Probleme mit sich:
\begin{enumerate}
\item Durch die akkumulierte Drehung kann eine Kollision bei schnellen Drehgeschwindigkeiten durch nur die Tests am Anfang und am Ende des Ticks komplett übersehen werden.
\item Nach der Rotationtransformation können Objekte sich in einem gegenseitigen Inklusionszustand $D(o_0, t_1)\cap D(o_1, t_1) \neq \emptyset$ befinden, da an dieser Stelle keine Intrusion erkannt werden kann. Der Zustand könnte durch einen Linearen-Interpolationsalgorithmus zwar erkannt werden, jedoch sind bis dato im Projekt keine Möglichkeiten enthalten, vernünftig auf diesen Fall zu reagieren.
Eine Idee ist die Suche nach einem Zeitpunkt vor der Überschneidung, durch beispielsweise Bisection des Drehungsausschlags. Es erscheint jedoch, dass der wiederholte Aufwand durch Transformation und Ausführen des Interpolationsalgorithmus in der gegebenen Zeit ebenfalls keine Qualitativ hochwertigen Ergebnisse erzielt. Der Inklusionszustand wird zudem in der rigiden Körperkollision als fehlerhafter Zustand angesehen, den die Kollisionsroutine eigentlich verhindern soll.
\end{enumerate}

%%TODO glitchpic

Es wird also weiter nach Lösungen gesucht, die sich dem Rotationsproblem besser annehmen.

\subsubsection{Gilbert-Johnson-Keerthi-Distanzalgorithmus (GJK)}
GJK ist ein Algorithmus, der zwischen kompakten, konvexen Mengen $K_0, K_1 \subseteq \mathbb{R}^3$ die minimale euklidische Distanz $distance(K_0, K_1) = min\{|k_0 - k_1| | k_0\in K_0 ; k_1 \in K_1\}$ errechnet. Er wurde 1988 publiziert \cite{gjk} und scheint heute gerne in Videospielen verwendet zu werden. Beispiele sind dabei Blizzard Entertainments Diabolo 3 \cite{gdc-physics} und Valves Half-Life 2\cite{gjk-blog}.\\
Im GJK-Algorithmus sollen Eigenschaften der sogenannten Minkowski-Differenz $\mathcal{M}:\mathcal{P}(\mathbb{R}^3)^2\mapsto \mathcal{P}(\mathbb{R}^3); \mathcal{M}(K_0, K_1) = \{a - b| a\in K_0 ; b\in K_1\}$ ausgenutzt werden, von der die Relevanten wie folgt lauten:\\
Sei $O = (0,0,0)$ der Koordinatenursprung.
	\begin{enumerate}
		\item Die Minkowski-Differenz zweier konvexer Körper ist ebenfalls konvex.
		\item $K_0 \cap K_1 \neq \emptyset \Rightarrow \exists a_o, b_o \in K_0 \cap K_1, a_0 \in K_0, b_0 \in K_1 : a_o = b_o \Rightarrow a_o - b_o = O \Rightarrow O \in \mathcal{M}(K_0, K_1)$
		\item $K_0 \cap K_1 = \emptyset \Rightarrow \forall a_o\in K_0, \forall b_o\in K_1 : a_o \neq b_o \Rightarrow a_o - b_o \neq O \Rightarrow O \notin \mathcal{M}(K_0, K_1)$.
		\item $distance(O, \mathcal{M}(K_0, K_1)) = distance(K_0, K_1)$
		\item (2.) $\wedge$ (3.) $\wedge$ (4.) $\Rightarrow K_0 \cap K_1 \neq \emptyset \Leftrightarrow O \in \mathcal{M}(K_0, K_1)	\Leftrightarrow distance(K_0, K_1) = 0$	
		\item (4.) $\Rightarrow distance: \mathcal{P}(\mathbb{R}^3)^2 \mapsto \mathbb{R}^+_0$ (gibt nur positive Distanzen aus)
	\end{enumerate}
	Man bemerkt: Das Verfahren behandelt das Inklusionsproblem, statt des Intrusionsproblems. Durch die Fähigkeit des GJK-Algorithmus, die Distanz zwischen zwei Objekten zu ermitteln, ist dieser jedoch ein praktisches Werkzeug auch um das Intrusionsproblem ebenfalls zu lösen.
Zum Beispiel ist ein Ansatz mit GJK, durch Abstraktion über einer Nullstellensuche der Distanz zwischen zwei Objekten das Intrusionsproblem zu lösen (vgl.~Eigenschaft 5, vgl. \cite{gdc-physics}).\\
GJK soll daher in diesem Projekt implementiert werden. Die Urquelle \cite{gjk} gibt den Algorithmus im mathematischen Kontext an. Die Umwandlung der Mathematik in eine gute Maschinenimplementierung wird nicht als trivial eingeschätzt. Es fehlen die nötigen Erfahrungswerte mit dem Algorithmus. Es werden daher weitere Quellen mit praktischem Fokus zu Rate gezogen, um die Implementierung umzusetzen \cite{gjk-blog}\cite{gjk_vid}.

Zunächst muss die Eingabeinformation von kompakten, konvexen Punktmengen, die der Algorithmus fordert, äquivalent hergestellt sein, um sicherzustellen, dass der Algorithmus auch mit Polygon-Meshes durchgeführt werden kann. Für diese ist allerdings keine Konvexität gefordert und eine entsprechende kompakte Repräsentation nicht gegeben.
Zunächst soll Konvexität hergestellt werden. Definierte nicht-konvexe Objekte können in Konvexe Partitionen aufgeteilt werden. Jede Partition zählt dann als eigenes Objekt gegenüber dem Algorithmus. Das erhöht die erwartete Komplexität der gesamten Kollisionsroutine mit GJK um den Grad der Partitionierung.\\
Es wurde zu einer automatisierten Methode recherchiert, um die Partitionierung herzustellen. Insbesondere die minimale Partitionierung scheint hier von Interesse, um die Erhöhung der Komplexität zu mitigieren.\\
Zusätzlich zu eigenen Überlegungen bestätigen die Quellen \cite{ARTIGAS20111968} und \cite{grelier2019minimum} die NP-Vollständigkeit, bzw. NP-Härte zusammengehöriger Probleme der minimalen konvexen Partitionierung.\\
Eine nicht minimale Partitionierung wäre zunächst auch ausreichend und auch lange Berechnungszeiten zur Partitionierung würden für rigide Objekte nur einmaligen Aufwand zum Programmstart bedeuten und wären daher nicht kritisch.\\
Bei Versuchen, Algorithmen für dieses Problem zu erstellen, wird jedoch eine weiter klare Hürde erkenntlich: Allein durch die Polygon-Mesh, sind Innen und Außenseite des Objektes prinzipiell nicht definiert.\\
Für grafische Zwecke kann dieser Bezug oft durch das sog. \textit{Winding} eines Dreiecks angegeben werden. Dabei wird die Reihenfolge der Angabe der Ecken eines Dreiecks konventionell festgelegt, wodurch die Richtung der Flächennormale eines Dreiecks kontrolliert werden kann, die dann Außen oder Innenseite spezifiziert. Auch die direkte Angabe einer Flächennormale zu jedem Dreieck (oder sogar jedem Eckpunkt) is im graphischen Kontext für bestimmte Effekte gebräuchlich. Diese Konzepte werden grafisch im Projekt jedoch bis dato nicht verwendet und sind daher auch nicht vorhanden.\\
Eine Konvention für ein \textit{Winding} wurde allerdings im Projekt eingeführt, nicht zuletzt, da diese die Implementierung des GJK vereinfacht.\\
Für die in dieser Projektarbeit verwendeten Testzwecke erscheint der Aufwand der Automatisierung der konvexen Partitionierung letztendlich doch zu hoch und thematisch zu fremd. Die konvexe Partitionierung des Objektes $o_x$ wird daher manuell und explizit bei Modellen durch die Angabe von Mengen von Indices von zueinander konvexen Eckpunkten $P_i \subseteq V_x$ angegeben.
Zusätzlich ist die Äquivalenz zur kompakten Punktemenge gefordert. Diese ist nun implizit durch die Konvexität gegeben. Ein kompakt dargestelltes Objekt $K$ ist konvex, wenn für alle enthaltenen Punktepaare gilt: $\forall p_0, p_1 \in K : p0 + x * (p_1 - p0) \in K ; x \in [0,1]$. In unserer Repräsentation $D(K, t)$ sind nicht alle diese Punkte enthalten. Jedoch wurde etabliert, dass die Hülle $\mathcal{H}: \mathcal{F}^3 \mapsto \mathcal{F}^3, \forall o\in OBJ: \mathcal{H}(D(o, t)) \subseteq D(o, t)$. Ist $D(K, t)$ konvex, so ist die Hülle ebenfalls konvex und alle Punkte innerhalb der konvexen Hülle können durch die Konvexitätsdefinition impliziert werden. Die zusätzlich mit der Objektrepräsentation mitgelieferte Partitionierung erweitert also die verfügbare Information und schafft in dieser Hinsicht Äquivalenz zwischen Polygon-Meshes und kompakten Mengen und die Unterscheidung von Innen und Außen. Informationen zu verbundenen Ecken einer Polygon-Mesh (Kanten \& Flächen) werden dadurch ebenfalls redundant. Der Algorithmus kann demnach außschließlich unter Einbeziehung von Eckpunkten verfahren.\\
Durch die Feststellung der Äquivalenz wurde ermittelt, dass eine Variante des GJK-Algorithmus auch an Polygon-Meshes durchgeführt werden kann.\\
Wegen der Diskretion/Beschränkung auf die Information der Eckpunkte der Polygon-Mesh $D(o,t)$ wird die daraus Berechnete Menge der Minkowski-Punkte als diskrete Minkowski-Differenz $\mathcal{M}_d$ bezeichnet.

%%TODO refactor and actually provide the images
Eine Illustation einer diskreten Minkowski-Differenz ist in den Abbildungen \ref{fig:minkov_col}, \ref{fig:minkov_col_solo} und \ref{fig:minkov_noncol}zu sehen. Zu sehen sind, in Rot und Blau, die 2 Objekte, welche sich in \ref{fig:minkov_col} überschneiden und in \ref{fig:minkov_noncol} nicht. Die Gebilde, deren Ecken grün sind und deren Kanten rot und blau sind sind die jeweiligen Minkowski-Differenzen.
Die grünen Punkte sind alle Punkte, die durch die Anwendung der Minkowski-Differenz auf die Menge der Eckpunkte beider Körper ermittelbar sind.\\
Die Kanten sind rot oder blau, je nachdem von welchem Ursprungskörper eine Kante Einfluss erhält. So ist die Beteiligung beider Körper an der Differenz besser zu erkennen. Theoretisch sind auch die Kanten zwischen jedem Punktepaar mit gemischten Einflüssen der beiden Ursprungsobjekte darstellbar, welche aus Darstellungsgründen nicht in der Grafik enthalten sind. Isoliert steht in Abbildung \ref{fig:minkov_col_solo} die Minkowski-Differenz aus Abbildung \ref{fig:minkov_col}.\\

An dieser Stelle kann auch erwähnt werden, dass das erhaltene Gebilde $\mathcal{M}_d(K_0, K_1)$ typischerweise kein Hüllenobjekt ist, da Punkte und Kanten im Innenraum bekannt sind, selbst, wenn $K_0$ und $K_1$ selbst Hüllenobjekte sind.\\

Eigenschaften der Minkowski-Differenz müssen den Umstand der Diskretion respektierend ausgenutzt werden.\\

\begin{enumerate}
		\item Die diskrete Minkowski-Differenz ist ebenfalls konvex.	
		\item Ein Simplex ist eine durch eine Anzahl $n\in\mathbb{N}$ Eckpunkte $V'\subseteq\mathcal{M}_d$ aufgespannte $n$-dimensionale geometrische, inherent konvexe und dadurch kompakt definierte Form (hier: Punkt für $n=1$, Gerade $n=2$, Dreieck $n=3$, Tetraeder $n=4$).
		\item Für das näherste Simplex $s' \in \mathcal{M}$ zum Ursprung $O$ gilt außerdem $distance(O, s') = distance(K_0, K_1)$. 
		\item Wenn gilt $O \in \mathcal{M}(K_0, K_1)$ muss aber durch die Beschränkung auf diskrete Eckpunkte nicht gelten, dass $O \in \mathcal{M}_d(K_0, K_1)$. Jedoch kann eine äquivalente Beziehung gefunden werden: $O \in \mathcal{M}(K_0, K_1) \equiv \exists Simplex~s \in \mathcal{M}_d(K_0, K_1) : O \in s$, , die den Ursprung enthält/umschließt.

\end{enumerate}

Die Implementierung des GJK bezieht sich also auf die Suche nach dem nähersten, bzw. umschließenden Simplex zum Ursprung.\\
Es wird im Folgenden der Algorithmus beschrieben:
\begin{enumerate}
	\item Initialisiere mit
	\begin{enumerate}
		\item Zwei konvexen Körpern $K_0, K_1$
		\item ihrer diskreten Minkowski-Differenz $\mathcal{M}_d = \mathcal{M}_d(K_0, K_1)$
		\item beliebigem Startpunkt $P_0 \in \mathcal{M}_d$
		\item der aktuellen Menge von Simplexeckpunkten $V' = \{V'_0, V'_1, ...\} = \{P_0\}$
		\item der Suchrichtung $D = -P_0$
		\item der minimalen gefundene Distanz $d_{min} = \infty$
		\item der Fähigkeit, Abstände zwischen einem Simplex und dem Ursprung zu errechnen $distance: Simplex \times \mathcal{F}^3 \mapsto \mathcal{F}$
	\end{enumerate}	
	\item Finde neuen Punkt $V'_{|V'|}$ zur $(|V'|+1)$-dimensionalen Erweiterung des Simplex den maximalen Punkt $v \in \mathcal{M}_d$ in der Suchrichtung $D$ über das Skalarprodukt $V'_{|V'|} = v \in \mathcal{M}_d : v \circ D = max\{D \circ v_i ; v_i \in M_d\}$. Da $\mathcal{M}_d$ konvex, existiert dieses Maximum in jede Richtung.
	\item $-V'_{|V'|} \circ D > 0 \Rightarrow$ Ursprung $O$ ist definitiv außerhalb von $\mathcal{M}_d$. Es wird $d_{min} = distance(V', O)$ gesetzt.
	\item Durch die Punkte $Simplex~s_{new} = V' \cup {V'_{|V'|}}$ kann eine Menge von Simplices $\mathcal{P}(s_{new})$ beschrieben werden. Es soll das zum Ursprung nächste Simplex ausgewählt werden. Die Auswahl $\mathcal{P}(s_{new})$ kann weiter dadurch eingeschränkt werden, dass nur neue Simplices, die durch den neuen Punkt $V'_{|V'|}$ bekannt wurden überprüft werden müssen, da durch die Suche des Punkts in Suchrichtung $D$ an dieser Stelle eine Distanzverbesserung durch den neuen Punkt erwartet wird. Das Gegenereignis hierzu kann durch Konvexität von $\mathcal{M}_d$ und der Maximaleigenschaft der Punkte in $V'$ ausgeschlossen werden, solange das näheste Simplex nicht bereits erreicht ist.  Die reduzierte Menge $s_{check} \subset \mathcal{P}(s_{new}); s_{check} = \{ \mathcal{s} \in s_{new} | V'_{|V'|} \in s \}$ wird demnach stattdessen überprüft und das neue nächste Simplex 
	 $s \in s_{check} : distance(s, O) = min\{distance(s', O) | s' \in s_{check}\}$
	 ermittelt.
	\item Falls keine kleinere Distanz gefunden wurde, ist das näherste Simplex zum Ursprung gefunden. $d_{min} \leq distance(s, O) \Rightarrow distance(K_0, K_1) = distance(V', O) = d_{min}$ (\textbf{RETURN})
	\item Ist das ermittelte nächste Simplex ein Tetraeder $|V'| = 4$, ist der Ursprung im 3D-Problem komplett davon umschlossen und $\Rightarrow distance(\mathcal{M}_d, O) = 0$ (\textbf{RETURN})
	\item $d_{min} = distance(V', O)$ 
	\item Die neue Suchrichtung wird orthogonal zum Simplex $V'$ in Richtung des Ursprungs festgelegt. Die konkrete Berechnung hängt dabei von $|V'|$ ab. Ein Beispiel ist: $|V'| = 2 \Rightarrow D = (b-a)\times (O-a)\times (b-a);  \{a, b\} = V'$, mit $\times$ als Kreuzprodukt zweier Vektoren.
	\item Wiederhole ab 2.
\end{enumerate}

\begin{figure}
	\centering
	\includegraphics[width=0.6\textwidth]{./res/why_criteria.png}
	\caption{Szenario, in dem das gezeigte Abbruchskriterium nicht am nähesten Merkmal zum Ursprung stoppt.}
	\label{fig:why_criteria}
\end{figure}

Der bisher dargestellte Algorithmus löst jedoch nicht das Problem der Distanzfindung. Nachdem der Algorithmus in 5.~ abbricht, muss weiter nach dem nähesten Merkmal am Ursprung gesucht werden. Die Grafik \ref{fig:why_criteria} zeigt dabei, dass das bisherige Abbruchkriterium aus 5. nicht ausreicht.
Dort zu sehen ist das schwarze Dreieck $ABC$ und der grüne Ursprung $O$. Angenommen der Algorithmus beginnt an Punkt $B$ und sucht in Richtung des Ursprungs (grüner Pfeil) so wird als nächster Supportpunkt A gewählt und somit die rote Linie gefunden. Das Abbruchkriterium prüft nun, ob der Ursprung noch hinter diesem maximalen Punkt in der Suchrichtung läge, was durch das Skalarprodukt zwischen dem blauen und dem grünen Pfeil festgestellt wird (Schritt 5). In dem in der Grafik gezeigten Szenario führte dies zum Abbruch. Das gefundene Merkmal $AB$, die rote Linie, ist jedoch nicht das näherste am Ursprung. Das näheste Merkmal am Ursprung wäre stattdessen die Linie AC.\\
Prinzipiell ist das weitere Vorgehen des Algorithmus wie folgt zu beschreiben:\\
Suche neues Vertex in Richtung des Ursprungs bis keine weitere Annäherung mehr erfolgt.\\
Es wurden verschiedene Versuche unternommen das Kriterium einfach umzusetzen:
\begin{enumerate}
	\item Das Simplex, welches durch die bekannten Punkte aus $S$ gebildet wird besitzt eine minimale Distanz zum Ursprung, der errechnet werden kann. Verringert sich dieser Abstand nicht durch Erweiterung um den neuen Support-Punkt $P_{\#S}$ in 4. wird abgebrochen. Die Lösung bedient sich dabei anderen mathematischen Routinen, wie z.B. Normalisierung (welche Division verwendet) und wird daher zwar als trivial, aber auch als relativ komplex angesehen.
	\item Neue Support Punkte sind nach 4.~ maximal. Ist das näherste Merkmal erreicht, so wird nach 4.~in eine Richtung gesucht, deren maximale Punkte bereits bekannt sein müssen. Daher ist ein mögliches Abbruchkriterium: Abbruch beim finden eines Support-Punktes, der bereits bekannt ist. Dieses Kriterium kann in Konstantzeit abgeprüft werden, da die bekannten Supportpunkte eine maximale Anzahl von 3 Annehmen können.
	\item Ein neuer Support Punkt soll eine Erweiterung in die Suchrichtung darstellen. Seien $S$ die bekannten Support-Punkte und $P_{\#S}$ der neu gewonnene. Nehmen wir die Suchrichtung als Ebenennormale an, liegt das von den in $S$ enthaltenen Punkten gebildete Simplex in der dieser Ebene. Alle neuen Punkte, welche eine Erweiterung in Richtung des Ursprungs darstellen sollen, müssen in Richtung der Ebenennormale liegen. Dies ist festzustellen, in dem Vektoren $V := P_{\#S}-s_i ; s_i\in S$ auf ein positives Skalarprodukt mit der Suchrichtung überprüft werden. Abbruchkriterium: $\exists V : D \cdot V <= 0$. Idee: Wird ein koplanarer oder gar schon bekannte Support-Punkt gefunden ist das Skalarprodukt $0$ und der Abbruch wird eingeleitet, da keine Erweiterung mehr schaffbar ist.
\end{enumerate}
Bei der Implementierung der Abbruchkriterien wurden jedoch Probleme erkennbar. Bei Abbruchkriterien 2 und 3 konvergierte der Algorithmus regelmäßig nicht. Es wurden folgende Fehler gefunden:
\begin{enumerate}
	\item Falsche Annahme: Das näherste Merkmal der Markovsumme ist eindeutig.\\
	Generierbare Simplices in der Markovsumme können überlappen (z.B. aber nicht ausschließlich mit anteilig gleichen Eckpunkten bei Dreiecken). Es können demnach mehrere nächste Simplices zum Ursprung existieren.\\

\begin{figure}
	\centering
	\includegraphics[width=0.6\textwidth]{./res/parallel_features.png}
	\caption{Szenario, in dem das gezeigte Abbruchskriterium nicht am nähesten Merkmal zum Ursprung stoppt.}
	\label{fig:parfeat}
\end{figure}

		Die Grafik \ref{fig:parfeat} zeigt ein solches Szenario. Oben im Bild, die rote und blaue Linie, geben die Ursprungsobjekte an. Der Ursprung ist in violett zu sehen. Die unten im Bild befindliche Linie ist die Markovsumme der roten und blauen Linie. An dieser zu sehen, in Grün, sind die generierten Eckpunkte. Die koplanarität erlaubt dem Algorithmus das Ignorieren des mittleren grünen Punktes. Demnach sind hier der mittlere Punkt und die Linie, welche die beiden äußeren Punkte verbindet exakt gleich weit vom Ursprung entfernt, genauso sind die Linien von jeweils einem Punkt von außen zur Mitte, und ein Dreieck aus allen grünen Punkten.

	\item Falsche Annahme: Verwendete Funktionen sind ausreichend genau.\\
		Theoretisch sollten (nahezu) koplanare Merkmale dieselbe neue Suchrichtung erzeugen. Daraufhin neu gefundene Support-Punkte sind deterministisch (Reihenfolge der Markovpunkte in der Maximumssuche ist immer gleich, Maximumssuche ebenfalls deterministisch). Ungenauigkeiten bei der Suchrichtungsermittlung arbeiten jedoch diesem Determinismus entgegen und erzeugte bei (unter floating Point präzision) nahezu gleichen Eingaben unterschiedliche Punkte, wenn eine auswahl von nahezu koplanaren Punkten existiert. Mehr noch: Alle neuen Punkte skalierten dabei regelmäßig auf Grund der Ungenauigkeit auch noch positiv mit der Suchrichtung was zur Folge hat das Abbruchkriterien 2 und 3 zu keiner konvergenz führten und den Worst Case einer Endlosschleife annehmen.\\
	Die Ungenauigkeit wurde auf die Kondition des Kreuzproduktes zurückgeführt, welches hauptsächlich zur Ermittlung der orthogonalen Suchrichtung verwendet wird. Das ursprüglich als zu komplex angesehene Abbruchkriterium 1 arbeitet nicht mit der aus Kreuzprodukten erzeugten Suchrichtung, sonder direkt auf den bekannten Merkmalen, verwendet selbst keine Kreuzprodukte und wird daher die einzig verbleibende Option. Das Kriterium führte auch in allen Tests erfolgreich zur Konvergenz.
\end{enumerate}
~\\
Weiter ist der Algorithmus bis jetzt nur ein statischer Intersektionstest. Um aus dem Inklusionsalgorithmus einen Intrusionsalgorithmus zu machen, wird eine Nullstellensuche auf die erhaltene Distanz angewandt.
Das Wurzelsuchverfahren hat zudem den Vorteil, dass es unabhängig von der Bewegungsart der Objekte ist. Animation, Rotation oder beliebige Transformation ist prinzipiell denkbar, solange über eine stetige Distanzfunktion abstrahiert wird.\\
Insbesondere im Falle der Rotation sind Überlegungen zu Startpunkten der Suche and der Distanzfunktionskurve anzustellen. Bei einer schnellen Rotation kollidieren Objekte unter Umständen mehrfach hintereinander, mit kollisionsfreien Episoden dazwischen. Bei Betrachtungen der Intrusion können Nullstellen sowohl beim Ein- als auch beim Austritt festgestellt werden.\\
Die Distanzfunktion verhält sich bei rotierenden Körpern wie eine Schwingung. Es gilt das Nyquist-Shannon-Abtasttheorem. Um beispielsweise die erste Nullstelle/Kollision zu finden ist die Rotationsgeschwindigkeit eines Objektes von Bedeutung, um den Suchraum bedeutend einzugrenzen.\\
Desweiteren ist an dieser Stelle festzustellen, dass das Verfahren wieder durch die Rotationsgeschwindigkeit beschränkt wird.
\\
Des weiteren bleibt ein Problem mit der Restriktion auf konvexe Objekte mit diesem Ansatz. 
Nicht-konvexte Objekte können in konvexe Objekte aufgetrennt werden. Praktisch wird diese Trennung meist extra mit dem Modell angegeben. Dabei sind in der Industrie die Form des tatsächlichen Modells und den zur Kollision angebenen Formen oft nicht kongruent und stellen daher inakkurate Hitboxen dar.\\
Automatisierte Methoden der Aufteilung sind komplex. Prinzipiell, für den Fall dass Objekte rigide sind, ist eine solche Aufteilung 1-mal Aufwand und daher Echtzeitfähig. Sind Objekte jedoch flexibel, z.B. durch Animation an Gelenken, müsste eine automatisierte Aufteilung bei jeder Änderung, u.U.~ mehrere male pro Tick auf ein Objekt angewendet werden. Das wäre Echtzeitfähig nicht tragbar. Deshalb werden in vielen Videospielen Gelenke ignoriert oder durch Kugeln approximiert.\\
Eine Familie nicht-konvexer Objekte, welche aus konvexen Objekten aufgebaut ist, ist prinzipiell noch abdeckbar. Dabei ist wiederholtes Verwenden des GJK-Algorithmus für jedes Teilobjektepaar verschiedener Ursprungsobjekte vonnöten. \\
Es lohnt sich, konvexe Teilobjektpaare durch andere Verfahren vorzufiltern und auszuschließen, wenn möglich.\\
Möglicher Ansatz hierfür ist z.B.~ wieder AABB-Pruning (wir bereits aus L1 bekannt).
\\
Überprüfung der erfüllung gestellter Anforderungen aus L3:\\
\begin{itemize}
	\item Die Kollision wird ermittelt
	\item Die Kollisionszeit wird in einem bestimmten Genauigkeitsrahmen während der Wurzelsuche bekannt.
	\item Die beteiligten Merkmale können aus den ermittelten nähersten Supportpunkten zurückgerechnet werden. Dazu werden zu den Supportpunkten die Quellpunkte mitgeführt.
\end{itemize}

Für das GJK-Wurzelverfahren wird ein Fazit gezogen:
\begin{itemize}
	\item abstrahiert erfolgreich das Intrusionsproblem durch die Distanz für beliebige stetige Bewegungen oder gar Objektveränderungen
	\item fordert Aufteilung in konvexe Objekte
	\item Wurzelsuche optimierbar oder erweiterbar durch weitere Verfahren (vgl. \cite{gdc-physics})
	\item Verwendbar in vielen weiteren Kollisionsverfahren, in denen Distanz ermittelt werden muss
\end{itemize}
%TODO Complexity

Weitere Anmerkungen zum GJK-Algorithmus.
Der Kernpart des GJK-Algorithmus behandelt das Inklusionsproblem in beliebigen Dimensionen. Denkbar ist daher die Verwendung des GJK bei 4-dimensionalen Objekten, wie sie bei der linearen Interpolation behandelt wurden. Generell ist dieser Part des GJK im O-Kalkül sogar schneller als die $O(m*n)$ der linearen Interpolation. Die Informationen über beteiligte Features und vor allem, der konkrete Zeitpunkt der Kollision gehen dabei allerdings verloren. In bestimmten Szenarien\\
Es gibt noch zahlreiche andere Algorithmen, welche den GJK-Algorithmus verwenden um Situationsbedingt bessere Ergebnisse zu erzielen (vgl.~\cite{gdc-physics}). GJK wird in diesem Projekt in ein vergleichsweise naives Verfahren eingabaut.\\



\section{Projektverlaufsbericht}
\section{Umsetzungen}

\section{Zusammenfassung}

Das Kollisionsproblem wird in die Teilprobleme Vorfilterung und modellgenauer Kollision
geteilt. Zur Vorfilterung der Objektpaare werden zwei Algorithmen Algorithmen untersucht, von denen beide für die meisten der Verwendungszwecke eine deutliche Verbesserung gegenüber naiven Ansätzen darstellen. Sie ermöglichen so für Echtzeitfähigkeit eine deutlich höhere Anzahl simulierter Objekte, vor allem wenn Kollisionen nur sporadisch auftreten. Für die meisten Anwendungsfälle scheint die Auswahl der Algorithmen Praxistauglich zu sein.\\
Für die Realisierung von modellgenauen Kollisionen werden zwei Verfahren vorgestellt.
Eines davon ist jedoch nur in einem beschränkten Kontext definiert und daher mit dem anderen nur schwer Vergleichbar. Sie treten letztendlich als Werkzeuge für konkrete Anwendungsfälle der Kollisionserkennung und nicht als allgemeine Lösung des Problems hervor. Insbesondere der GJK-Algorithmus zeigt weiteres Optimierungspotential. Wir erfahren im Kontext des GJK auch, dass, wenn nicht allgemein darauf geachtet wird, numerische Grenzen von herkömmlichen Datentypen, und nattürliche Grenzen, wie das Nyquist-Shannon-Theorem, in den Verfahren schnell erreicht werden können.\\
Bei Verfahren, welche das 3D-Kollisionsproblem von rigiden Objekten im simulierten Raum behandeln, erfahren wir in daher Widerstand bei Steigerung der qualitativen Anforderungen, jedoch kaum bei den Quantitativen.\\
Es werden bestimmte nicht-triviale Schwierigkeiten des Kollisionsproblems erkennbar und so können einige Fehler in Produkten aus der Industrie relativiert werden. Für einige Kritikpunkte, insbesondere hinsichtlich Performance bei mehreren Objekten, kann jedoch auch eine Rechtfertigung der Kritik gezeigt werden.


\newpage

\addcontentsline{toc}{chapter}{Bibliography}
\begin{thebibliography}{12}
%\bibitem[HaKT1 98]{HaKT1 98} \footnote{In die 
%Bibliographie sollte s"amtliche benutzte Literatur 
%rein, auch nicht beim eigenen Vortrag angegebene, aber benutzte Papiere 
%und B\"ucher. Gleichzeitig sollte aber alles in der Literaturliste angegebene
%mindestens einmal im Artikel zitiert werden, sonst nicht auflisten.}
        %Michael Harkavy, J. D. Tygar, Hiroaki Kikuchi: {\sl Multi-round 
        %Anonymous Auction Protocols}; 1st IEEE Workshop on Dependable and 
        %Real-Time E-Commerce Systems, 1998.

%Here go sources, which both grr and hel might need
% for example

\bibitem[GrrDef\_79]{grr}
        Graf Default: {\sl How to Cite something}; 
        Communications of the ACM 22/11 (1979), S. 612-613.

\bibitem[2]{csgoprice}
		Online-Listung von Preisgeldern im E-Sport auf www.esportsearnings.com;
		\url{https://www.esportsearnings.com/games/245-counter-strike-global-offensive}; Zuletzt aufgerufen: 2020-02-12

\bibitem[3]{buyminecraft}
		Online-Artikel im Gamepedia Minecraft Wiki über den Microsoft Kauf von Mojang
		\url{https://minecraft-de.gamepedia.com/Microsoft-Mojang-Kauf}; Zuletzt aufgerufen: 2020-02-12

\bibitem[4]{skyrimwallglitch}
		Wall-Glitch in The Elder Scrolls V:Skyrim
		\url{https://youtu.be/YpJB7aC2kg0?t=85}; 
		alternativ: \url{https://youtu.be/9ywKvhznAE0?t=75};
		Zuletzt aufgerufen: 2020-02-12

\bibitem[5]{floatdistribution}
		Moler, Cleve; \sl{Floating Point Numbers}; 2014-07-07;
		\url{https://blogs.mathworks.com/cleve/2014/07/07/floating-point-numbers/};
		Zuletzt aufgerufen: 2020-02-15

\input{hel_bibliography.tex}
\end{thebibliography}
\bibliography{grr.bib}
\bibliographystyle{ieeetr}
\end{document}





