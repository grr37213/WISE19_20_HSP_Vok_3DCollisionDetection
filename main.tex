\documentclass[11pt,twoside,a4paper]{article}
\usepackage{german,a4wide,amsmath,amssymb}

% Mann will direkt Umlaute eingeben k�nnen statt \"a, \"o, \"u usw.
% Entweder:
%\usepackage[latin1]{inputenc}
% oder:
%\usepackage{umlaut}


% Trennvorschl"age (in {} einfuegen, wenn nicht automatisch getrennt wird:
% z.B. Authen-ti-ka-tions-sys-tem)
\hyphenation{}

\hyphenation{min-des-tens}


%-------------------------- Formatsachen --------------------------%

% Bild-, Tabellenunterschriften veraendern:
% Nummer fett, kleinerer Text fuer Bildunterschrift
\usepackage[bf,small]{caption}

\usepackage{mathpazo}  % -- Palatino als Zeichensatz -- einfach diese
					   % Zeile auskommentieren, falls nicht installiert
%\usepackage{mathptmx}  % -- Times als Zeichensatz

% Zum Unterscheiden von Entwurfs- und endgueltiger Fassung
%\usepackage{draftcopy}
%\draftcopySetGrey{0.90}   %   90% = sehr helles Grau
%\draftcopyName{ENTWURF}{155}   % statt ``DRAFT''
%\draftcopySetScale{1}

%--------------- Zeilen- und Absatzabstaende ----------------------%
\setlength{\parindent}{0em}
\setlength{\parskip}{\medskipamount}    % Abstand zwischen Abs"atzen

% ---------- Umgebungen f"ur Satz/ Lemma, etc. --------------------%
\newtheorem{satz}{Satz}
\newtheorem{nota}{Notation}
\newtheorem{defi}{Definition}
\newtheorem{kons}{Konstruktion}

\newenvironment{notation}{\noindent \textbf{Notation: }}{}
\newenvironment{beweis}{\noindent \textbf{Beweis: }}{}
\newenvironment{anmerkung}{\noindent \textbf{Anmerkung: }}{}
\newenvironment{anmerkungen}{\noindent \textbf{Anmerkungen: }}{}
\newenvironment{beispiel}{\noindent \textbf{Beispiel: }}{}
\newenvironment{beispiele}{\noindent \textbf{Beispiele: }}{}

% URLs und Mailadressen etc. richtig trennen:
\usepackage{url}
% Auch praktisch fuer Mailadressen: \url{blabla@laberlaber.de}

% --------------------- Eigene Befehle fuer math. Mengen ---------%
\newcommand{\N}{{\rm I\!N}}             % die natuerlichen Zahlen
\newcommand{\Z}{\mathbb{Z}}             % fuer ganze Zahlen
\newcommand{\R}{\mathbb{R}}             % die reellen Zahlen
\newcommand{\Prim}{{\rm I\!P}}          % die Primzahlen


% -- Sinnvolle Befehle, um sich selbst Notizen im Text zu machen --%

% "Ungeordnete Gedanken, die noch irgendwo reinsollen":
\newcommand{\kramsubsection}[1][Unsortierte Textfragmente]{%
\subsection*{#1}%
\addcontentsline{toc}{subsection}{#1}%
}

% Randbemerkung:
\newcommand{\bemerkung}[1]{\marginpar{\small\textsl{\textsf{#1}}}}

% "Hier muss noch [weiter-]geschrieben werden" (Baustellensymbol am Rand)
%
% [Damit dieser Befehl funktioniert, muss man natuerlich erstmal
%  das Icon "Baustelle.eps" besorgen!!  Also entweder selbermachen
%  oder downloaden:
%  http://www.net.in.tum.de/teaching/WS04/routing/Baustelle.eps.gz  ]
\newcommand{\baustelle}[1][]{
 \marginpar{%
   \centerline{\includegraphics[scale=0.3]{Baustelle.eps}}
   {\small\textsl{\textsf{\raggedright #1}}}
}}




\begin{document}

\title{HSP-Projektarbeit im Master Informatik \\
\small Kollisionsdetection in in echtzeit-gerenderten 3D Simulationen}
\author{Robert Graf, Lukas Hermann\\
%  (\texttt{fridolw@in.tum.de})\\[5mm]
%  Seminar "`Internetrouting"' , \\
  Ostbayerische Technische Hochschule Regensburg\\
  \\
  Projektbetreuung: Prof. Dr. Klaus Volbert
}
  
\date{WS\, 2019/2020 (Version vom \today)}


\maketitle


\abstract{Das Projekt umfasst die Erstellung einer 3D Simulation, so wie die Implementierung von Algorithmen zur Kollisionserkennung von 3D Objekten. In diesem Bericht werden vollzogene T"atigkeiten und H"urden dargestellt.}


\section{Einleitung}

3D Kollisionserkennung und -behandlung wird in Bereichen wie z.B. der Robotik, Fabrikation, Animation oder Echtzeitgraphik ben"otigt.
Insbesondere die Unterhaltungsindustrie im Bereich der Videospiele sieht sich sehr oft dem Problem der Kollisionserkennung und -behandlung gegen"ubergestellt, um die Simulation physikalischer Prozesse, oder die Illusion davon, in ihren Produkten zu erzeugen.
Oft jedoch scheint eine gewisse Diskrepanz zwischen der Erwartung der Konsumenten und der Umsetzung im Produkt zu bestehen. Physikalische Prozesse, insbesondere Kollision, scheint oft nicht akkurat Umgesetzt zu werden. Die Konsequenz daraus sind Bugs, Glitches und inimmersives Verhalten.
Da das Problem der 3D Kollision von der Wissenschaft schon seit einiger Zeit gut verstanden scheint, erscheint es umso merkw"urdiger, dass eine Milliardenindustrie an dieser Stelle immernoch Abstriche in der Entwicklung zu machen scheint.
Um Gr"unde hierf"ur herauszufinden wird in diesem Projekt versucht eine Echtzeit-3D-Simulationsumgebung zu erstellen, die Kollisionen von 3D-Objekten miteinander erkennt.
Es wird sich erhofft dabei die generellen, unoffensichtlichen H"urden zu erkennen, mit denen die Industrie zu k"ampfen hat.



TODO REMOVE

\begin{defi}[Beispieldefinition]
Gegeben ... wird ... definiert als ..
\end{defi}

TODO REMOVE

\section{Kapitel}

Grr f"uhrt in \cite{grr} das Konzept der Geheimniszerlegung ein 
(Beispielzitat).

Tse f"uhrt in \cite{tse} das Konzept der Geheimniszerlegung ein 
(Beispielzitat).

\section{Zusammenfassung}

Das Kollisionsproblem wird in die Teilprobleme Vorfilterung und modellgenauer Kollision
geteilt. Zur Vorfilterung der Objektpaare werden zwei Algorithmen Algorithmen untersucht, von denen beide für die meisten der Verwendungszwecke eine deutliche Verbesserung gegenüber naiven Ansätzen darstellen. Sie ermöglichen so für Echtzeitfähigkeit eine deutlich höhere Anzahl simulierter Objekte, vor allem wenn Kollisionen nur sporadisch auftreten. Für die meisten Anwendungsfälle scheint die Auswahl der Algorithmen Praxistauglich zu sein.\\
Für die Realisierung von modellgenauen Kollisionen werden zwei Verfahren vorgestellt.
Eines davon ist jedoch nur in einem beschränkten Kontext definiert und daher mit dem anderen nur schwer Vergleichbar. Sie treten letztendlich als Werkzeuge für konkrete Anwendungsfälle der Kollisionserkennung und nicht als allgemeine Lösung des Problems hervor. Insbesondere der GJK-Algorithmus zeigt weiteres Optimierungspotential. Wir erfahren im Kontext des GJK auch, dass, wenn nicht allgemein darauf geachtet wird, numerische Grenzen von herkömmlichen Datentypen, und nattürliche Grenzen, wie das Nyquist-Shannon-Theorem, in den Verfahren schnell erreicht werden können.\\
Bei Verfahren, welche das 3D-Kollisionsproblem von rigiden Objekten im simulierten Raum behandeln, erfahren wir in daher Widerstand bei Steigerung der qualitativen Anforderungen, jedoch kaum bei den Quantitativen.\\
Es werden bestimmte nicht-triviale Schwierigkeiten des Kollisionsproblems erkennbar und so können einige Fehler in Produkten aus der Industrie relativiert werden. Für einige Kritikpunkte, insbesondere hinsichtlich Performance bei mehreren Objekten, kann jedoch auch eine Rechtfertigung der Kritik gezeigt werden.


\newpage

\begin{thebibliography}{12}
%\bibitem[HaKT1 98]{HaKT1 98} \footnote{In die 
%Bibliographie sollte s"amtliche benutzte Literatur 
%rein, auch nicht beim eigenen Vortrag angegebene, aber benutzte Papiere 
%und B\"ucher. Gleichzeitig sollte aber alles in der Literaturliste angegebene
%mindestens einmal im Artikel zitiert werden, sonst nicht auflisten.}
        %Michael Harkavy, J. D. Tygar, Hiroaki Kikuchi: {\sl Multi-round 
        %Anonymous Auction Protocols}; 1st IEEE Workshop on Dependable and 
        %Real-Time E-Commerce Systems, 1998.

%Here go sources, which both grr and hel might need
% for example

\bibitem[GrrDef\_79]{grr}
        Graf Default: {\sl How to Cite something}; 
        Communications of the ACM 22/11 (1979), S. 612-613.

\bibitem[2]{csgoprice}
		Online-Listung von Preisgeldern im E-Sport auf www.esportsearnings.com;
		\url{https://www.esportsearnings.com/games/245-counter-strike-global-offensive}; Zuletzt aufgerufen: 2020-02-12

\bibitem[3]{buyminecraft}
		Online-Artikel im Gamepedia Minecraft Wiki über den Microsoft Kauf von Mojang
		\url{https://minecraft-de.gamepedia.com/Microsoft-Mojang-Kauf}; Zuletzt aufgerufen: 2020-02-12

\bibitem[4]{skyrimwallglitch}
		Wall-Glitch in The Elder Scrolls V:Skyrim
		\url{https://youtu.be/YpJB7aC2kg0?t=85}; 
		alternativ: \url{https://youtu.be/9ywKvhznAE0?t=75};
		Zuletzt aufgerufen: 2020-02-12

\bibitem[5]{floatdistribution}
		Moler, Cleve; \sl{Floating Point Numbers}; 2014-07-07;
		\url{https://blogs.mathworks.com/cleve/2014/07/07/floating-point-numbers/};
		Zuletzt aufgerufen: 2020-02-15

\input{hel_bibliography.tex}
\end{thebibliography}
\end{document}





