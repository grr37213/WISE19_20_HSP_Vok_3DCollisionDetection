\label{sec:bounding_volume}

Ein Hüllkörper, oder Bounding-Volume $B_o$ zu einem Objekt $o$ ist eine kompakte Menge.
\begin{align}
B_o \supseteq K_o
\end{align}
$B_o$ kann als Hitbox fungieren.\\
Eine Bounding-Box ist ein spezielles Bounding-Volume in Form eines Quaders.\\
\subsubsection{Axis-Aligned-Bounding-Box}
\label{sec:aabb}
\label{sec:AABB}
Eine in diesem Projekt extensiv verwendete, tiefere Spezialform der Bounding-Box ist die Axis-Aligned-Bounding-Box (AABB). Alle Kanten dieser Bounding-Box sind achsenparallel zu den Koordinatenachsen $\{(1,0,0), (0,1,0), (0,0,1)\}$ des Weltraums (vgl.~\ref{sec:space}).\\
Hier relevante Eigenschaften dieser sind:
\begin{itemize}
\item kleine Datenrepresentation:
\begin{align}
\AABB_o \in \mathcal{S}^{3^2};\\
\AABB_o = (b_{min}, b_{max}) = ((x_{min}, y_{min}, z_{min}), (x_{max}, y_{max}, z_{max}))
\end{align}
		 Es werden nur absolute Minimal- und Maximalpositionen der AABB festgehalten.
		 Diese Positionen werden im Worldspace $\mathcal{S}^{3^2}$ angegeben, da AABBs hier die primäre vereinfachende Abstraktion sein sollen, die in der Simulation für Objekte verwendet wird. Die Angabe im Worldspace macht AABBs für Berechnungen im absoluten, zum Beispiel räumliche Partitionierung für Teile-und-Herrsche Algorithmen, direkt zugänglich.
		 Die theoretische, kompakte Punktemenge 
		 \begin{align}
		 \calAABB: \mathcal{S}^{3^2} \mapsto \mathcal{P}(\mathcal{F}^3);\\
		 \calAABB ((x_{min}, y_{min}, z_{min}), (x_{max}, y_{max}, z_{max})) = \\
		 \{\tometer((x_{min} + x * (x_{max} - x_{min}), y_{min} + y * (y_{max} - y_{min}),\\
		  z_{min} + z * (z_{max} - z_{min}))| x, y, z \in [0,1] \} 
		 \end{align}
		 ist dann ein Bounding-Volume $\calAABB(\AABB_o) \supseteq K_o$
	\item Ermittlung einer minimalen AABB für ein Objekt durch Suche der Minima und Maxima der Ausdehnung eines Objektes in jeder Koordinatenachse:
	 $x_{min} = x : (x, y, z) \in V_o , \forall (x', y', z') \in V_o: x \leq x'; y \& z, min \& max $ analog.
	 Die Findung dieser Werte ist in $ \mathcal{O}(|V_o|) $.
		Bei rotierenden Objekten ist auch eine minimale AABB zu diesem Objekt einer Bewegung ausgesetzt, standardmäßig durch Neuermittlung der AABB($\mathcal{O}(|V_o|)$ pro Tick). Optimierungen für verschiedene Arten von Bewegung sind oft möglich (Positionsänderung, Skalierung, etc.), aber manchmal schwierig bis unmöglich (z.B. bei Rotation).\\
		Aus diesem Grund macht es auch Sinn, nicht-minimale AABBs zu wählen, um den wiederholten Berechnungsaufwand zu vermeiden.
	\item Schnelle Kollisionsüberprüfung zwischen AABBs durch Vergleiche der Extrema $\mathcal{O}(1)$
\end{itemize}

AABBs, bzw. Bounding-Volumes generell, werden nicht ausschließlich für statische Objekte $K_{o,t}$ zu einem Zeitpunkt erstellt und verwendet. Es macht in bestimmten Kontexten zum Beispiel auch Sinn das komplette durchlaufene Volumen eines Objektes $\{K_{o,t'} | t' \in \Upsilon_{\delta_i}\}$ während eines Ticks durch ein Bounding-Volume zu abstrahieren. Dies ermöglicht korrekte Kollisionsfilterung unabhängig von der Geschwindigkeit und wurde deshalb für dieses Projekt standardmäßig verwendet.

Im Falle des Spiels Minecraft werden AABBs als finale Hitboxen verwendet (vgl. Abbildungen \ref{fig:mwhitbox}, \ref{fig:mphitbox}), welche jedoch zumindest grafisch dem Kriterium $K_o \subseteq B_o$ zuwiderlaufen. Diese Designentscheidung selbst soll an dieser Stelle nicht eingeschätzt werden.