
Dieses Dokument dient als Spezifikation für ein 3D-Simulationsprogramm, welches für die Umsetzung diverser Videospielideen konzipiert ist.
Es sollen ausgewählte Aspekte der Simulation deklarativ beschrieben und/oder spezifiziert werden, ohne eine spezifische Umsetzung zu sehr einzuschränken.
Das Dokument kann ebenfalls detaillierte Informationen zur derzeitigen Umsetzung bestimmter Aspekte der Simulation liefern.
Weiter dient es als Leitfaden, in dem relevantes Kontextwissen gesammelt und referenziert ist.

\section{Problemdefinition}
Die 3D-Simulation umfasst hier die Echtzeitsimulation von Entitäten in einem 3D-Kontext. Darin enthalten sind ebenfalls die Aufgaben der Video- und Audioanzeige des Inhalts der Simulation und die Steuerung von bestimmten Inhalten in Echtzeit.\\
Die konkreten Ausmaße des Problems der Erstellung dieser Simulation ist von den Features der zu simulierenden Entitäten abhängig.

Um einen grundlegenden Einblick zu bieten können einige der umgesetzten, bzw. angestrebten Features für diese Simulation beispielhaft beschrieben werden.\\
Klassischerweise im Kontext der 3D-Simulation per se umfasst dies:
\begin{enumerate}
\item Rigide physikalische Objekte
\item passiv physikalische rigide Objektkollision
\end{enumerate}

Im Kontext eines Videospiels können weitere Aspekte hinzugefügt werden. Einige sind dabei abhängig vom jeweiligen zu realisierenden Videospiel:
\begin{enumerate}
\item Projektile
\item Nicht Spieler Charactere (NPC)/Gegner
\item Spieleravatare\\
Von einem Benutzer Steuerbare Entitäten, welche diverse Aktionen ausführen können.
\item Terrain
\item diverse benutzbare Gegenstände/Items und Inventare (Werkzeuge/Waffen)
\end{enumerate}
