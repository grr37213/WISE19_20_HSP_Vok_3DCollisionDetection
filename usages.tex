\label{sec:usages}
Im generellen sind die möglichen Anwendungsfälle für Kollisionserkennung in Videospielen sehr divers. Exemplarisch können einige für dieses Projekt in Betracht gezogene aufgelistet werden:

\begin{enumerate}
	\item logische Kollision als räumliche Anwesenheitsermittlung von Objekten, Auslöser für Ereignisse oder Interaktionen. Spezifischere Konzepte sind zum Beispiel: 
		\begin{enumerate}
			\item Raycasting\\
	Kollision mit einem Strahl (auch Raytracing) \cite[ch.6 , p.131]{fourcrossfour}.
			\item Areale\\
	Kollision mit einem meist ideal geformten Areal.
		\end{enumerate}
	\item Physikalische Kollision
		\begin{enumerate}
			\item Clipping\\
	Verhinderung des typischerweise verbotenen Schnittzustand zweier Objekte. Insbesondere im Kontext mit dem Boden als Plattform zur Fortbewegung.
			\item Trefferermittlung bei Projektilen
			\item Rigide Kollision\\
		Kollision von unzerstörbaren, unveränderlichen Objekten mit korrektem physikalischem Verhalten hinsichtlich Impuls.
			\item Zerstörung/Zerteilung von Objekten durch Kollision
		\end{enumerate}
\end{enumerate}
