
Dreidimensionale Kollisionserkennung wird in Bereichen wie z.B. der Robotik, Fabrikation, Animation, Simulation oder Echtzeitgraphik verwendet.
Die Unterhaltungsindustrie im Bereich der Videospiele sieht sich oft dem Problem der Kollisionserkennung und -behandlung gegenübergestellt, um Interaktionen zwischen physikalischen Objekten umzusetzen.
Oft jedoch scheint eine gewisse Diskrepanz zwischen der Erwartung der Konsumenten und der Umsetzung im Produkt zu bestehen. Physikalische Prozesse, insbesondere Kollision, scheinen oft nicht akkurat Umgesetzt zu werden. Die Konsequenz daraus sind Bugs, Glitches und Störungen der Immersion des Konsumenten.
Da das Problem der 3D Kollision von der Wissenschaft schon seit einiger Zeit bearbeitet wird (z.B. 1988, zum Erscheinungszeitpunkt des Papiers \cite{gjk}), erscheint es umso merkwürdiger, dass die Videospieleindustrie sich des Problems oft scheinbar nicht ausreichend annimmt.\\
Ein anschauliches Beispiel hierfür in freier Wildbahn ist in \cite{skyrimwallglitch} zu sehen. Es handelt sich dabei um einen Glitch im Spiel The Elder Scrolls V: Skyrim. Dabei wird eine gleichzeitige Kollision des Spielermodells (welches sich durch eine Ingame-Fähigkeit unüblich schnell bewegt), eines Objekts und einer Wand oder Tür provoziert. Die Kollision, bzw. die wiederholten Kollisionen zwischen dem Spielermodell, dem Objekt (Kessel/Teller/Korb) und der Wand/Tür werden nicht richtig aufgelöst, was dazu führt, dass der Spieler, der Spielregeln zuwider, durch die Wand laufen kann.\\
Auch die Performance lässt bei vielen Spielen zu wünschen übrig. Dies fällt besonders oft auf, wenn Physik dort eine Hauptrolle spielt (wie z.B. im Videospiel Kerbal Space Program) und die Anzahl der zu berechnenden Objekte etwas höher wird.\\
Es scheint jedoch unintuitiv, warum bei wenigen hundert Objekten, von denen nur ein kleiner Bruchteil kollidiert, schon deutliche Defizite in der Performanz zu erkennen sind (z.B. 114ms pro Berechnungsschritt in Kerbal Space Program bei 320 nicht kollidierenden Objekten mit ausreichend Abstand selbst bei naiver ($\mathcal{O}(n^2)$) Abhängigkeit von der Objektzahl).\\
In diesem Projekt wird daher versucht Methoden zu ermitteln und zu implementieren, um die Problemstellung der Kollisionserkennung zu lösen. Es wird sich erhofft dabei nicht offensichtliche Hürden zu identifizieren, die die Industrie bei den o.g. Problemstellungen zu überkommen hat. \\
Das Projekt baut dabei auf einer selbst erstellten Simulation auf, die vor diesem aus privatem Interesse erstellt wurde und einige vor allem grundlegende Features (wie z.B. eine Umgebung für OpenGL-3D-Grafik) bereitstellt, sich jedoch explizit nicht einer etablierten industriellen Grafikengine bedient, um so diese nicht offensichtlichen Hürden freizulegen. 
Eingeflossen sind dabei neben Features aus privatem Interesse auch Teile, die in anderen Projekten an der OTH Regensburg entstanden sind. Hier aufgeführte Features, die nicht im Rahmen des konkret hier beschriebenen Projekts entstanden sind, werden entsprechend explizit gekennzeichnet.