
3D Kollisionserkennung und -behandlung wird in Bereichen wie z.B. der Robotik, Fabrikation, Animation oder Echtzeitgraphik ben"otigt.
Insbesondere die Unterhaltungsindustrie im Bereich der Videospiele sieht sich sehr oft dem Problem der Kollisionserkennung und -behandlung gegen"ubergestellt, um die Simulation physikalischer Prozesse, oder die Illusion davon, in ihren Produkten zu erzeugen.
Oft jedoch scheint eine gewisse Diskrepanz zwischen der Erwartung der Konsumenten und der Umsetzung im Produkt zu bestehen. Physikalische Prozesse, insbesondere Kollision, scheint oft nicht akkurat Umgesetzt zu werden. Die Konsequenz daraus sind Bugs, Glitches und inimmersives Verhalten.
Da das Problem der 3D Kollision von der Wissenschaft schon seit einiger Zeit gut verstanden scheint, erscheint es umso merkw"urdiger, dass eine Milliardenindustrie an dieser Stelle immernoch Abstriche in der Entwicklung zu machen scheint.
Um Gr"unde hierf"ur herauszufinden wird in diesem Projekt versucht eine Echtzeit-3D-Simulationsumgebung zu erstellen, die Kollisionen von 3D-Objekten miteinander erkennt.
Es wird sich erhofft dabei die generellen, unoffensichtlichen H"urden zu erkennen, mit denen die Industrie zu k"ampfen hat.
