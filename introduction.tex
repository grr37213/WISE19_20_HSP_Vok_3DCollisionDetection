
Dreidimensionale Kollisionserkennung wird in Bereichen wie z.B. der Robotik, Fabrikation, Animation, Simulation oder Echtzeitgraphik ben"otigt.
Insbesondere die Unterhaltungsindustrie im Bereich der Videospiele sieht sich sehr oft dem Problem der Kollisionserkennung und -behandlung gegen"ubergestellt, um die Simulation physikalischer Prozesse, oder die Illusion davon, in ihren Produkten zu erzeugen.
Oft jedoch scheint eine gewisse Diskrepanz zwischen der Erwartung der Konsumenten und der Umsetzung im Produkt zu bestehen. Physikalische Prozesse, insbesondere Kollision, scheinen oft nicht akkurat Umgesetzt zu werden. Die Konsequenz daraus sind Bugs, Glitches und inimmersives Verhalten der Simulation.
Da das Problem der 3D Kollision von der Wissenschaft schon seit einiger Zeit gut verstanden scheint, erscheint es umso merkw"urdiger, dass die Videospieleindustrie an dieser Stelle Abstriche in der Entwicklung zu machen scheint.
In diesem Projekt wird daher versucht Methoden zu ermitteln und zu implementieren, um die Problemstellung der Kollisionserkennung zu lösen.
Es wird sich erhofft dabei unoffensichtliche H"urden zu identifizieren, mit denen die Industrie zu k"ampfen hat. \\
Das Projekt baut auf eine Codebasis auf, die schon vorher aus privatem Interesse erstellt wurde und viele vor allem grundlegende Features (wie z.B. die Bereitstellung einer Umgebung für OpenGL-3D-Grafik) bereitstellt. Eingeflossen sind dabei neben großem Aufwand aus privatem Interesse auch Teile, die in anderen Projekten an der OTH Regensburg entstanden sind. Features, die nicht im Rahmen dieses konkreten Projekts entstanden sind, werden entsprechend explizit gekennzeichnet.