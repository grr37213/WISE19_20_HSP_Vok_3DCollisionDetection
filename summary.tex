
Bei der Bearbeitung von Problemstellungen zur paarweisen Modellkollision werden verschiedene Methoden in ihrem eigenen beschränkten Kontext behandelt und ein Einblick in die Materie zur Kollisionsberechnung gegeben. Hinsichtlich der verschiedenen Verwendungszwecke von Kollision konnte jedoch keine allgemein akzeptable Lösung gefunden werden. Die hier beschriebenen Methoden dienen trotzdem hervorragend als Werkzeuge zur Behandlung von den konkreten Problemen, die von entsprechenden Verwendungszwecken gestellt werden. Es scheint daher für die Zukunft ein guter Entwicklungsansatz zu sein, Methoden im Konkreten zu behandeln und zu wählen.\\
Zur Vorfilterung der Objektpaare werden 2 Algorithmen behandelt. Sie sind beide für die meisten der Verwendungszwecke  eine deutliche Verbesserung gegenüber naiven Ansätzen und ermöglichen so für Echtzeitfähigkeit eine deutlich höhere Anzahl Objekte, vor allem wenn Kollisionen nur sporadisch auftreten. Für die meisten Anwendungsfälle scheint die Auswahl der Algorithmen schon jetzt Praxistauglich zu sein.\\
