
Das Kollisionsproblem wird in die Teilprobleme Vorfilterung und modellgenauer Kollision
geteilt. Zur Vorfilterung der Objektpaare werden zwei Algorithmen Algorithmen untersucht, von denen beide für die meisten der Verwendungszwecke eine deutliche Verbesserung gegenüber naiven Ansätzen darstellen. Sie ermöglichen so für Echtzeitfähigkeit eine deutlich höhere Anzahl simulierter Objekte, vor allem wenn Kollisionen nur sporadisch auftreten. Für die meisten Anwendungsfälle scheint die Auswahl der Algorithmen Praxistauglich zu sein.\\
Für die Realisierung von modellgenauen Kollisionen werden zwei Verfahren vorgestellt.
Eines davon ist jedoch nur in einem beschränkten Kontext definiert und daher mit dem anderen nur schwer Vergleichbar. Sie treten letztendlich als Werkzeuge für konkrete Anwendungsfälle der Kollisionserkennung und nicht als allgemeine Lösung des Problems hervor. Insbesondere der GJK-Algorithmus zeigt weiteres Optimierungspotential. Wir erfahren im Kontext des GJK auch, dass, wenn nicht allgemein darauf geachtet wird, numerische Grenzen von herkömmlichen Datentypen, und nattürliche Grenzen, wie das Nyquist-Shannon-Theorem, in den Verfahren schnell erreicht werden können.\\
Bei Verfahren, welche das 3D-Kollisionsproblem von rigiden Objekten im simulierten Raum behandeln, erfahren wir in daher Widerstand bei Steigerung der qualitativen Anforderungen, jedoch kaum bei den Quantitativen.\\
Es werden bestimmte nicht-triviale Schwierigkeiten des Kollisionsproblems erkennbar und so können einige Fehler in Produkten aus der Industrie relativiert werden. Für einige Kritikpunkte, insbesondere hinsichtlich Performance bei mehreren Objekten, kann jedoch auch eine Rechtfertigung der Kritik gezeigt werden.
