

Zur Vorfilterung der Objektpaare sind 2 Algorithmen untersucht worden. Sie sind beide für die meisten der Verwendungszwecke  eine deutliche Verbesserung gegenüber naiven Ansätzen und ermöglichen so für Echtzeitfähigkeit eine deutlich höhere Anzahl Objekte, vor allem wenn Kollisionen nur sporadisch auftreten. Für die meisten Anwendungsfälle scheint die Auswahl der Algorithmen schon jetzt Praxistauglich zu sein.\\
Für modellgenaue Kollisionen haben wir 2 Verfahren behandelt, welche jedoch jedes in seinem eigenen Kontext erst Validität erhält. Die Verfahren sind dadurch nicht Vergleichbar und treten letztendlich als Werkzeuge für konkrete Anwendungsfälle und nicht als allgemeine Lösung hervor. Insbesondere der GJK-Algorithmus zeigt weiteres Optimierungspotential. Wir erfuhren im Kontext des GJK auch, dass, wenn nicht allgemein darauf geachtet wird, numerische Grenzen von herkömmlichen Datentypen, die zur Repräsentation von Zeit, Raum, Modell, etc.~verwendet werden, in Algorithmen schnell erreicht werden können.
Es wurden bestimmte nicht-triviale Schwierigkeiten des Kollisionsproblems erkennbar und so können einige Fehler in Produkten aus der Industrie relativiert werden. Für einige Kritikpunkte konnte jedoch auch eine Rechtfertigung der Kritik gezeigt werden.
