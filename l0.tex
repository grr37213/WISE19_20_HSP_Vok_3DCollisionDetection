Das L0-Level (siehe \ref{l0}) beschäftigt sich mit der Datenrepräsentation von Modellen, Zeit und Raum.\\

Die Repr"asentation stellt sich dabei jedoch Anforderungen verschiedener technologischer Perspektiven:
		\begin{itemize}
			\item graphische Darstellung/Rendering
				F"ur die graphische Darstellung m"ussen bestimmte Formate eingehalten werden. Datenkonfomit"at mit graphischen Bibliotheken erspart umst"andliche Umwandlungen von Repr"asentationen.
			\item physikalische Berechnung
				F"ur die Kollisionsberechnung k"onnen bestimmte Repr"asentationen vorteilhaft sein. Die Repr"asentation versucht m"oglichst statisch Information "uber Objekte bereitzustellen und vermeidet dynamische Nachberechnung von Objekteigenschaften.
			\item menschliche Manipulation
				F"ur Entwicklungszwecke ist es zum Vorteil, wenn Objektrepr"asentationen menschlich lesbar, verst"andlich und nachvollziehbar sind. Dieses Konzept vereinfacht die Erstellung von Tests und die Behebung von Fehlern. Unter diesen Aspekt fällt ebenso die Konformität mit Tooling.
		\end{itemize}

Im Folgenden werden die zu repräsentierenden, angesprochenen Konzepte näher erläutert.

\subsection{Zeit}
Es existieren die Realzeit der echten Welt und die Simulationszeit. Die Zeiten können prinzipiell asynchron ablaufen.\\
In einer Echzeitsimulation müssen beide zeiten jedoch synchronisiert werden.\\
Es genügt dabei, diese Synchronisation in kurzen Zeitabschnitten herzustellen.\\
Diese zeitlichen Abstände werden oft Ticks genannt (siehe \ref{sec:tick}).\\
Gefordert sind hierbei Tickraten von ca. $60 Ticks/s \Rightarrow 16.6ms /Tick$.\\
Realzeit wird in Microsekunden-Genauigkeit vom ausführenden Betriebssystem zu Beginn jedes Ticks erhalten. Das Intervall der vergangenen Simulationszeit kann so durch den Abgleich mit der erhaltenen Zeit des vorherigen Ticks errechnet werden. Dieser Anzahl Microsekunden wird dann in einen Floating-Point-Wert in Sekunden umgewandelt, welche als Zeitfaktor in physikalischen Berechnungen verwendet werden kann.

\subsection{Raum}
Der geforderte Raum ist 3-dimensional. Die Darstellung der Positionen im Raum erfolgt über 3-dimensionale Vektoren in der Einheit von Metern.\\
Es wird daher ein Datentyp für Dezimalbrüche verwendet um kleinere Raumanteile zu erfassen.
Floating-Point-Dezimalbrüche würden sich anbieten, jedoch tritt für große Räume ein Genauigkeitsproblem auf Grund der Werteverteilung in Floating-Point Datentypen auf.\\
Sei $\mathbb{F} \subset \mathbb{R}$ mit $\mathbb{F}$ als Floating-Point-Datentyp, so ist die Verteilung der verfügbaren Werte des Floats dichter je näher am Ursprung($0.0$) \cite{floatdistribution}.\\
Für eine Darstellung im Raum $\mathbb{F}^3$ existiert dabei das selbe Problem in 3 Dimensionen. Physikalische Prozesse können daher inkonsistent in Abhängigkeit zum Ort im Raum sein.
Es wird allerdings an dieser stelle Konsistenz der simulierten Prozesse unabhängig vom Ort im Raum gefordert.\\
Lösungen des Problems sind
\begin{enumerate}
	\item Relativierung interagierender Objekte zueinander.\\
		Funktioniert unter der Annahme, dass Entfernungen zwischen interagierenden Objekten gegenüber der Gesamtgröße des Raums relativ klein sind.
	\item Aufteilung des Raums und Positionswerte von Objekten relativiert zu einem nahen Raumanteilsursprung\\
		Sorgt dafür das Objektpositionswerte nicht dem Ungenauigkeitsproblem verfallen, wenn Objekte sich weit vom Raumursprung befinden. Dafür muss ein Positionsdatentyp $\mathbb{S}$ definiert werden, welcher den betreffenden Raumanteil pro Positionswert mitführt $\mathbb{S}:\mathbb{Z}\times\mathbb{F}$. Die Relativierung zweier Positionen kann dann über eine Funktion $ \mathbb{S}\times \mathbb{S} \mapsto \mathbb{F}$ realisiert werden, wodurch die Distanz wieder auf eine gängige Einheit euklidischen Maßes zurückgeführt werden kann (hier in Metern).
\end{enumerate}

Positionen und Richtungen im Raum werden demnach mit Vektoren $s\in\mathbb{S}^3$ dargestellt. Für Berechnungsvorgänge werden Positionen zunächst relativiert, dann in $\mathbb{F}^3$ in der Einheit Meter umgewandelt.

\subsection{Objektform}
\label{sec:l0_objects}
Theoretisch können Objekte als eine Menge von Punkten dargstellt werden. Die Menge der Punkte ist dabei begrenzt durch die Form des Objektes und die Art der Darstellung ($\mathbb{S}$, bzw. im Relativen $\mathbb{F}$).
Bei einer zeitlichen Bewegung nimmt das Objekt entlang einer Zeitdimension ebenfalls Positionen ein. Theoretisch ist das Kollisionsproblem also das Problem der Ermittlung des Schnittes zweier Objekte $O_0 \cap O_1$ in 4-dimensionaler Darstellung. Diese Objektdarstellung ist im gegebenen Kontext in der Mathematik üblich, in der Computergraphik und Simulation jedoch auf Grund potentiell unendlich großer Punktemengen suboptimal.\\
In der Computergraphik hat sich daher eine andere Darstellung etabliert:\\
Objekte besitzen eine geometrische Form, welche relativ zu einer zentralen Objektposition angegeben wird. Wir nennen diese zentrale Position an dieser Stelle das \glqq Center of Mass(COM)\grqq . 
Wie in der Computergrafik üblich werden Objekte in Form von Vertices (Ecken) angegeben, welche zu Dreiecken verbunden werden. Dreiecke, Kanten und Ecken werden als Objektmerkmale bezeichnet.\\
Für das Intrusionsproblem im speziellen genügt trivial die Betrachtung der Hülle des Objektes. Es ist üblicherweise auch nur die Hülle des Objektes, die für die Grafik von Interesse ist. Die Hülle stellt also die hier minimal benötigte Objektrepräsentation dar.\\

Im Verlauf dieses Projekts werden Objekte außerdem als starr/rigide, unveränderlich und unzerstörbar angenommen.
	
\subsection{Bewegung}
Entitäten in der 3D-Simulation nehmen eine Position im Raum ein.\\
Physikalische 3D-Objekte sind Entitäten und haben zusätzlich eine Ausrichtung im Raum (Rotation).\\
Beide dieser Größen (Position und Rotation) können einer zeitlichen Änderung unterliegen (Geschwindigkeit, Drehgeschwindigkeit).\\
An dieser Stelle wird festgelegt: Während eines Ticks ändern sich diese konstanten zeitlichen Änderungsgrößen nicht. Vertexpositionen zu bestimmten Zeiten können durch Matrixtransformationen errechnet werden.
Andere Arten der Bewegung (wie z.B. durch Animation oder andere Transformationen wie Skalierung) sind in der Zukunft des Projektes denkbar, allerdings nicht Teil dieses Projektes.


