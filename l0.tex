Implementierungstechnisch schränkt eine Rechenmaschine die mathematischen Zahlenräume ein:
\begin{itemize}
\item Die Reelen Zahlen $\mathbb{R}$ beschränken sich auf Floating-Point-Datentypen, welche hier im weiteren mit $\mathcal{F} \subset \mathbb{R}$ bezeichnet werden
\item Integern $\mathbb{Z}$ sind maschinell in ihrer Darstellungsgröße beschränkt. Diese beschränkte Menge an Integern wird $\mathcal{I} \subset \mathbb{Z}$ genannt
\end{itemize}


\subsection{Zeit}
\label{sec:time}
\def\finite#1{\ooalign{\hfil$\mapstochar\mkern 3mu\mapstochar\mkern 5mu$\hfil\cr$#1$}}

Die Simulation läuft gezeitet ab. Es existieren dabei zwei relevante zeitliche Sequenzen:
\begin{enumerate}
\item Die durch maschinelle Abtastung diskrete Realzeit der echten Welt\\
	\begin{itemize}
	\item $T_r:=\langle t_{epoch}, ... , t_{max}\rangle ; t_{epoch}, t_{max}$ als minimal, bzw.~maximal darstellbare Zeit
	\item die in einer maschinellen Genauigkeit in Mikrosekunden $$\epsilon_t:=10^{-6}s ; \forall c \in \mathbb{Z}: T_r(c) + \epsilon_t = T_r(c+1)$$ gegeben ist
	\item und immer monoton wächst $\forall c \in \mathbb{Z}: T_r(c) < T_r(c+1)$. 
	\end{itemize}

\item Die Simulationszeit $T_s:=\langle t_{start}, ... , t_{end}\rangle; t_{start}, t_{end}$ als Start- und Endzeitpunkt der Simulation.
	\begin{itemize}
	\item Zwischen den beiden Zeitbasen besteht eine totale, nicht-injektive, surjektive Abbildung $\mathcal{T}:T_r \twoheadrightarrow T_s$
	\item Die Simulationszeit ist dadurch relativ zur Realzeit definiert $T_s:=\langle\mathcal{T}\rangle$
	\item Um die Kontinuität der Zeit herzustellen wird weiter eine Zeittate $r_t:T_r\mapsto\mathcal{F}$ definiert, welche das relative verstreichen der Zeit in der Simulation steuert. 
	$$\forall t_{r0},  t_{r1} \in T_r ; t_{diff}=t_{r1}-t_{r0} :\mathcal{T}(t_{r1}) = \mathcal{T}(t_{r0}) + t_{diff}*r_t(t_{r1})$$ unter der Vorraussetzung, dass die Rate während aller Zeiten zwischen $t_{r0}$ und $t_{r1}$ gleich bleibt $\forall t \in [ t_{r0},t_{r1}]: r_t( t_{r0}) = r_t(t)$. Wird die Eigenschaft des Zeitflusses in der festgelegten Rate verletzt, werden die aktuellen Echtzeitanforderungen verletzt. \\
	Soll die Rate also geändert werden muss dies zu festgelegten Umschaltpunkten geschehen, welche die Simulationszeit in Zeitbereiche trennen, zwischen denen keine Berechnungsvorgänge zuverlässig durchführbar sind.\\
Beispiele für Raten sind 
\begin{itemize}
\item $r_t(x) = 1 \Leftrightarrow$ Simulation synchron zur Echtzeit
\item $r_t(x) = 0 \Leftrightarrow$ Simulation ist pausiert/läuft nicht
\end{itemize}
Technisch wird für große $|r_t|$ die Simulation schwierig, da viele Vorgänge schnell simuliert werden müssten. Diese werden daher vermieden.\\
Theoretisch kann die Rate auch negative Werte annehmen. Die Simulationszeit würde dann rückwärts laufen. Dieses Verhalten ist technisch durch die monoton steigende Realzeit nicht leicht in konsistenter Weise umzusetzen, da Realzeiteinflüsse durch Tasteneingaben existieren und soll daher hier ebenfalls vermieden werden.
	\end{itemize}
\end{enumerate}	


\subsection{Tick \& Frame}

Die Simulation behandelt das Verstreichen von Zeit in Zeitschritten, während dem der interne Zustand der Simulation, bzw.~der simulierten Objekte, zu einem zeitlich neuen Zustand aktualisiert wird.
Dieser Zeitschritt, bzw.~Verarbeitungsschritt, wird oft als Tick bezeichnet.\\

Es ist besonders anzumerken, das der Begriff des Ticks sich ausschließlich auf das Voranschreiten der Simulation bezieht und nicht dem Anzeigen einer Szene. Die Äquivalente Bezeichnung im Kontext der Grafik wird als Frame bezeichnet, in welchem eine Szene (ein Grafischer Zustand der Simulation) gerendert wird. Es besteht Verwechslungsgefahr. Von beiden Größen können Raten $r_{tick}, r_{frame}$ angegeben werden (üblicherweise in Tick/Frame pro Realzeitsekunde). Praktisch kann eine Grafikengine durch Inter- oder Extrapolation dynamisch verschiedene, von der Tickrate unabhängige Frameraten erreichen.\\
Wir definieren die Menge der Ticks $\delta:T_r^2; \delta:=\{\delta_1, \delta_2, ...\}$ anhand ihrer Start- und Endzeitpunkte in Echtzeit $\delta_i := (\delta_{i0}, \delta_{i1})$ und erweitert die zu einem Tick gehörenden Zeitpunkte als $\delta_{id}; d \in [0,1]$. Die Inklusivität/Exklusivität muss in bestimmten Berechnungskontexten manchmal angepasst werden um bei sukzessiven Ticks doppelte Behandlungen von Ereignissen zu vermeiden. Diese Einschätzung sei für jeden Kontext individuell zu vollziehen.\\
Es gilt außerdem die Kontinuität der Zeit auch bei Ticks $\delta_{j1} = \delta_{(j+1)0}$, d.h. ein Tick beginnt am Endzeitpunkt des Vorherigen.\\
Durch die Abbildung $\mathcal{T}$ erhält der Tick eine Entsprechung in Simulationszeit.\\
Ist im aktuellen Kontext nur ein Tick $\delta_i$ von Belang, wird auch die Terminologie $t_d =\mathcal{T}(\delta_{id})$, also $t_0$ für den Tickbeginn und $t_1$ für dessen Ende in Simulationszeit, verwendet.\\
Man kann weiter die Sequenz $\Upsilon_{\delta i} = \langle t_0, ...,  t_1\rangle$ als die zusammenhängende Partition der Simulationszeitsequenz $T_s$ denotieren, welche die geordneten Zeitpunkte eines Ticks in Simulationszeit enthält.\\
Die in Abschnitt~\ref{sec:time} beschriebenen möglichen Umschaltzeiten zur Änderung der Zeitflussrate in der Simulation werden auf die Grenzen von Ticks gelegt.\\
Die Größe der Zeitdifferenz $t_1 - t_0$ unterliegt meist Einschränkungen. Bestimmte Simulationsalgorithmen wie z.B. die Methode der kleinen Schritte erfordern für eine bestimmte Genauigkeit eine maximale Schrittgröße. Die verfügbare Rechenleistung hingegen beschränkt die Tickrate nach oben. Reicht die Berechnungszeit während eine Ticks nicht um den Status der Simulation von $t_0$ auf $t_1$ zu aktualisieren, läuft die Simulation langsamer als die reale Zeit. Die Echtzeitanforderung ist dann verletzt. Oft wird die Tickrate als Konstante festgelegt, in diesem Projekt ist jedoch nur eine Mindestrate festgelegt.\\
Das Konzept eines Ticks und dessen Mindestrate war schon vor dem Projekt in der Codebasis enthalten, die Implementierung erfuhr jedoch einige Anpassungen.


\subsection{Raum}
\label{sec:space}
Der geforderte 3D Raum kann durch 3-dimensionale Vektoren $\in \mathcal{F}^3$ in der Einheit Meter beschrieben werden.\\
Durch die Werteverteilung in $\mathcal{F}$ treten jedoch bei großen Räumen für Positionen mit großer Entfernung zum Ursprung $O$ Genauigkeitsdefizite auf, die zur Verletzung von Genauigkeitsanforderung führen können. Mögliche Floating-Point-Werte liegen dabei dichter beieinander, je näher am Ursprung \cite{floatdistribution}. Ein Beispiel für die Auswirkungen dieses Sachverhalts in Simulationen kann in der Quelle \cite{floatdistributionexample} betrachtet werden.\\
Physikalische Prozesse berechnet auf Basis von Positionen in $\mathcal{F}^3$ können daher inkonsistent in Abhängigkeit zum Ort im Raum sein.\\
Das Problem wird hier durch einen neuen Längendatentypen $\mathcal{S} : \mathcal{I} \times \mathcal{F}$ gelöst, welcher den Raum zunächst gleichmäßig durch $\mathcal{I}$ aufteilt und indiziert und $\mathcal{F}$ als Offset innerhalb seines Raumteils verwendet. Es wird daher eine Größe der initialen Aufteilung $size_{grid}$ definiert.\\
Die Umrechnung zu Metern ist dann: $$ meter: \mathcal{S} \mapsto \mathcal{F};  meter((i, f)) = i * size_{grid} + f * size_{grid}$$ 
Typischerweise gilt $f \in [0;1[$, um eine eindeutige Repräsentation für einen beschriebenen Punkt zu erhalten.

Diese Darstellung hat folgende weitere Vorteile
\begin{itemize}
\item Einfache Implementierung
\item Schnelle Indizierung der durch $\mathcal{I}$ indizierten Raumanteile für raumaufteilende Teile-und-Herrsche-Algorithmen
\end{itemize}

Absolute Positionen im Raum werden demnach mit Vektoren $s\in\mathcal{S}^3$ dargestellt. Für Berechnungen von Interaktionen zwischen Objekten werden Positionen zunächst relativiert, d.h. Positionen $p \in \mathcal{S}$ sollen zu $p_0$ relativ gesetzt werden, dann sind die relativen Positionen $p' = p - p_0$. Diese werden dann in in die für lokale Interaktionen sinnhafte  Einheit Meter $\mathcal{F}^3$ umgewandelt, um darauf Berechnungen durchzuführen. Man geht dabei davon aus, das relative Strecken zwischen Objekten kurz genug sind, sodass die Genauigkeitsänderung in $\mathcal{F}$ vernachlässigbar ist.\\
Effektiv ist dabei die Eigenschaft $\mathcal{F}\subset\mathcal{S}$ nicht gefordert, auch wenn sie in der in diesem Projekt verwendeten Implementierung prinzipiell gilt.

Implementierungstechnisch bestehen verschiedene Räume je nach Anwendungsfall, in denen Objekte durch Relativierung, Längenumrechnung und Transformation dargestellt werden.

\begin{enumerate}
\item Worldspace $= \mathcal{S}^3$ absolute Positionen von Objekten
\item Cameraspace $= \mathcal{F}^3$, Ursprung $O$ ist die Position der Kamera zum Rendern einer Szene, Objekte werden zur Kamera relativiert.
\item Viewspace $= \mathcal{F}^3$, Verzerrung durch die Kameralinse, um einen Blickwinkel auf einen Bildschirm anzupassen.
\item Objektspace,$= \mathcal{F}^3$, Ursprung ist der vom Modell definierten Mittelpunkt eines Objektes (Massenmittelpunkt), zur Verarbeitung von physikalischen Objektinteraktionen wird ein Objekt zu einem anderen Objekt relativiert.
\end{enumerate}

Auf diese Weise gelingt es selbst extreme absolute Entfernungen und Geschwindigkeiten im relativen akkurat zu behandeln.

\subsection{Objektform}
\label{object_form}
Die Form eines Objektes ist in einem hier rigiden, d.h. unveränderlichen Modell beschrieben, welche die Form relativ zum Ursprung $O$ ihres eigenen Objektraums angibt.
Für Modelle werden mathematisch oft als kompakte Punktemengen verwendet. Wir denotieren diese kompakten Modelle zugehörig zum Objekt $o$ als $ K_o \subseteq \mathcal{F}$.\\

Durch die kompakte Mengendarstellung führen Rechenoperationen mit Objekten auf maschinell relativ rechenaufwändige Mengenoperationen zurück. In der Computergraphik werden deshalb Objekte durch sogenannte Polygon-Meshes dargestellt. 
Für das Objekt $o$ ist eine Polygon-Mesh $M_o := (V_o, I_o); V \subseteq \mathcal{F}^3, I \subseteq [0, |V|-1]_\mathbb{N}^3 )$ beschreibt ein Polytop durch seine Eckpunkte $V_o$(Ecken, eng. \textit{vertices}), die durch 3-Gruppierungen ihrer Indices $I_o$ zu Dreiecksflächen verbunden sind.\\
Aus der Definition der Polygon-Mesh gehen implizit weitere Definitionen hervor:

\begin{enumerate}
\item Kanten (eng. \textit{edges}) $E_o = \{(v_a, v_b), (v_b, v_c),(v_c, v_a) | \{v_0, ...\} = V_o;(a, b, c) \in I\} $\\
, welche jeweils die Punkte $\mathcal{E}:E_o\mapsto\mathcal{F}^3; \mathcal{E}((v_a, v_b)) = \{v_a + (v_b-v_a)* k; k \in [0,1]\} $ im Raum einnehmen.
\item Flächen (eng. \textit{areas})$ A_o = \{(v_a, v_b, v_c) | \{v_0, ... \} = V(o, t); (a, b, c) \in I\} $,\\
welche jeweils die Punkte $\mathcal{A}:A_o\mapsto\mathcal{F}^3; \mathcal{A}((v_a, v_b, v_c)) = \{v_a + (v_b-v_a)* k + (v_c-v_a)*l; (k+l) \in [0,1]\} $ im Raum einnehmen.
\item Gesamtpunktemenge $G_o = V_o \cup (\bigcup_{e\in E_o} \mathcal{E}(e)) \cup (\bigcup_{a\in A_o} \mathcal{A}(a)) $
\item Praktisch immer gilt: $V_o \subset (\bigcup_{e\in E_o} \mathcal{E}(e)) \subset (\bigcup_{a\in A_o} \mathcal{A}(a)) = G_o \subset K_o$
\item Die Gesamtpunktemenge enthält zu allen Zeiten $t$ mindestens die Objekthülle $\mathcal{H}: \mathcal{F}^3 \mapsto \mathcal{F}^3, \forall o\in OBJ: \mathcal{H}(K_o) \subseteq G_o$ die den eingenommenen Raum des Objekts/Modells vom übrigen Raum durch Flächen abgrenzt.
\end{enumerate}

Vorteile \& Nachteile dieser Darstellung o.B.d.A:
\begin{itemize}
\item [+]Kürzere Iterationslängen im Vergleich zu kompakten Punktmengen: $V_o \ll G_o \ll K_o$
\item [+]Berechnungen durch relativ schnelle klassische Vektorarithmetik, Skalar- , Kreuzprodukte statt Mengenoperationen.
\item [-]Verlust der Information von Innen \& Außenseite am Hüllobjekt.
\item [-]Zum sinnvollen Einsatz von Polygon-Meshes ist die verwendbare Menge an Objekten auf Polytope beschränkt. Eine schwierig darstellbare Objektform sind beispielsweise Ellipsoide ($|V_o|$ geht dann gegen $|G_o|$).
\end{itemize}

Während semantisch von der Simulation die kompakte Repräsentation eines Objektes $K_o$ respektiert werden muss, rechnet diese tatsächlich mit gegebenem $G_o$ welches in den meisten Kontexten genügt.

\subsection{Objektplatzierung}
\label{sec:objects_sim}

Objekte $o\in OBJ$ müssen nun in der Simulation absolut im Worldspace $\mathcal{S}^3$ platziert werden, sollen sonst für Berechnungen aber in relativen Räumen $\mathcal{F}^3$ behandelt werden. Es werden dafür zunächst absolute Beschreibungskriterien für Objekte angelegt.
\begin{itemize}
\item Raumposition zu Beginn eines Ticks $pos : OBJ \times \delta \mapsto \mathcal{S}^3$
\item Ausrichtung zu Beginn eines Ticks $rot : OBJ \times \delta \mapsto \mathcal{F}^3$. Der Vektor $(x, y, z) \in\mathcal{F}^3$ wird für die Drehung um x (Radialmaß) für die Drehung um die x-Achse relativ zum Raum definiert, bzw. y und z analog. Andere etablierte Formate, wie Quaternionen, werden hier nicht verwendet.
\end{itemize}
Die Kontinuitätseigenschaft $\delta_{j1} = \delta_{(j+1)0}$ ermöglicht die Übernahme/Speicherung des Wertes zu Tickbeginn aus dem letzten Tick.

Objekte sind in der Simulation zusätzlich zeitlichen Änderungen unterlegen.
An dieser Stelle wird festgelegt: Während eines Ticks ändern sich diese konstanten zeitlichen Änderungsgrößen nicht und werden daher ebenfalls pro Tick definiert.\\
 Es wird sich hier auf
\begin{enumerate}
\item Geschwindigkeit $v: OBJ \times \delta \mapsto \mathcal{S}^3$  und
\item Winkelgeschwindigkeit $\omega : OBJ \times \delta \mapsto \mathcal{F}^3 $
\end{enumerate}
beschränkt.

Während eines Ticks $\delta_i$ kann demnach eine Transformationsmatrix $Q: OBJ \times \Upsilon_{\delta_i} \mapsto \mathcal{F}^{4\times 4}$ für alle für jedes Objekt zu jedem Zeitpunkt des Ticks angegeben werden, die die einem Objekt via Modell zugeordneten Punkte, seien dies $V_o, K_o$ usw., die relativ zum jeweiligen Objektspace angegeben sind nun in die entsprechende Position im Worldspace transformieren kann. Transformationsmatrizen sind typischerweise im Kontext der Computergrafik und Simulation von 3D-Räumen $4\times 4$ gewählt, um z.B. auch Translation zu realisieren \cite[ch. 4.4.1, p.76]{fourcrossfour}. Wir beschreiben dazu die 
Translationsmatrix, die aus einer Position, und die Rotationsmatrix, die aus einer Rotationsanweisung hervorgeht $Q_{trans}, Q_{rot}:\mathcal{F}^3 \mapsto \mathcal{F}^{4\times 4}$. Beide dieser Funktionen, insbesondere $Q_{trans}$ sind für $\mathcal{F}^3$ anstatt für $\mathcal{S}^3$ definiert, denn es wird erwartet, dass für die Berechnung zu einem relativ nahen Punkt $P \in\mathcal{S}^3$ relativiert und daraufhin in $\mathcal{F}^3$ durch die Funktion $meter$ umgewandelt wird.\\
Sei $o \in OBJ; pos = pos(o, \delta_i); v = v(o, \delta_i); rot = rot(o, \delta_i); \omega = \omega(o, \delta_i)$, dann gilt
$Q(o, t) = Q_{trans}(meter(pos + v * (t-t_0))) * Q_{rot}(rot + \omega * (t - t_0))$.\\
Bei der Relativierung von Objekten ,welche hier als $o_1 - o_0$ denotiert wird, werden die Beschreibungskriterien beider Objekte mit denen eines Objektes $o_0$ jeweils subtrahiert. Die relative Transformationsmatrix 
\begin{align}
R: OBJ^2 \times \Upsilon_{\delta_i} \mapsto \mathcal{F}^{4\times 4}; R(o_1, o_0, t) = Q(o_1 - o_0, t) = \\
Q_{trans}( meter(  pos(o_1, t)-pos(o_0, t) ) + meter(v(o_1, t)-v(o_0, t)) * (t-t_0) ) \\
* Q_{rot}((rot(o_1, t)-rot(o_0, t)) + (\omega(o_1, t)-\omega(o_0, t)) * (t-t_0))= I_4
\end{align}
. Ein Vorteil besteht dabei, dass bei Relativierung eines Objektes zu sich selbst dabei die Identitätsmatrix $I_4 = R(o_0, o_0, t) \forall t$ entsteht. Auf entsprechenden Modellpunkten von $o_0$ muss dann keine Operation ausgeführt werden, da $\forall p\in \mathbb{F}^3: p*I_4=p$. Dadurch kann bei der Berechnung einer Interaktion, bei der u.U.~ viele Modellpunkte Transformiert werden müssen, die Hälfte der Transformationen gespart werden.\\
Zur Vereinfachung wird für alle Punktmengen $X_o$ zu einem Objekt $o$ die Notation $X_{o, t} = X_o * Q(o, t)$ und für relative Transformationen $X_{o_1, t, o_0} = X_{o_1} * R(o_1, o_0, t)$ verwendet, sodass z.B. die Transformierte Menge der Eckpunkte von $o$ zu Tickzeitpunkt $t$ als $V_{o,t}$ beschrieben werden kann.

\subsection{Projektleistung}
Im Rahmen des Projektes wurden Schnittstellen für~\ref{sec:objects_rep}, angelegt um entsprechenden Zugriff auf die Information von Ecken, Kanten und Flächen während Interaktionsroutinen zu erhalten. Außerdem wurden physikalische Entitäten im Code Repräsentiert, die die für die paarweise Kollisionsberechnung benötigten Attribute für Rotation und Rotationsgeschwindigkeit als Erweiterung zur herkömmlichen Entität beinhalten, die diese nicht besitzt. Alle anderen Teile des bisher behandelten Kontext von Zeit, Raum Objektrepräsentation und Objektbewegung enthalten im Rahmens dieses Projektes maximal marginale Neuleistung.


