Technische Repräsentationen von reelen Zahlen $\mathbb{R}$ beschränken sich auf Floating-Point-Datentypen, welche hier im weiteren mit $\mathbb{F} \subset \mathbb{R}$ bezeichnet werden.

Die Repr"asentation stellt sich dabei jedoch Anforderungen verschiedener technologischer Perspektiven:
		\begin{itemize}
			\item graphische Darstellung/Rendering
				F"ur die graphische Darstellung m"ussen bestimmte Formate eingehalten werden. Datenkonfomit"at mit graphischen Bibliotheken erspart umst"andliche Umwandlungen von Repr"asentationen.
			\item physikalische Berechnung
				F"ur die Kollisionsberechnung k"onnen bestimmte Repr"asentationen vorteilhaft sein. Die Repr"asentation versucht m"oglichst statisch Information "uber Objekte bereitzustellen und vermeidet dynamische Nachberechnung von Objekteigenschaften.
			\item menschliche Manipulation
				F"ur Entwicklungszwecke ist es zum Vorteil, wenn Objektrepr"asentationen menschlich lesbar, verst"andlich und nachvollziehbar sind. Dieses Konzept vereinfacht die Erstellung von Tests und die Behebung von Fehlern. Unter diesen Aspekt fällt ebenso die Konformität mit Tooling.
		\end{itemize}

Im Folgenden werden die zu repräsentierenden, angesprochenen Konzepte näher erläutert.

\subsection{Zeit}
Es existieren die durch maschinelle Abtastung diskrete Realzeit der echten Welt $T_r$ und die Simulationszeit $T_s$. Beide Zeiten sind Sequenzen 
\begin{center}
$T:=\langle t_{epoch}, ..., t_{max} \rangle ; t_{epoch}, t_{max}$ die minimal, bzw.~maximal darstellbare Zeit
\end{center} 
von potentiellen Zeitpunkten in einer gegebenen minimalen maschinellen Genauigkeit in Mikrosekunden $\epsilon_t:=10^(-6)s ; T(c) + \epsilon_t = T(c+1)$.
Zwischen den beiden Zeitbasen besteht eine totale, nicht-injektive, surjektive Abbildung $\mathcal{T}:T_r \twoheadrightarrow T_s$.
Um die Kontinuität der Zeit herzustellen wird weiter eine Zeittate $r_t \in \mathbb{F}$ definiert, welche das relative verstreichen der Zeit in der Simulation steuert $$\forall t_{r0},  t_{r1} \in T_r^2 ; t_{diff}=t_{r1}-t_{r0} : \mathcal{T}(t_1) = \mathcal{T}(t_0) + t_{diff}*r_t)$$
Beispiele für die Rate sind $r_t = 1 \Leftrightarrow$ synchron zur Echtzeit, $r_t = 0 \Leftrightarrow$ Simulation ist pausiert/läuft nicht.\\
Wird diese Eigenschaft verletzt, werden die aktuellen Echtzeitanforderungen verletzt.

\subsection{Raum}
Der geforderte Raum ist 3-dimensional. Es werden daher 3-dimensionale Vektoren verwendet um Raumpositionen zu beschreiben.
Positionen $P \in \mathbb{F}^3$ können dafür verwendet werden.\\
Durch die Eigenschaften von Floating-Point-Datentype treten jedoch bei großen Räumen für Positionen mit großer Entfernung um Ursprung auf Grund der Werteverteilung in $\mathbb{F}$ auf.\\
Diese ist dichter, je näher am Ursprung \cite{floatdistribution}.\\
Physikalische Prozesse berechnet auf Basis von Positionen in \mathbb{F}^3 können daher inkonsistent in Abhängigkeit zum Ort im Raum sein.\\
An dieser Stelle soll diese Unabhängigkeit jedoch gefordert werden.\\
Das Problem wird hier durch einen neuen Längendatentypen $$\mathbb{S} : \mathbb{Z} \times \mathbb{F}$$ gelöst, welcher den Raum zunächst gleichmäßig durch $\mathbb{Z}$ aufteilt und indiziert und $\mathbb{F}$ als Offset innerhalb seines Raumteils verwendet. Es wird daher eine Größe der initialen Aufteilung $size_{grid}$ definiert. Die Umrechnung zu Metern $ m: (\mathbb{Z}\times\mathbb{F}) \mapsto \mathbb{R}; m(i, f) = i * size_{grid} + f * size_{grid}$. Typischerweise gilt $f \in \lceil 0;1\rceil$.
\end{enumerate}

Positionen und Richtungen im Raum werden demnach mit Vektoren $s\in\mathbb{S}^3$ dargestellt. Für Berechnungsvorgänge werden Positionen zunächst relativiert, dann in $\mathbb{F}^3$ in der Einheit Meter umgewandelt, um darauf Berechnungen durchzuführen.

Implementierungstechnisch bestehen verschiedene Räume je nach Anwendungsfall, in denen Objekte durch Typumwandlung, Relativierung und Transformation dargestellt werden.

\begin{enumerate}
\item Worldspace, in \mathbb{S}^3 zur Darstellung von Positionen im Raum
\item Cameraspace, \in{mathbb{F}^3, Ursprung an der Position der Kamera zum Rendern einer Szene, Objekte werden zu Kamera relativiert
\item Objektspace, \in{mathbb{F}^3, Ursprung am Center of Mass eines Objektes, zur Verarbeitung von physikalischen Objektinteraktionen++
\end{enumerate}



\subsection{Objektform}
\label{sec:l0_objects}
Theoretisch können Objekte als eine Menge von Punkten dargstellt werden. Die Menge der Punkte ist dabei begrenzt durch die Form des Objektes und die Art der Darstellung ($\mathbb{S}$, bzw. im Relativen $\mathbb{F}$).
Bei einer zeitlichen Bewegung nimmt das Objekt entlang einer Zeitdimension ebenfalls Positionen ein. Theoretisch ist das Kollisionsproblem also das Problem der Ermittlung des Schnittes zweier Objekte $O_0 \cap O_1$ in 4-dimensionaler Darstellung. Diese Objektdarstellung ist im gegebenen Kontext in der Mathematik üblich, in der Computergraphik und Simulation jedoch auf Grund potentiell unendlich großer Punktemengen suboptimal.\\
In der Computergraphik hat sich daher eine andere Darstellung etabliert:\\
Objekte besitzen eine geometrische Form, welche relativ zu einer zentralen Objektposition angegeben wird. Wir nennen diese zentrale Position an dieser Stelle das \glqq Center of Mass(COM)\grqq . 
Wie in der Computergrafik üblich werden Objekte in Form von Vertices (Ecken) angegeben, welche zu Dreiecken verbunden werden. Dreiecke, Kanten und Ecken werden als Objektmerkmale bezeichnet.\\
Für das Intrusionsproblem im speziellen genügt trivial die Betrachtung der Hülle des Objektes. Es ist üblicherweise auch nur die Hülle des Objektes, die für die Grafik von Interesse ist. Die Hülle stellt also die hier minimal benötigte Objektrepräsentation dar.\\

Im Verlauf dieses Projekts werden Objekte außerdem als starr/rigide, unveränderlich und unzerstörbar angenommen.
	
\subsection{Bewegung}
Entitäten in der 3D-Simulation nehmen eine Position im Raum ein.\\
Physikalische 3D-Objekte sind Entitäten und haben zusätzlich eine Ausrichtung im Raum (Rotation).\\
Beide dieser Größen (Position und Rotation) können einer zeitlichen Änderung unterliegen (Geschwindigkeit, Drehgeschwindigkeit).\\
An dieser Stelle wird festgelegt: Während eines Ticks ändern sich diese konstanten zeitlichen Änderungsgrößen nicht. Vertexpositionen zu bestimmten Zeiten können durch Matrixtransformationen errechnet werden.
Andere Arten der Bewegung (wie z.B. durch Animation oder andere Transformationen wie Skalierung) sind in der Zukunft des Projektes denkbar, allerdings nicht Teil dieses Projektes.


