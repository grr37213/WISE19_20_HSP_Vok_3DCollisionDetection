Implementierungstechnisch schränkt eine Rechenmaschine die mathematischen Zahlenräume ein:
\begin{itemize}
\item Die Reelen Zahlen $\mathbb{R}$ beschränken sich auf Floating-Point-Datentypen, welche hier im weiteren mit $\mathcal{F} \subset \mathbb{R}$ bezeichnet werden
\item Integern $\mathbb{Z}$ sind maschinell in ihrer Darstellungsgröße beschränkt. Diese beschränkte Menge an Integern wird $\mathcal{I} \subset \mathbb{Z}$ genannt
\end{itemize}


\subsection{Zeit}
\label{sec:time}
\def\finite#1{\ooalign{\hfil$\mapstochar\mkern 3mu\mapstochar\mkern 5mu$\hfil\cr$#1$}}

Die Simulation läuft gezeitet ab. Es existieren dabei zwei relevante zeitliche Sequenzen:
\begin{enumerate}
\item Die durch maschinelle Abtastung diskrete Realzeit der echten Welt\\
	\begin{itemize}
	\item $T_r:=\langle t_{epoch}, ... , t_{max}\rangle ; t_{epoch}, t_{max}$ als minimal, bzw.~maximal darstellbare Zeit
	\item die in einer maschinellen Genauigkeit in Mikrosekunden $$\epsilon_t:=10^{-6}s ; \forall c \in \mathbb{Z}: T_r(c) + \epsilon_t = T_r(c+1)$$ gegeben ist
	\item und immer streng monoton wächst $\forall c \in \mathbb{Z}: T_r(c) < T_r(c+1)$. 
	\end{itemize}

\item Die Simulationszeit $T_s:=\langle t_{start}, ... , t_{end}\rangle; t_{start}, t_{end}$ als Start- und Endzeitpunkt der Simulation.
	\begin{itemize}
	\item Zwischen den beiden Zeitbasen besteht eine totale, nicht-injektive, surjektive Abbildung $\mathcal{T}:T_r \twoheadrightarrow T_s$
	\item Die Simulationszeit ist dadurch relativ zur Realzeit definiert $T_s:=\langle\mathcal{T}\rangle$
	\item Um die Kontinuität der Zeit herzustellen wird weiter eine Zeitrate $r_t:T_r\mapsto\mathcal{F}$ definiert, welche das relative verstreichen der Zeit in der Simulation steuert.
	$$\forall t_{r0},  t_{r1} \in T_r ; t_{diff}=t_{r1}-t_{r0} :\mathcal{T}(t_{r1}) = \mathcal{T}(t_{r0}) + t_{diff}*r_t(t_{r1})$$ unter der Vorraussetzung, dass die Rate während aller Zeiten zwischen $t_{r0}$ und $t_{r1}$ gleich bleibt $\forall t \in [ t_{r0},t_{r1}]: r_t( t_{r0}) = r_t(t)$. Wird die Eigenschaft des Zeitflusses in der festgelegten Rate verletzt, werden die aktuellen Echtzeitanforderungen verletzt. \\
	Soll die Rate also geändert werden muss dies zu festgelegten Umschaltpunkten geschehen, welche die Simulationszeit in Zeitbereiche trennen, zwischen denen keine Berechnungsvorgänge zuverlässig durchführbar sind.\\
Beispiele für Raten sind 
\begin{itemize}
\item $r_t(x) = 1 \Leftrightarrow$ Simulation synchron zur Realzeit
\item $r_t(x) = 0 \Leftrightarrow$ Simulation ist pausiert/läuft nicht
\end{itemize}
Technisch wird für große $|r_t|$ die Simulation schwierig, da viele Vorgänge schnell simuliert werden müssten. Diese werden daher vermieden.\\
Theoretisch kann die Rate auch negative Werte annehmen. Die Simulationszeit würde dann rückwärts laufen. Dieses Verhalten ist technisch durch die monoton steigende Realzeit nicht leicht in konsistenter Weise umzusetzen, da Realzeiteinflüsse durch Tasteneingaben existieren und soll daher hier ebenfalls vermieden werden.
	\end{itemize}
\end{enumerate}	


\subsection{Tick \& Frame}

Die Simulation behandelt das Verstreichen von Zeit in Zeitschritten, während dem der interne Zustand der Simulation, bzw.~der simulierten Objekte, zu einem zeitlich neuen Zustand aktualisiert wird.
Dieser Zeitschritt, bzw.~Verarbeitungsschritt, wird oft als Tick bezeichnet.\\

Es ist besonders anzumerken, das der Begriff des Ticks sich ausschließlich auf das Voranschreiten der Simulation bezieht und nicht dem Anzeigen einer Szene. Die Äquivalente Bezeichnung im Kontext der Grafik wird als Frame bezeichnet, in welchem eine Szene (ein Grafischer Zustand der Simulation) gerendert wird. Es besteht Verwechslungsgefahr. Von beiden Größen können Raten $r_{tick}, r_{frame}$ angegeben werden (üblicherweise in Tick/Frame pro Realzeitsekunde). Praktisch kann eine Grafikengine durch Inter- oder Extrapolation dynamisch verschiedene, von der Tickrate unabhängige Frameraten erreichen.\\
Wir definieren die Menge der Ticks $\delta:T_r^2; \delta:=\{\delta_1, \delta_2, ...\}$ anhand ihrer Start- und Endzeitpunkte in Echtzeit $\delta_i := (\delta_{i0}, \delta_{i1})$ und erweitert die zu einem Tick gehörenden Zeitpunkte als $\delta_{id}; d \in [0,1]$. Die Inklusivität/Exklusivität muss in bestimmten Berechnungskontexten manchmal angepasst werden um bei sukzessiven Ticks doppelte Behandlungen von Ereignissen zu vermeiden. Diese Einschätzung sei für jeden Kontext individuell zu vollziehen.\\
Es gilt außerdem die Kontinuität der Zeit auch bei Ticks $\delta_{j1} = \delta_{(j+1)0}$, d.h. ein Tick beginnt am Endzeitpunkt des Vorherigen.\\
Durch die Abbildung $\mathcal{T}$ erhält der Tick eine Entsprechung in Simulationszeit.\\
Ist im aktuellen Kontext nur ein Tick $\delta_i$ von Belang, wird auch die Terminologie $t_d =\mathcal{T}(\delta_{id})$, also $t_0$ für den Tickbeginn und $t_1$ für dessen Ende in Simulationszeit, verwendet.\\
Man kann weiter die Sequenz $\Upsilon_{\delta i} = \langle t_0, ...,  t_1\rangle$ als die zusammenhängende Partition der Simulationszeitsequenz $T_s$ denotieren, welche die geordneten Zeitpunkte eines Ticks in Simulationszeit enthält.\\
Die in Abschnitt~\ref{sec:time} beschriebenen möglichen Umschaltzeiten zur Änderung der Zeitflussrate in der Simulation werden auf die Grenzen von Ticks gelegt.\\
Die Größe der Zeitdifferenz $t_1 - t_0$ unterliegt meist Einschränkungen. Bestimmte Simulationsalgorithmen wie z.B. die Methode der kleinen Schritte erfordern für eine bestimmte Genauigkeit eine maximale Schrittgröße. Die verfügbare Rechenleistung hingegen beschränkt die Tickrate nach oben. Reicht die Berechnungszeit während eine Ticks nicht um den Status der Simulation von $t_0$ auf $t_1$ zu aktualisieren, läuft die Simulation langsamer als die reale Zeit. Die Echtzeitanforderung ist dann verletzt. Oft wird die Tickrate als Konstante festgelegt, in diesem Projekt ist jedoch nur eine Mindestrate festgelegt.\\
Das Konzept eines Ticks und dessen Mindestrate war schon vor dem Projekt in der Codebasis enthalten, die Implementierung erfuhr jedoch einige Anpassungen.


\subsection{Raum}
\label{sec:space}
Der geforderte 3D Raum kann durch 3-dimensionale Vektoren $\in \mathcal{F}^3$ beschrieben werden.\\
Durch die Werteverteilung in $\mathcal{F}$ treten jedoch bei großen Räumen für Positionen mit großer Entfernung zum Ursprung Genauigkeitsdefizite auf, die zur Verletzung von Genauigkeitsanforderung führen können. Mögliche Floating-Point-Werte liegen dabei dichter beieinander, je näher am Ursprung \cite{floatdistribution}. Ein Beispiel für die Auswirkungen dieses Sachverhalts in Simulationen kann in der Quelle \cite{floatdistributionexample} betrachtet werden.\\
Physikalische Prozesse berechnet auf Basis von Positionen in $\mathcal{F}^3$ können daher inkonsistent in Abhängigkeit zum Ort im Raum sein.\\
Das Problem wird hier durch einen neuen Längendatentypen $\mathcal{S} : \mathcal{I} \times \mathcal{F}$ gelöst, welcher den Raum zunächst gleichmäßig durch $\mathcal{I}$ aufteilt und indiziert und $\mathcal{F}$ als Offset innerhalb seines Raumteils verwendet. Es wird daher eine Größe der initialen Aufteilung $size_{grid}$ definiert.\\
Die Umrechnung zu Metern ist dann: $$ meter: \mathcal{S} \mapsto \mathcal{F};  meter((i, f)) = i * size_{grid} + f * size_{grid}$$ 
Typischerweise gilt $f \in [0;1[$, um eine eindeutige Repräsentation für einen beschriebenen Punkt zu erhalten.

Diese Darstellung hat folgende weitere Vorteile
\begin{itemize}
\item Einfache Implementierung
\item Schnelle Indizierung der durch $\mathcal{I}$ indizierten Raumanteile für raumaufteilende Teile-und-Herrsche-Algorithmen
\end{itemize}

Absolute Positionen im Raum werden demnach mit Vektoren $s\in\mathcal{S}^3$ dargestellt. Für Berechnungen von Interaktionen zwischen Objekten werden Positionen zunächst relativiert, d.h. Positionen $p \in \mathcal{S}$ sollen zu $p_0$ relativ gesetzt werden, dann sind die relativen Positionen $p' = p - p_0$. Diese werden dann in $\mathcal{F}^3$ in der Einheit Meter umgewandelt, um darauf Berechnungen durchzuführen. Man geht dabei davon aus, das relative Strecken zwischen Objekten kurz genug sind, sodass die Genauigkeitsänderung in $\mathcal{F}$ vernachlässigbar ist.\\
Effektiv ist dabei die Eigenschaft $\mathcal{F}\subset\mathcal{S}$ nicht gefordert, auch wenn sie in der in diesem Projekt verwendeten Implementierung prinzipiell gilt.

Implementierungstechnisch bestehen verschiedene Räume je nach Anwendungsfall, in denen Objekte durch Relativierung, Längenumrechnung und Transformation dargestellt werden.

\begin{enumerate}
\item Worldspace $= \mathcal{S}^3$ absolute Positionen von Objekten
\item Cameraspace $= \mathcal{F}^3$, Ursprung an der Position der Kamera zum Rendern einer Szene, Objekte werden zur Kamera relativiert.
\item Viewspace $= \mathcal{F}^3$, Verzerrung durch die Kameralinse, um einen Blickwinkel auf einen Bildschirm anzupassen.
\item Objektspace,$= \mathcal{F}^3$, Ursprung am vom Modell definierten Mittelpunkt eines Objektes (Massenmittelpunkt), zur Verarbeitung von physikalischen Objektinteraktionen wird ein Objekt zu einem anderen Objekt relativiert.
\end{enumerate}

Auf diese Weise gelingt es selbst extreme absolute Entfernungen und Geschwindigkeiten im relativen akkurat zu behandeln.


\subsection{Objektrepräsentation}
\label{sec:objects_rep}
Objekte $o\in OBJ$ werden mathematisch oft als eine Menge von Punkten betrachtet, welche einen Raumanteil darstellen $ K_o \subseteq \mathcal{S}$.\\
Durch die in Abschnitt~\ref{sec:space} angesprochene Relativierung ist ein Objekt jedoch in allen Berechnungskontexten (nicht Darstellungskontexten) durch die Relativierung definiert. Ein Objekt enthält folgende Informationen:
\begin{itemize}
\item Raumposition zu bestimmter Zeit $pos : OBJ \times T_s \mapsto \mathcal{S}^3$
\item Ausrichtung zu bestimmter Zeit $rot : OBJ \times T_s \mapsto \mathcal{F}^3$. Der Vektor $(x, y, z) \in\mathcal{F}^3$ wird für die Drehung um x (Radialmaß) für die Drehung um die x-Achse relativ zum Raum definiert, bzw. y für y-Achse und z um z-Achse. Es wird eine Rotationsmatrix $rotm: \mathcal{F}^3 \mapsto \mathcal{F}^{4\times 4}$ definiert, welche Positionen relativ zum Ursprung um diesen nach der Rotationsanweisung dreht. Andere etablierte Formate, wie Quaternionen, werden hier nicht verwendet.
\item Ein relativ zum einem eigenen Ursprung angegebenes rigides unveränderliches Modell.\\
\sloppy
Im Kontext der Mathematik würde das Modell wieder als Punktmenge $$ MODEL : OBJ \mapsto \mathcal{P}(\mathcal{F}^3) $$ definiert sein.
\fussy
Durch diese mathematische Darstellung führen Rechenoperationen mit Objekten auf relativ rechenaufwändige Mengenoperationen zurück. In der Computergraphik werden deshalb Objekte durch sogenannte Polygon-Meshes dargestellt. 
Polygon-Mesh $$M: (V \times I); V \subseteq \mathcal{F}^3, I \subseteq [0, |V|-1]_\mathbb{Z}^3 )$$ beschreibt ein Objekt durch zusammenhängende Dreiecksflächen. In der Repräsentation werden dabei nur die Dreieckspunkte $V$(Ecken, eng. \textit{vertices}) und die Kombinationen dieser zu Dreiecken durch Indizierung $I$ gegeben.\\

\end{itemize}  
Die in der Simulation verwendeten Meshobjekte $O$ können demnach wie folgt beschrieben werden:
\begin{enumerate}
\item $ o:=(POS, ROT, M); POS \in dom(pos); ROT \in dom(rot) $
\item Ecken zur Zeit $t$: $V:OBJ\times T_s\mapsto\mathcal{F}^3; V(Oo, t) = \{s + (f * rotm(rot(o, t))\\ | s = pos(o, t), f\in V\} = \{v_0, v_1 ... , v_{|V|-1}\}$
\item Kanten (eng. \textit{edges}) zur Zeit $t$: $ E:OBJ\times T_s\mapsto V^2; E(o, t) = \{(v_a, v_b), (v_b, v_c),(v_c, v_a) | \{v_0, ...\} = V(o, t);(a, b, c) \in I\} $\\
, welche die Punkte $E_p:ran(E)\mapsto\mathcal{F}^3; E_p((v_a, v_b)) = \{v_a + (v_b-v_a)* k; k \in [0,1]\} $ im Raum einnehmen.
\item Flächen (eng. \textit{areas}) zur Zeit $t$: $ A:OBJ\times T_s\mapsto V^3; A = \{(v_a, v_b, v_c) | \{v_0, ... \} = V(o, t); (a, b, c) \in I\} $,\\
welche jeweils die Punkte $A_p:ran(A)\mapsto\mathcal{F}^3; A_p((v_a, v_b, v_c)) = \{v_a + (v_b-v_a)* k + (v_c-v_a)*l; (k+l) \in [0,1]\} $ im Raum einnehmen.
\item Die Gesamtpunktemenge $D$, die zum Zeitpunkt $t$ vom Objekt $O$ eingenommen wird ist dann:
$ D: OBJ\times T_s: \mathcal{F}^3; D(o, t) = V(o, t)\cup \{E_p(e)|e \in E(o, t)\} \cup \{A_p(a) |a \in A(o, t)\}$. Dabei sind jeweils die Punkte von Ecken in Kanten und von Kanten in Flächen jeweils enthalten $ V(o, t) \subset \{E_p(e)|e \in E(o, t)\} \subset \{A_p(a) |a \in A(o, t)\} \Rightarrow D(o, t) =  \{E_p(e)|e \in E(o, t)\} \cup \{A_p(a) |a \in A(o, t)\} = \{A_p(a) |a \in A(o, t)\}$
\end{enumerate}
Die Gesamtpunktemenge enthält zu allen Zeiten $t$ mindestens die Objekthülle $\mathcal{H}: \mathcal{F}^3 \mapsto \mathcal{F}^3, \forall o\in OBJ: \mathcal{H}(D(o, t)) \subseteq D(o, t)$ die den eingenommenen Raum des Objekts/Modells vom übrigen Raum durch Flächen abgrenzt.

Vorteile \& Nachteile dieser Darstellung o.B.d.A:
\begin{itemize}
\item [+]Kürzere Iterationslängen im Vergleich zu Punktmengen: $V\ll K_o$
\item [+]Berechnungen durch klassische Geometrie.
\item [-]Verlust der Information von Innen \& Außenseite am Hüllobjekt.
\item [-]Beschränkung auf durch Polygon-Meshes darstellbare Objekte $ran(D)\subseteq OBJ$
\end{itemize}

Im Verlauf dieses Projekts werden Objekte außerdem als starr/rigide, unveränderlich und unzerstörbar angenommen.

\subsection{Objektbewegung}
\label{sec:objects_mov}
Objekte sind in der Simulation zusätzlich zeitlichen Änderungen unterlegen. Es wird sich hier auf
\begin{enumerate}
\item Geschwindigkeit $v: OBJ \times T_s \mapsto \mathcal{S}^3$  und
\item Winkelgeschwindigkeit $\omega : OBJ \times T_s \mapsto \mathcal{F}^3 $
\end{enumerate}
beschränkt.
An dieser Stelle wird festgelegt: Während eines Ticks ändern sich diese konstanten zeitlichen Änderungsgrößen nicht $\forall t \in ]t_0, t_1] : v(\mathcal{T}(t)) = v(\mathcal{T}( t_1 )); \omega(\mathcal{T}(t)) = \omega(\mathcal{T}(t_1));$.\\
Im vorigen Abschnitt wurden die gezeitete Position und Rotation zu allen Zeiten $T_s$ definiert.
Es werden jedoch tatsächlich nur Position $pos(o, \mathcal{T}(t_0))$ und Rotation $rot(o, \mathcal{T}(t_0))$ zu Beginn eines Ticks $\delta_i$ festgehalten. Diese Zählen zum Objektstatus, der während eines Ticks von $t_0$ auf $t_1$ aktualisiert werden muss. Dadurch, dass sich die zeitlichen Änderungsgrößen während eines Ticks nicht ändern, können Statusattribute des Objekts während des Ticks(Position, Rotation, Eckpunkte, ...) durch ihre Linearität leicht bestimmt werden.  

Andere Arten der Bewegung (wie z.B. durch Animation oder andere Transformationen wie Skalierung) sind in der Zukunft des Projektes denkbar, allerdings nicht Teil dieses Projektes.

\subsection{Projektleistung}
Im Rahmen des Projektes wurden Schnittstellen für~\ref{sec:objects_rep}, angelegt um entsprechenden Zugriff auf die Information von Ecken, Kanten und Flächen während Interaktionsroutinen zu erhalten. Außerdem wurden physikalische Entitäten im Code Repräsentiert, die die für die paarweise Kollisionsberechnung benötigten Attribute für Rotation und Rotationsgeschwindigkeit als Erweiterung zur herkömmlichen Entität beinhalten, die diese nicht besitzt. Alle anderen Teile des bisher behandelten Kontext von Zeit, Raum Objektrepräsentation und Objektbewegung enthalten im Rahmens dieses Projektes maximal marginale Neuleistung.


