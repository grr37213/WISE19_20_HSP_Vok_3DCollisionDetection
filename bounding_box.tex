Die Bounding-Box (BB) um ein 3-dimensionales Objekt ist ein Quader mit besonderen Eigenschaften in Bezug auf das Objekt. Die Box schließt das Objekt vollständig ein ($Objekt \subseteq BB$)\\
Bounding-Boxen können selbst als Approximation für ein Objekt angesehen werden.\\
\subsubsection{Axis-Alligned-Bounding-Box}
Eine Spezialform der Bounding-Box ist die Axis-Alligned-Bounding-Box (AABB). Alle Kanten einer AABB sind achsenparallel zu den Koordinatenachsen des Koordinatensystems.\\
Die AABB hat folgende weitere relevante Eigenschaften:
\begin{itemize}
	\item kleine Datenrepresentation möglich\\
		Sei ein Objekt $O$ als Ansammlung der Positionen gegeben, welche es im Raum $\mathbb{R}^3$ einnimmt, so kann die AABB durch zwei d-dimensionale Vektoren $(v_{min}, v_{max}) \in (\mathbb{R}^{d}\times\mathbb{R}^{d})$ dargestellt werden. In ihnen werden Minimal- und Maximalpositionen der AABB festgehalten.
	\item Schnelle Kollisionsüberprüfung zwischen AABBs durch Vergleiche der Extrema (Komplexität $= \mathcal{O}(1)$)
	\item Ermittlung einer minimalen AABB für ein Objekt durch Suche der Minima/Maxima (Komplexität $\le \mathcal{O}(n); n $ ist die Anzahl der Objektmerkmale (z.B. Ecken bei Polygon-Meshes))\\
		Die AABB muss sich an das eingeschlossene Objekt bei Bewegung anpassen. Standardmäßig durch Neuermittlung der AABB($\mathcal{O}(n)$). Optimierungen für verschiedene Arten von Bewegung möglich (Positionsänderung, Skalierung, etc.), aber manchmal schwierig (z.B. bei Rotation).
\end{itemize}

Im Fall von dem Spiel Minecraft werden AABBs als Hitboxen verwendet (vgl. \ref{fig:mwhitbox, fig:mphitbox}), welche jedoch scheinbar dem Kriterium $Objekt \subseteq BB$ zuwiderlaufen. Es muss an dieser Stelle zwischen der mathematischen Korrektheit einer Bounding-Box gegenüber einem gegebenen physikalischen Modell und der Designentscheidung gemacht werden, dass das sichtbare Modell nicht die Grundlage des physikalischen Modells ist. In Minecraft ist die AABB das physikalische Modell ($Objekt = AABB$) und hat keinen Bezug zum grafisch sichbaren Modell. Das Bounding-Box-Kriterium ist damit theoretisch erfüllt. Die Designentscheidung selbst soll an dieser Stelle nicht eingeschätzt werden.

