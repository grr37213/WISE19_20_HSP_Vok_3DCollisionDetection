Ein Bounding-Volume zu einem Objekt $o$ ist eine kompakte Menge $B_o \supset K_{o}$. $B_o$ kann als Hitbox fungieren.\\
Eine Bounding-Box ist ein spezielles Bounding-Volume in Form eines Quaders.\\
Eine in diesem Projekt extensiv verwendete, tiefere Spezialform der Bounding-Box ist die Axis-Alligned-Bounding-Box (AABB). Alle Kanten dieser Bounding-Box sind achsenparallel zu den Koordinatenachsen $\{(1,0,0), (0,1,0), (0,0,1)\}$ des 3D-Koordinatensystems.\\
Hier relevante Eigenschaften dieser sind:
\begin{itemize}
	\item kleine Datenrepresentation $AABB := (AABB_{o, min}, AABB) \in \mathcal{S}^{3^2}$ möglich.
		 In ihnen werden Minimal- und Maximalpositionen der AABB festgehalten.
	\item Schnelle Kollisionsüberprüfung zwischen AABBs durch Vergleiche der Extrema (Komplexität $= \mathcal{O}(1)$)
	\item Ermittlung einer minimalen AABB für ein Objekt durch Suche der Minima/Maxima (Komplexität $\le \mathcal{O}(n); n $ ist die Anzahl der Objektmerkmale (z.B. Ecken bei Polygon-Meshes))\\
		Die AABB muss sich an das eingeschlossene Objekt bei Bewegung anpassen. Standardmäßig durch Neuermittlung der AABB($\mathcal{O}(n)$). Optimierungen für verschiedene Arten von Bewegung möglich (Positionsänderung, Skalierung, etc.), aber manchmal schwierig (z.B. bei Rotation).
\end{itemize}

Im Falle des Spiels Minecraft werden AABBs als finale Hitboxen verwendet (vgl. \ref{fig:mwhitbox, fig:mphitbox}), welche jedoch scheinbar dem Kriterium $K_o \subseteq B_o$ zuwiderlaufen. Es muss an dieser Stelle zwischen der mathematischen Korrektheit einer Bounding-Box gegenüber einem gegebenen physikalischen Modell und der Designentscheidung gemacht werden, dass das sichtbare Modell nicht oder nur marginal die Grundlage des physikalischen Modells ist. In Minecraft ist die AABB die finale Hitbox  $H_{o, [min]} = K_o$ und definiert dadurch das Modell $K_o$. Das Bounding-Box-Kriterium ist damit theoretisch erfüllt. Die Designentscheidung selbst soll an dieser Stelle nicht eingeschätzt werden.

