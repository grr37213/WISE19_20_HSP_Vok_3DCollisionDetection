Der Begriff Kollisionserkennung (\textit{eng. collision detection}) scheint, je nach Anwendungsfall, ein spezifisches Problem zu sein und beschreibt eine Sammlung verschiedener Teilprobleme.\\
Wir beschränken uns für die Zwecke dieses Projektes auf Folgende:
\begin{enumerate}
\item Die Ermittlung der Menge von Objektpaaren $C \subseteq OBJ^2$ aus einer Menge von sich bewegenden Objekten $OBJ$ in einem Raum, welche sich innerhalb eines Zeitraums, hier einem Tick $\delta_i$, überschneiden.
$$C = \{(o_0, o_1) | o_0, o_1 \in OBJ; t\in \Upsilon_{\delta_i}; G_{o_0, t} \cap G_{o_1, t} \neq \emptyset\}$$
\item Die Ermittlung von Informationen über den Hergang einer Kollision.
Die genauen Anforderungen, welche Informationen ermittelt werden müssen, gehen aus den Verwendungszwecken hervor (vgl.~\ref{sec:usages}). 
\end{enumerate}

Während mögliche Anforderungen von Verwendungszwecken zwar betrachtet werden müssen, sollen die Verwendungszwecke selbst in diesem Projekt konkret nicht realisiert werden.

Wir schätzen an dieser Stelle modellgenaue Kollisionen als schwierig ein. In Teilproblem 1 entsteht daraus das Problem einer schnell steigenden Komplexität, da die Anzahl der möglichen Kollisionspaare quadratisch wächst $|C|\leq |OBJ|^2$.\\
Das Teilproblem 1 wird daher in zwei separaten Schritten behandelt.
\begin{itemize}
\item[1.1] Elegante und effiziente Vorfilterung von Objekten $OBJ$, um die modellgenaue Überprüfung von großen Mengen von Objektpaaren in $C$ zu vermeiden.
\item[1.2] Die abstrakte Behandlung des modellgenauen Kollisionsproblems bei genau zwei Objekten. Dabei werden Algorithmen gewählt, welche die für Teilproblem 2 benötigten Informationen mitliefern.
\end{itemize}

Für beide dieser Schritte sollen im Folgenden Algorithmen beschrieben werden.\\
\\
\label{sec:physical_realism}
Es ist weiter üblicherweise ein Ziel von physikalischen Simulationen einen realen Sachverhalt möglichst akkurat und konform mit dem etablierten physikalischen Verständnis umzusetzen. Diese Anforderungen müssen für den bestehenden Verwendungszweck der Simulation relativiert werden.\\
In Videospielen dient die akkurate Umsetzung nur der Immersion des Konsumenten,~d.h. der Konsument soll der Illusion von absolutem Realismus ausgesetzt sein, während absoluter Realismus aber teilweise sogar nicht ausreichend umgesetzt werden kann. Je nach Umstand ist sogar der Bruch der Illusion in maßen nicht kritisch. Es sollte dem Konsumenten möglich sein mit Hilfe des vorhandenen physikalischen Verständnis ein Verständnis für die Vorgänge der Simulation intuitiv entwickeln zu können.\\
Videospiele sind im Kontext von Simulationen generell sehr vergebend. Zum Vergleich: Simulationen mit zu niedrigen Fehlertoleranzen in z.B.~der Industrierobotik können zu erheblichem Sach- oder gar Personenschaden führen.
Dem Entwickler ist die Aufgabe gestellt, zwischen den Faktoren Performanz, Realismus und Umsetzbarkeit einen Vernünftigen Kompromiss zu finden. Die Unterschiedlichkeit dieser Kompromissfindung kann in~\ref{sec:hitbox} am Beispiel von Minecraft und CSGO betrachtet werden. Realismus ist danach kein ausschlaggebender Faktor für Erfolg.\\

Auch in diesem Projekt müssen die Erwartungen an physikalischen Realismus relativiert werden. Es bestehen Freiheitsgrade in sowohl der Definition der physikalischen Vorgänge als auch der algorithmischen Umsetzung. 
Videospiele, als Kunstform, können so auch surreale Konzepte umsetzen und Echzeitanforderungen können leichter eingehalten werden.

