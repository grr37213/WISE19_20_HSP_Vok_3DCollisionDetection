\label{ref:problem_def}
Ein hoher Prozentsatz von Quellen zum Begriff der Kollisionserkennung (eng. \textit{collision detection}) scheinen sich auf die Kollision von zwei konkreten Objekten oft idealisierter Form zu konzentrieren \cites[ch.12]{fourcrossfour}{gjk}{gdc-physics}.
Im Kontext einer 3D-Simulation ist das Problem jedoch weiter für nahezu beliebige Mengen von Objekten zu lösen.\\
Zu unseren Zwecken soll dieser Umstand in das Problem der Kollisionserkennung mit aufgenommen werden. Wir behandeln demnach die folgenden Teilprobleme:
\begin{enumerate}
\item \label{prob1}Die Ermittlung der Menge von Objektpaaren $C \subseteq \obj^2$ aus einer Menge von sich bewegenden Objekten $\obj$ in ihrer kompakten Darstellung $K_{o,t}$, welche sich innerhalb eines Zeitraums, hier einem Simulationsschritt(Tick), d.h. einer Sequenz von Zeitpunkten $\Upsilon_{\delta_i} = \langle t_0 , ... , t_1 \rangle$ zu irgendeinem Zeitpunkt überschneiden.
$$C = \{(o_0, o_1) | o_0, o_1 \in \obj; t\in \Upsilon_{\delta_i}; K_{o_0, t} \cap K_{o_1, t} \neq \emptyset\}$$
\item \label{prob2}Die Ermittlung von Informationen über den Hergang einer Kollision.
Die genauen Anforderungen, welche Informationen ermittelt werden müssen, gehen aus den Verwendungszwecken hervor (vgl.~\ref{sec:usages}). 
\end{enumerate}

Objektmodelle sollen die geometrische Diversität von Polyedern ausschöpfen können und können sehr komplex werden. Wir schätzen daher Kollisionen auf Basis ihrer exakten geometrischen Modelle berechnungstechnisch als schwierig ein.\\
In Teilproblem \ref{prob1} entsteht daraus das Problem einer schnell steigenden Komplexität, da die Anzahl der möglichen Kollisionspaare quadratisch wächst $|C|\leq |OBJ|^2$ und wird daher in zwei separaten Schritten behandelt:
\begin{itemize}
\item[1.1] Effiziente Vorfilterung von Objektpaaren, bei der die modellgenaue Überprüfung der Kollision zunächst vermieden werden soll.
\item[1.2] Die Behandlung des modellgenauen Kollisionsproblems bei genau zwei Objekten. Dabei werden Algorithmen gewählt, welche die in Teilproblem \ref{prob2} geforderten Informationen mitliefern.
\end{itemize}
