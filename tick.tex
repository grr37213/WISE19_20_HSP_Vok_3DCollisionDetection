
Die Simulation behandelt das Verstreichen von Zeit in Zeitschritten, während dem der interne Zustand der Simulation, bzw.~der simulierten Objekte, zu einem zeitlich neuen Zustand aktualisiert wird.
Dieser Zeitschritt, bzw.~Verarbeitungsschritt, wird oft als Tick bezeichnet.\\

Es ist besonders anzumerken, das der Begriff des Ticks sich ausschließlich auf das Voranschreiten der Simulation bezieht und nicht dem Anzeigen einer Szene. Die Äquivalente Bezeichnung im Kontext der Grafik wird als Frame bezeichnet, in welchem eine Szene (ein Grafischer Zustand der Simulation) gerendert wird. Es besteht Verwechslungsgefahr. Von beiden Größen können Raten $r_{tick}, r_{frame}$ angegeben werden (üblicherweise in Tick/Frame pro Realzeitsekunde). Praktisch kann eine Grafikengine durch Inter- oder Extrapolation dynamisch verschiedene, von der Tickrate unabhängige Frameraten erreichen.\\
Wir definieren die Menge der Ticks $\delta:T_r^2; \delta:=\{\delta_1, \delta_2, ...\}$ anhand ihrer Start- und Endzeitpunkte in Echtzeit $\delta_i := (\delta_{i0}, \delta_{i1})$ und erweitert die zu einem Tick gehörenden Zeitpunkte als $\delta_{id}; d \in [0,1]$. Die Inklusivität/Exklusivität muss in bestimmten Berechnungskontexten manchmal angepasst werden um bei sukzessiven Ticks doppelte Behandlungen von Ereignissen zu vermeiden. Diese Einschätzung sei für jeden Kontext individuell zu vollziehen.\\
Es gilt außerdem die Kontinuität der Zeit auch bei Ticks $\delta_{j1} = \delta_{(j+1)0}$, d.h. ein Tick beginnt am Endzeitpunkt des Vorherigen.\\
Durch die Abbildung $\mathcal{T}$ erhält der Tick eine Entsprechung in Simulationszeit.\\
Ist im aktuellen Kontext nur ein Tick $\delta_i$ von Belang, wird auch die Terminologie $t_d =\mathcal{T}(\delta_{id})$, also $t_0$ für den Tickbeginn und $t_1$ für dessen Ende in Simulationszeit, verwendet.\\
Man kann weiter die Sequenz $\Upsilon_{\delta i} = \langle t_0, ...,  t_1\rangle$ als die zusammenhängende Partition der Simulationszeitsequenz $T_s$ denotieren, welche die geordneten Zeitpunkte eines Ticks in Simulationszeit enthält.\\
Die in Abschnitt~\ref{sec:time} beschriebenen möglichen Umschaltzeiten zur Änderung der Zeitflussrate in der Simulation werden auf die Grenzen von Ticks gelegt.\\
Die Größe der Zeitdifferenz $t_1 - t_0$ unterliegt meist Einschränkungen. Bestimmte Simulationsalgorithmen wie z.B. die Methode der kleinen Schritte erfordern für eine bestimmte Genauigkeit eine maximale Schrittgröße. Die verfügbare Rechenleistung hingegen beschränkt die Tickrate nach oben. Reicht die Berechnungszeit während eine Ticks nicht um den Status der Simulation von $t_0$ auf $t_1$ zu aktualisieren, läuft die Simulation langsamer als die reale Zeit. Die Echtzeitanforderung ist dann verletzt. Oft wird die Tickrate als Konstante festgelegt, in diesem Projekt ist jedoch nur eine Mindestrate festgelegt.
