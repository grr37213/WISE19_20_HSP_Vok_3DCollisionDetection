

Ein Zeitschritt bei der Berechnung einer Simulation wird oft als Tick bezeichnet. Dies sollte nicht mit einem Frame verwechselt werden, was die Berechnung eines Bildes von der Grafikengine darstellt. Von beiden kann eine Rate angegeben werden (pro Sekunde), diese Raten müssen aber nicht übereinstimmen, da die Grafikengine für ein neues Bild auch die vorliegenden Daten inter/extrapolieren kann. \\
Während eines Ticks wird der Status der Simulation vom Zeitpunkt $t_0$ auf den Status der Simulation vom Zeitpunkt $t_1$ geändert. Die Größe der Zeitdifferenz $t_1 - t_0$ unterliegt meist Einschränkungen. Bestimmte Simulationsalgorithmen wie z.B. die Methode der kleinen Schritte erfordert für eine bestimmte Genauigkeit eine maximale Schrittgröße. Die verfügbare Rechenleistung hingegen beschränkt die Tickrate nach oben. Kann die Berechnung nicht schnell genug ausgeführt werden, läuft die Simulation langsamer als die reale Zeit. Die Echtzeitanforderung ist dann verletzt. Oft wird die Tickrate als Konstante festgelegt, in diesem Projekt ist jedoch nur eine Mindestrate festgelegt.
